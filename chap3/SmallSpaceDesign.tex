As part of the design guidelines for Google Glass, Google provides developers with a card layout template, seen in Figure~\ref{GlassDesignStyle}. The different coloured regions are intended for different use. The red area is the main area intended for presenting information in text form with the green squares representing the preferred margins. The thick blue stripe almost at the bottom marks the footer. The footer should hold supplementary information, such as a user name or a timestamp. The blue,  slightly transparent, area to the left is mainly intended for images with associated text being presented to the right. The grey area, seemingly appearing behind all the other coloured areas, represent the entire card, with a size of 640 pixels wide and 360 pixels high~\cite{glassDesignStyle}.

	\begin{figure}[ht!]
		\centering
		\includegraphics[width=110mm]{images/standard-template}
		\caption{Google's design guidelines include a card layout template~\cite{glassDesignStyle}.}
		\label{GlassDesignStyle}
	\end{figure}

Google goes even further in providing developers with guidelines for design of cards. Google provides developers with a set of fixed card layouts. The specific card layouts sets up the necessary margins and leaves to the developers only to input the information to be displayed. Four examples of fixed card layouts can be seen in Figure~\ref{cardLayouts}.

	\begin{figure}[ht!]
		\centering
   		\subfloat[Text layout.]{{\includegraphics[width=70mm]{images/card_text} }}
  	 \qquad
   		\subfloat[Fixed text layout.]{{\includegraphics[width=70mm]{images/card_text_fixed} }}
	\qquad
		\subfloat[Columns layout.]{{\includegraphics[width=70mm]{images/card_columns} }}
   	\qquad
		\subfloat[Title layout.]{{\includegraphics[width=70mm]{images/card_title} }}
   	\qquad
		\caption{Four different standard card layouts~\cite{cardLayout}.}
		\label{cardLayouts}
	\end{figure}

In terms of the information displayed, Google's default typeface family is called ``Roboto''. Google states that Roboto's geometrical forms and open curves makes for a natural reading rhythm.\cite{googleTypefaceRoboto} Roboto is the typeface family used on all of Google's standard layouts, some of which are seen in Figure~\ref{cardLayouts}. Google uses different typfaces from the Roboto typeface family for different texts~\cite{glassDesignStyle}. Roboto Light is most common, with Roboto Regular being used for footnote text and Roboto Thin being used for larger texts, such as titles on the title card layout seen in Figure~\ref{cardLayouts}~(d).

One of the advantages of using Google's default layout is the fact that the text is dynamically resized to fit the card. Dynamically resized text means that the text is only as small as the text needs to be. However, there is a minimum size text may be downsized to. At 32 pixels the text is as small as the text may be and any text that does not fit on the card at that point gets truncated.

Due to the limitation on the amount of text that may be presented on screen at the same time Google have provided developers with guidelines regarding how to present written information on Google Glass~\cite{glassDesignStyle}. The guidelines for writing are five in total and read as follow:

\begin{itemize}
	\item \textbf{Keep it brief.} Be concise, simple and precise. Look for alternatives to long text such as reading the content aloud, showing images or video, or removing features.
	\item \textbf{Keep it simple.} Pretend you're speaking to someone who's smart and competent, but doesn't know technical jargon and may not speak English very well. Use short words, active verbs, and common nouns.
	\item \textbf{Be friendly.} Use contractions. Talk directly to the reader using second person ("you"). If your text doesn't read the way you'd say it in casual conversation, it's probably not the way you should write it.
	\item \textbf{Put the most important thing first.} The first two words (around 11 characters, including spaces) should include at least a taste of the most important information in the string. If they don't, start over. Describe only what's necessary, and no more. Don't try to explain subtle differences. They will be lost on most users.
	\item \textbf{Avoid repetition.} If a significant term gets repeated within a screen or block of text, find a way to use it just once.
\end{itemize}

In terms of other restrictions Google puts on developers of applications for Google Glass the voice commans aer quirte restricted~\cite{googleGlassVoiceCommand}. Although Google provides developer with a easy-to-use API for voice command they are still restrictive as to how it may be used. Voice commands not officially approved by Google may not exist in an application if the application is to be released on MyGlass, which is basically the store from which users may buy their google glass applications.

Only using the approved main voice commands will allow applications to be approved. These approved main voice commands does not, however, include for instance ``next slide''. Unapproved voice commands may be used during development and for seperate releases. Developers may also use the buolt in speech recognition. However, using speech recignition would mean not being able to launch the voice recignition simply saying ``ok glass''.

HOW TO DESIGN VOICE COMMANDS >>> \cite{glassVoiceChecklist}

[TODO Design guidelines Google Glass]




\subsubsection{Glassware Flow Designer}
Google also provides developers with a design tool to help them visualise applications prior to implementation. The design tool, called ``Glassware Flow Designer''~\cite{glasswareFlowDesigner}, allows developers to discover recommended design patterns and to see the flow of their application prior to implementation.

	\begin{figure}[ht!]
		\centering
		\includegraphics[width=110mm]{images/glaswareFlowDesignerScreenshot}
		\caption{The Google Glass Flow Designer TODO FIX ANOTHER IMAGE.}
		\label{GlassDesignStyle}
	\end{figure}