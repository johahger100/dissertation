The application designed and implemented in parallel with this dissertation is a proof-of-concept of an application used for assembling components. The functionality of the application may be divided in to several steps. First the application scans a QR code. The QR code is then decoded and the decoded information from the QR code is used in order to download instructions on the specific product belonging to the QR code which was scanned. The downloaded instructions are sorted into classes and displayed on screen.

The application was designed to be run on Google Glass, however a proof-of-concept of a similar application for smartphones has been designed and implemented as well. The smartphone application provides a point of reference as well as means to evaluate the pros and cons of using Google Glass. The application is designed as a slide view, with each instruction appearing on a separate slide. The slide view design was used so that users may focus on one instruction at a time.

The design of the application also follows the design guidelines provided by Google~\cite{glassDesignPrinciples}. For instance, Google recommends developers to keep their application and the content within the application simple. Google recommend developers of Google Glass applications especially to mind the small screen and as such to keep the information brief, and the most important information first.

Google also provides developers, designing applications for Google Glass, with even more specific recommendations, as Google recommend using the predefined card layouts~\cite{cardLayout}. The predefined card layout sets up the recommended margins and leaves developers only to input the information which should be displayed on the specific card.

The design guidelines for smartphone applications are not as specific, although Google does recommend keeping information simple~\cite{androidDesignPrinciples}. Google also recommend developers to highlight the most important feature of an application. One example of how the guidelines have been taken into consideration when designing the application is how both the Google Glass application and the smartphone application starts in camera view, where the user may scan a QR code. There is no start menu, instead the most important feature, the camera screen, is the first screen the user sees when launching the application.

The application will also be tested, both on Google Glass as well as the smartphones. One test case is text length, as one limitation of Google Glass is the small display. The test will consist of maximising the number of characters that may fit on one slide in the smartphone application, and comparing the result with how many slides the same number of characters would require in the Google Glass application.

Another test compares the distances from the device to the QR code. Are there any differences between the smartphone application and the Google Glass application when the distance to the QR code is altered? The complexity of the QR code is another test that will be performed. Does the complexity of the QR code, which increases with the number of characters encoded, have any impact specific to either the smartphone application or the Google Glass application? The fourth, and final test, to be performed measures the display time. Does the display time differ between the smartphone application and the Google Glass application for different sizes of information?

All of these tests will be performed 30 times for statistical significance~\cite{30sampleSize}, and all of these tests will be performed on Google Glass as well as the smartphones, the Samsung Galaxy SII and the Samsung Galaxy SIII. The reason for using the two specific smartphone models was for one the need of physical devices used for testing, but also the fact that these two smartphones were very prominent on the market when Google Glass was released in 2013. The Samsung Galaxy SII and the Samsung Galaxy SIII have sold a total of 100 million units and are as such good representatives of smartphones comparable to Google Glass~\cite{samsungGalaxyS2Sales, samsungGalaxyS3Sales}.


%o   Design - Present your project design in general\\
%o   Information - Give details here (possibly several sub-sections)