\begin{itemize}
	\item Google Glass
	\item Samsung Galaxy SII
	\item Samsung Galaxy SIII
\end{itemize}

The reason for using these three units was the fact that the testing required physical devices. Since the testing included scanning a QR code from a various distances, physical devices were necessary. The use of the two specific smartphone models (the Samsung Galaxy SII and the Samsung Galaxy SIII) was partially due to availability during testing. However, another reason was the fact that both models were among the most widely distributed smartphone models during the time of Google Glass' release. The Samsung Galaxy SII was released in 2011 and the Samsung Galaxy SIII in 2012, with Google Glass being released in 2013. The Samsung Galaxy SII and the Samsung Galaxy SIII have in total sold 100 million units to date~\cite{samsungGalaxyS2Sales, samsungGalaxyS3Sales}.

Another reason for not using more modern smartphones was due to the Google Glass version used. The Google Glass unit used was the so called ``Explorer Edition 1''. Google Glass Explorer Edition 1 was released in February, 2013~\cite{historyOfGlass}. An updated version, ``Explorer Edition 2'', was released in the summer of 2014~\cite{googleGlassEdition2RAM}. 

A few updates were made to Google Glass with the new version, most noticeably a doubling of available RAM. Google Glass Explorer Edition 1 has only 1 GB RAM~\cite{googleGlassEdition1RAM} where Google Glass Explorer Edition 2 has 2 GB RAM~\cite{googleGlassEdition2RAM}. The Samsung Galaxy SII and the Samsung Galaxy SIII both have 1 GB RAM as well~\cite{samsungGalaxyS2Wiki, samsungGalaxyS3Wiki}. As such using more modern smartphones in testing could be seen as unfair to Google Glass since Google Glass also exists in a more modern version.

The Samsung Galaxy SII and the Samsung Galaxy SIII both run Android as their operating system~\cite{samsungGalaxyS2Wiki, samsungGalaxyS3Wiki}. Since Google Glass' operating system is also Android~\cite{googleGlassWiki}, the development of the application was deemed to be easier and faster, not having to convert specific functionality to iOS or Windows Phone. Comparing the devices would also be more easily done if the operating system were the same. All of the test devices having Android as their operating system also meant that they would all be able to follow Google's design guidelines, as opposed to, for instance, Apple's design guidelines and principles~\cite{iosDesignGuidelines}.

Additional technical specifications regarding the test units can be found in Table~\ref{tab:textUnitsSpecs}.

	\begin{table}[H]%ht!]
    		\caption{Technical specifications of the three test units.} \label{tab:textUnitsSpecs}
		\centering \begin{tabularx}{\textwidth}{l|X|X|X} \hline
		\textbf{Component} & \textbf{Google Glass~\cite{googleGlassWiki}} & \textbf{Samsung Galaxy SII~\cite{samsungGalaxyS2Wiki}} & \textbf{Samsung Galaxy SIII~\cite{samsungGalaxyS3Wiki}} \\ \hline \hline
       
		RAM				&	1 GB	~\cite{googleGlassEdition1RAM}			&	1 GB				&	1 GB		\\ \hline
		CPU				&	OMAP 4430 dual core~\cite{googleGlassCPU}				&	1.2 GHz dual core ARM Cortex A9				&	1.4 GHz quad core Cortex A9		\\ \hline
		Screen Size		&	Equivalent of a 25 inch screen from 2.5 meters~\cite{GlassSpecs}				&	4.3 inches				&	4.8 inches		\\ \hline
		Screen Resolution	&	640 * 360 pixles	&	480*800 pixles				&	720*1280 pixles		\\ \hline
		Operating System	&	Android			&	Android					&	Android				\\ \hline
		
		\end{tabularx}
	\end{table}

% GG CPU - http://www.catwig.com/google-glass-teardown/