\begin{itemize}
	\item Google Glass
	\item Samsung Galaxy SII~\cite{samsungGalaxyS2Wiki}
	\item Samsung Galaxy SIII~\cite{samsungGalaxyS3Wiki}
\end{itemize}

The reason for using these three units were the fact that the testing required physical devices. Since the testing included scanning a QR code from a various distances physical devices were necessary. What led to the two specific smartphone models (Samsung Galaxy SII and Samsung Galaxy SIII) were partially due to availability during testing. However, another reason was the fact that both models were among the most widely distributed during the time of Google Glass' release. Samsung Galaxy SII released in 2011 and Samsung Galaxy SIII in 2012, with Google Glass releasing in 2013. Samsung Galaxy SII and Samsung Galaxy SIII have in total been sold in 100 million units to date~\cite{samsungGalaxyS2Sales, samsungGalaxyS3Sales}.

Another reason for not using more modern smartphones is due to the Google Glass version used. The Google Glass unit used was the so called ´´Explorer Edition 1''. Google Glass Explorer Edition 1 was released in February, 2013~\cite{historyOfGlass}. An updated version, ``Explorer Edition 2'', was released in the summer of 2014~\cite{googleGlassEdition2RAM}. 

A few updates were done to Google Glass with the new version, most noticeably a doubling of available RAM. Google Glass Explorer Edition 1 have only 1 GB RAM~\cite{googleGlassEdition1RAM} where Google Glass Explorer Edition 2 have 2 GB RAM~\cite{googleGlassEdition2RAM}. Samsung Galaxy SII and Samsung Galaxy SIII both have 1 GB RAM as well~\cite{samsungGalaxyS2Wiki, samsungGalaxyS3Wiki}. As such using more modern smartphones in testing could be seen as unfair for Google Glass since Google Glass also exist in a more modern version.