The following aspects of the application are to be tested in order to help determine wether Google Glass is a viable option to use as replacement for an instruction manual when assembling components. Each test will be performed 30 times, for statistical significance~\cite{30sampleSize}.

\subsubsection{Text Length}
Since the Google Glass display is small and limited in space the amount of text that may by displayed on screen is as a result also limited. As such one interesting test case is to see where the limit in text lies. The test consists of trying finding the limit in text length, on both the Google Glass display as well as a smartphone display. The text used should consist of characters distributed in similar fashion to English text~\cite{englishTextStat}.

When the smartphone display has been filled with information, how many slides does the same amount of information make up on Google Glass? If the amount of slides that the Google Glass application must use in order to display all the information is significantly larger than that of the smartphone application, the use of a Google Glass application might not be preferred as the user must slide significantly many more times than when using a smartphone equivalent application.

%\subsubsection{Tap Counter}


\subsubsection{Distance to the QR Code}
Generally, the recommendation regarding at what distance the user should be positioned in relation to the QR code is ten times the size of the QR code~\cite{qrCodeSizeComplexity}. However, different devices will have different delay time before registering the QR code in frame. As such, one test regards the time from start of the camera until Google Glass has registered the QR code. The time is to be compared with that of smartphones.

As seen in Table~\ref{tab:distanceAverage} three different distances will be tested. The size of the QR code will be optimised according to the formula stated above, with the scanning distance set at two decimeter. The reason for testing for other scanning distances, yet with the same QR code size is because most users might not be aware of the optimal scanning distance, as well as to determine wether Google Glass has any advantages when scanning either closer or further away from the QR code.

The size of the QR code is calculated as \[\frac{2}{10} = 0.2~decimeters\].
%SCALING FACTOR:
% http://www.growthvine.com/blogs/kelly-flowers/2011/11/28/optimal-size-for-qr-codes-depend-on-a-few-things

	\begin{table}[ht!]
    		\caption{Average time of registering a QR code with varying distances.} \label{tab:distanceAverage}
		\centering \begin{tabularx}{\textwidth}{l|X|X|X} \hline
		\textbf{Distance (dm)} & \textbf{Google Glass (ms)} & \textbf{Samsung Galaxy SII (ms)} & \textbf{Samsung Galaxy SIII (ms)} \\ \hline \hline
       
		1.0	&	&	&	\\ \hline
		2.0	&	&	&	\\ \hline
		3.0	&	&	&	\\ \hline
		
		\end{tabularx}
	\end{table}

\subsubsection{Complexity of the QR Code}
Depending on the amount of characters encoded by the QR code, the density of the QR code changes. The density of the QR code increases as the number of characters encoded by the QR code increases, where the number of black and white squares increase. As such one interesting test case is where the variable is the density of the QR code, or more specific; the number of characters encoded in the QR code.

The number of characters will vary between 1, 50 and 100. The values were chosen in order to give the results big enough room so that potential difference can be determined while still keeping the number of characters to a realistic minimum. Table~\ref{tab:complexityAverage} shows how the results will be presented, where the results will be the average of 30 test runs.

	\begin{table}[ht!]
    		\caption{Average time of registering a QR code with varying density.} \label{tab:complexityAverage}
		\centering \begin{tabularx}{\textwidth}{l|X|X|X} \hline
		\textbf{Encoded Characters} & \textbf{Google Glass (ms)} & \textbf{Samsung Galaxy SII (ms)} & \textbf{Samsung Galaxy SIII (ms)} \\ \hline \hline
       
		1	&	&	&	\\ \hline
		50	&	&	&	\\ \hline
		100	&	&	&	\\ \hline
		
		\end{tabularx}
	\end{table}

\subsubsection{Display Time}
The speed at which Google Glass registers the QR code is important to wether the device is to prefer over regular smartphones. However, another interesting aspect is how fast downloaded information may be displayed on screen, from the point that the information has been downloaded. As seen in Table~\ref{tab:averageDisplaySpeedGoogleGlass} the test will evaluate three different information sizes, meant to represent three different ways of presenting information. 100 kB represent text, 1 MB represent an image and 10 MB represent video.

	\begin{table}[ht!]
    		\caption{Average display time for Google Glass with varying information size.} \label{tab:averageDisplaySpeedGoogleGlass}
		\centering \begin{tabularx}{\textwidth}{l|X|X|X} \hline
		\textbf{Information Size (Byte)} & \textbf{Google Glass (ms)}  & \textbf{Samsung Galaxy SII (ms)}  & \textbf{Samsung Galaxy SIII (ms)} \\ \hline \hline
       
		100 k	&	&	&	 \\ \hline
		1 M		&	&	&	 \\ \hline
		10 M		&	&	&	 \\ \hline

		\end{tabularx}
	\end{table}


%[TODO Inledande text]


%Text Length - how much text can you fit on the Google Glass screen compared to smartphone? How many slides does a full slide on the smartphone take up on Google Glass?


%Speed - Compare speed between google glass and smartphone, from scanning the qr code till the slide view is ready


%recognizing the qr code - is there a difference between google glass and smartphone? Use different sizes of the qr code, as well different scaled versions. DIfferent complexity. Is there any difference in speed? Does Google Glass recognize them all?


%tap counter - how many taps to start compared to smartphone (widget vs simple app)


%user experience - do they prefer google glass or smartphone?


%background noise? How to test scientifically?






%\subsection{Text Length}
%Since the Google Glass display is small and limited in space the amount of text that may by displayed on screen is as a result also limited. As such one interesting test case is to see where the limit in text lies. The test consists of trying different text lengths and reaching a conclusion on how much text may be displayed. The test also includes using different characters as different characters allocates different amounts of space.

%\subsection{Image Size}
%Similar to text images are also limited to the screen size. However, in terms of images there is a slightly different issue compared to text. Images may be resized to fit the screen. Is there a point where and image is no longer usable as details in the original resolution can no longer be spotted in the resized version? The test consists of using images with different original resolutions and comparing how well details are shown.

%\subsection{Comparing Text and Images}
%todo

%\subsection{Download Speed}
%The download speed is important as users might not want to wait too long for the application to load in the instructions after having scanned the QR code. As such the download speed will be measured for different amounts of data sized, on both the smartphone application as well as the Google Glass application. 

%\subsection{Interaction Delay}
%todo

%\subsection{Background Noise}
%todo

%\subsection{Size of QR Code}
%todo

%\subsection{Complexity of QR Code}
%todo

%\subsection{``Tap Counter''}
%The ``tap counting test'' simply consists of counting the amount of taps a user must perform in order to reach specific destinations. For instance, how many taps must the user perform in order to start the application?

%\subsection{User Experience}
%todo

%\subsection{Multitasking}
%todo

%\subsection{Battery}
%todo

%\subsection{Connected to Mobile Device}
%todo

%\subsection{Overall Personal Opinions}
%todo