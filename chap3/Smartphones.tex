Despite being two different devices, the smartphone and Google Glass, Google's design recommendations does share similarities between the two. For instance, similar to Google's design guidelines for Google Glass Google recommend developers to keep information brief. Google states that applications should not display too much information.

However, in contrast to Google's design guidelines for Google Glass, Google give developers freedom to make their own decisions. ``Deviate with purpose'', as Google states. Google recommend developers to develop applications which are easy and fun to use. Google recommend the use of pictures and icons rather than menus and buttons.

Most of all Google encourage developers to simply help users, but to ultimately let users decide. Applications designed for smartphones should make users feel powerful, as well as remember what the user has done previously.

An example of how developers might make users feel powerful would for instance be by, in a image editor application, give users tools which applies multiple effects at the same time. In terms of remembering what the user did previously an example would be by remembering what the user has entered in to a text field previously, and displaying previous results in a drop-down menu as the user starts typing in the text field, as seen in Figure~%\ref{}

TODO ADD Get to know me image %https://developer.android.com/design/get-started/principles.html

Another design guideline provided by Google is to highlight what is most important in the application. For instance, in a camera application what is most important is the shutter button. As such the shutter button should be most prominent in the application and easy to find, making the application easy to use in terms of the central feature.

TODO ADD Make important things fast %https://developer.android.com/design/get-started/principles.html

% Google ask developers who are designing for smartphones to think of simplicity and clarity. Google put much emphasis on making applications easy to use.

%However, there are some differences. For smartphones Google also recommend developers to keep track of what users have done in the past. Google ask developers to remember the user's input history and customisation in order to make the experience more pleasant for returning users~\cite{androidDesignPrinciples}.

%Google differ in how they want developers to design applications for smartphones and Google Glass respectively. On smartphones, Google are much more open to developers using their own ideas. Google encourage freedom and give more subtle hints of how to design in contrast to the more strict guidelines for Google Glass. For instance, Google want developers to make applications for smartphones fun and easy to use. They recommend consistency and a rewarding application.

[TODO expand and elaborate with more examples]

Compare with smartphones - design guidelines for smartphones, similar to google glass?
