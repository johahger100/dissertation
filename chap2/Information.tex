In 1985 Sture All{\'e}n, professor of computational linguistics, and Einar Selander, honorary doctor at Ume{\aa} University, in their book---Information on Information---defined information after having gone through ``a large number of examples from texts of different kinds''. All{\'e}n and Selander defined information as ``a certain amount of facts or ideas''.\cite{informationDef1} Presenting information can be done in a number of different ways.

\subsubsection{Text}
Text is one of the oldest forms of presenting information, with written text dating back to 4000 years B.C..\cite{cuneiform} Text is also a simple presenting form to use that does not require much high end hardware. Other forms of presenting information require more memory, more computational power and more graphical power. Text also has the advantage that the user can read through text at their own speed. Text does not have any perception of time.
\\
\\
Text does however have the disadvantage of requiring attention. The person reading the text must keep the attention on the text throughout and can not look away in order to receive all the information being presented. Text is also restricted to the language the text has been written in. In order to globalise an information presentation several texts must be written so that users from different nations can read the text. For instance, English was in 2010 only the third most spoken language, behind Mandarin and Spanish.\cite{sprakNe}

\subsubsection{Images}
The advantage of using images as the form of presenting information is that one can show the viewer the information rather than telling the viewer the information. Showing the viewer could potentially mean that more information could be presented within a smaller space than text could achieve. Images also gives the same advantage as text in terms of at what speed the viewer can perceive the information---at any speed. Images, similar to text, does not have a perception of time.
\\
\\
Similar to text though, images require the viewers attention in order to present the information. The viewer can not look away from an image and still receive the information. Another disadvantage with images is the fact that images can be interpreted in different ways. The saying ``a picture is worth a thousand words'' goes both ways. On one hand images may present much information with one single images. On the other hand the information may not be crystal clear and not as clear cut as a describing text may be.
\\
\\
Images may present information in two different way. One way is with photographs. Photographs may present abstract information and visualise what may be difficult to describe only using words. Another way of using images is by presenting information with graphs. Graphs are usually more clear cut in what information they want to present, however graphs may in some cases be an insufficient way of presenting information. Graphs are used to present statistic information and are usually easier than photographs to translate into words. Statistic information, and as such also graphs, can summarise a period in time where as photographs captures a moment. 

\subsubsection{Audio}
Images and text both have the disadvantage of requiring full attention in order for the information to be perceived. Audio solves this problem. With audio os information presentation form the listener can look away and yet still receive the information that is being presented. In other words audio is well suited for multitasking as long as the other task the listener is performing does not involve audio as well.
\\
\\
Audio does however have the disadvantage of not being insusceptible to time. The listener does not possess the same amount of control as he or she does with either text or images. Audio may be paused and rewinded but the fact that audio is still tied to a timeline is a disadvantage. Another disadvantage with audio is that, similar to text, audio is dependent on the language. If a information presentation were to be spread globally several audio files would be required (given that the audio contained spoken words).

\subsubsection{Video}
Since video consist of many images bundled together video gives the same advantages as images in terms of showing the viewer the information instead of telling. Video presents the viewer with images at such speed that the images gives the impression of movement. Video may also include audio. The inclusion of audio gives video all the same advantages as audio. In other words video could potentially give the advantages of two other forms of information presentation.
\\
\\
Similar to audio, video is constantly moving. The viewer are bound to the playback speed of the video. Even though a video may be paused or even rewind the viewer is not in the same amount of full control as with images or text. With text and images the reader (or reader) can deceide the speed at which the information should be perceived for themselves. If the video does not include audio video, similar to images or text requires full attention in order for the information to be perceived.