In 1985, Sture All{\'e}n, professor of computational linguistics, and Einar Selander, honorary doctor at Ume{\aa} University, in their book---Information on Information---defined information after having gone through ``a large number of examples from texts of different kinds''~\cite{informationDef1}. All{\'e}n and Selander defined information as ``a certain amount of facts or ideas''. While defining the concept of information can be done, there are several ways of presenting information.

\newpage
\subsubsection{Text}
Text is one of the oldest forms of presenting information, with written text dating back to 4000 years B.C.~\cite{cuneiform}. Text is also a simple form of presentation that does not require much high end hardware. Other forms of presenting information require more memory, more computational power and more graphical power. Text also has the advantage that users can read through text at their own pace. Text may be viewed independently of time, in contrast to sound and video.

Text does however have the disadvantage of requiring attention. The person reading the text must focus their attention on the text throughout and cannot look away in order to receive all the information being presented. Text is also restricted to the language the text has been written in. Several texts must be written in order to make the text available to people who might not have the native language in the specified country as their first language. For instance, in 2010, 44.7\% of the Spanish speaking population in the United States spoke English less than ``very well''~\cite{spanishUS}.

\subsubsection{Images}
The advantage of using images as the form of presenting information is that one can show the viewer the information rather than telling the viewer the information. Showing the viewer could potentially mean that more information could be presented within a smaller space than text could achieve. Images also give the same advantage as text in terms of at what pace the viewer could perceive the information. Images, similar to text, may be viewed independently of time, in contrast to sound and video.

In a similar way to text, images require the viewers attention in order to understand the information. The viewer cannot look away from an image and still receive the information. Another disadvantage with images is the fact that images can be interpreted in different ways. The saying ``a picture is worth a thousand words'' goes both ways. On one hand images may present much information with one single image. On the other hand the information may not be crystal clear and not as clear cut as a text might be.

Images may also present information in two different ways. One way is with photographs. Photographs may present abstract and/or concrete information and visualise what might be difficult to describe only using words. Another way of using images is by presenting information using graphs. Graphs are usually clearer with regard to the information they present. However, graphs may in some cases be a less useful way of presenting information. Graphs are used to present statistical information and are usually easier than photographs to translate into words. Statistical information, and as such also graphs, can summarise a period in time where as a photograph captures a moment. 

\subsubsection{Audio}
Images and text both share the disadvantage of requiring the user's vision in order for the information to be perceived. Audio solves this problem. While audio does require the user's attention audio uses a different sense than text and images. With audio as the form of presenting information, the listener can look away and yet still receive the information that is being presented. In other words audio is well suited for multitasking as long as the other task the listener is performing does not involve listening to audio as well.

Audio does, however, have the disadvantage of having to be understood in real-time. The listener does not possess the same amount of control as he or she does with either text or images. Audio may be paused and rewound, but the fact that audio is still tied to a timeline is a disadvantage. Another disadvantage with audio is that, similar to text, audio is dependent on the language. If information were to be used on a world-wide basis, several audio files would be required (given that the audio contained spoken words) translated into different languages.

\subsubsection{Video}
Since video consists of many images bundled together, video gives the same advantages as images in terms of showing the viewer the information instead of telling him or her the information. Video presents the viewer with images at such speed that the images give the impression of movement. Video may also include audio. The inclusion of audio gives video all the same advantages as audio, as long as the audio is not dependent on the images (in such a case, the viewer would then not be able to look away from the video). In other words video could potentially give the advantages of two other forms of information presentation.

However, similar to audio, video is presented in real-time. The viewer is bound to the playback speed of the video. Even though a video may be paused or even rewound, the viewer does not possess the same amount of control as with images or text. With text and images the reader/viewer can deceide the pace at which the information should be perceived for themselves. If the video does not include audio, video (similar to images or text) requires full attention in order for the information to be perceived.