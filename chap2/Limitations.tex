%Screen size
%Resolution
%Information on screen (possible result/discussion)
%How about people wearing glasses normally? Will the boss require employees to get lenses?
An early concern with Google Glass came from people who wore regular glasses every day as Google Glass seemed to require their own separate frames. Isabelle Olsson at Google responded on the issue on April 12th 2012 with the following: ``We ideally want Project Glass to work for everyone, and we're experimenting with designs that are meant to be extendable to different types of frames.''.\cite{GoogleGlassFrameResponse}

Today many eyecare providers have been trained for Google Glass and Glass frames. These trained eyecare providers are however mostly located in the United States,\cite{frameProviders} but Google points out that many eyecare providers should be able to help replace the lenses on Google Glass' frame\cite{framesGlass}.

% not relevant reference
%\url{http://www.google.com/glass/help/frames/} 
A more alarming concern has been the health of the eyes. \cite{ackerman13} Concerns regarded eye pain and misalignment of the eyes as Google Glass placed a screen above one eye and not both. Google also saw these potential issues and approached Eli Peli, professor of ophthalmology who had been studying HMDs for two decades, as the development och Google Glass started.

Peli claims that Google Glass has been designed with more safety and comfort in mind than previous, similar products. Peli pointed out that Google Glass is see-through and only covers a small part of the user's eyes. As such Google Glass does not require a potentially poorly adjusted camera to capture the environment and display the environment to the user, which could cause eye pain.

Peli also pointed out that Google Glass is meant only to be used in short bursts. The user should not be starring at the display for long periods of time, which would have potential to cause a misalignment of the eyes. While Peli states that the risk is not zero, he still claims that the likelihood of Google Glass causing any damage is small.

%Eye pain? 
%\url{http://www.forbes.com/sites/eliseackerman/2013/03/04/could-google-glass-hurt-your-eyes-a-harvard-vision-scientist-and-project-glass-advisor-responds/}
%
%documented eye pain from looking at a screen for too long. Also conserns regarding looking at something that not both eyes can see. Can give headache and slighted eye aligntment.\\
%
Even though there, according to Google's expert, might not be any health risks involved, there is still a question of how much help Google Glass may be to users. A study performed in 2002\cite{laramee02}, regarding the effects of HMDs, showed that HMDs may only be of help to users under controlled forms. Whenever the surrounding gets too distracting, for instance within a moving crowd, performance goes down. The study however noted that pilots had been able to successfully turn HMD into something they could use to their advantage. Since the study was not done over a long period of time the participants was potentially not given enough time to get used to wearing and using their HMDs.

%HMD:s could potential be of great service to users as long as users take the time to get use to the HMD device.\cite{laramee02}\\