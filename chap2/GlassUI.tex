Google Glass' graphical user interface (GUI) is called a timeline (see Figure \ref{GoogleGlassUI}). The timeline consists of a row of cards. Cards are basic applications such as a clock or information about the weather. Cards can also represent more in-depth applications, on Google Glass called ``Immersions''. Iimmersions handles activities such as browsing an image gallery or playing a game.\\

	\begin{figure}[ht!]
		\centering
		\includegraphics[width=110mm]{images/GoogleGlassUI}
		\caption{A visual representation of the Google Glass GUI as the GUI is perceived by the user. In reality only one card can be displayed at a time.\cite{ImagesGoogleGlassUI}}
		\label{GoogleGlassUI}
	\end{figure}

The first thing the user sees when starting up Google Glass is the home screen. The home screen displays a clock and also shows the text "ok glass". The home screen is a part of the timeline and acts as the center point. Cards to the left of the home screen are upcoming activities such as an event in the user's calendar or an upcoming flight. Cards to the right of the home screen are from the past. Cards from the past will for instance show text messages or photos.\\

In order to move left on the timeline (forward in time) the user must swipe a finger backwards on the touchpad. In order to move right on the timeline (backward in time) the user must swipe a finger forward on the touchpad. The fact that the user must swipe backwards when stepping forward in time might not seem especially intuitive. In western culture a timeline is normally represented as going from left to right. One example of that are books. However, one might think of this action as swiping cards behind the back. Swiping forward when stepping backwards in time would then in turn mean bringing cards placed behind the back into focus. Cards in the past are behind the user while cards in the future are in front of the user.\\

When the user wants to turn off Google Glass the user swipes down on the touchpad. Swiping down on the touchpad will put Google Glass in stand by. If the user wants to turn off Google Glass entirely, in other words power down the device, there is a power button on the opposite side of the touchpad. Holding down the power button for a few seconds will turn off Google Glass. For a better visual understanding of how Google Glass works see Figure \ref{GoogleGlassUI} as well as the video referenced in the caption.
%The main way for a user to give input to Google Glass is via the touchpanel that is mounted on the right hand side of Glass, along the frame. Users are able to swipe as well as tap, which gives them control similar to that of a Smart TV's user interface. Where with a TV controller the user would maneuver with a simple cross layout (up, down, right and left) the buttons have on Glass been replaced by a touchpanel.\\

% Insert image of Google Glass graphical user interface here!!!

%The graphical interface is displayed at the top right through a projection coming from the right on a thick piece of glass. This technique lines up the image with the users sight but does not give any projection outwards.\\

%The interface is built with cards. Each card represents an activity.\\

%What’s unique?
%Standards?
%\url{https://developers.google.com/glass/design/}