Today there are several products either already on the market or under development that are more or less similar to Google Glass. Following is a brief list describing some of the competition Google Glass faces.

\begin{itemize}
\item \textbf{Microsoft Hololens}\cite{hololens}

Microsofts offer in the augmented reality device space is a HUD that displays information over both of the user's eyes. The intention, according to Microsoft, is not to be an immediate competitor to Google Glass. Microsofts aim is not to make the same device asGoogle Glass. Google Glass are meant to be worn all the time, at all times. Microsoft Hololens is rather a device users only puts on when they intends to use it.\\

But the mot striking difference between Microsoft Hololens and Google Glass lies in the interaction with the real world. Google Glass is a two dimensional display that sits slightly above the users line of sight (see \ref{subsec:googleglass}). Microsoft Hololens on the other hand is meant to interact with the world even further. \\

Microsoft intends to give the user tools to work in a 3D space. Microsofts concept video\cite{hololensConceptVideo} of Microsft Hololens shows examples of 3D modelling with the use of kinetic handmovement detection, meaning that users will be able to see what they are working on from different angles simply by walking around it, just as if the object in question was real and had a physical mass.

\item \textbf{Recon Jet} (HUD for sports)

[TODO]

\item \textbf{GlassUp}\cite{glassUp} (Sued by Google)

GlassUp is an Italian company that received most of its founding through the crowdfunding site Indiegogo.\cite{glassUpIndiegogo} GlassUp have been sued by Google for being to similar to Google Glass [TODO REFERENCE]. GlassUp does however make distinctions between the two products. On GlassUp's Indiegogo page the company made the comparison that looking at Google Glass was like looking in the back view mirror in while GlassUp was like looking out the windscreen.\\

GlassUp display information close to the center of the user's vision where as Google Glass keeps the information on the user's upper right. GlassUp claim that this decision was made so that there would be less strain on the user's eye.

\item \textbf{C Wear Interactive Glasses}\cite{penny}

C Wear Interactive Glasses is an industry focues device developed by Penny in V{\"a}ster{\aa}s, Sweden. It does not feature the same slick design many of the other virtual reality devices have (although many of them look terrible as well). One of the examples of this is that one key user interface is where the user can bite on a stick that is connected to the glasses. Probably because of the loud envornment that surronds most workers.\\

The GUI of Penny have the look of a normal PC application which comes from the fact that Penny keeps connected to a computer. However this might not be the most optimal interface since nagivation comes from head movements.
\end{itemize}

%\url{http://www.microsoft.com/microsoft-hololens/en-us}
%\url{http://www.searchenginejournal.com/google-glass-alternatives/67018/}
%\url{http://www.penny.se}