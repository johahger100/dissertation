\begin{itemize}
\item Microsoft Hololens (Hologram)\\
	Microsofts offer in the augmented reality device space is a heads up display that displays over both of the user's eyes. The intention is, according to Microsoft themselfs, is not to be a direct competitor to Google Glass. Their aim is not to make the same device. Where Google Glass are meant to be worn all the time, at all times, Microsoft Hololens is rather a device the user only puts on when he or she intends to use it.\\
	But the mot striking difference lies in the interaction with the real world. The interface Google Glass displays is a heads up display. It is a two dimensional display that floats slightly above the users line of sight (see \ref{subsec:googleglass}). The interface in Microsoft Hololens on the other hand is meant to interact with the world. \\
	Microsoft intends to give the user hologram and tools to work in a 3D space. Their concept video shows examples of 3D modeling with the use of kinetic handmovement detection. This means that users will be able to see what their working on from different angles simply by walking around it, just as if the object in question was real and had a physical mass.
\item Recon Jet (HUD for sports)
\item GlassUp (Sued by Google)
\item Penny (Västerås)\\
Penny is an idustry focues device developed in Västerås, Sweden. It does not feature the same slick design many of the other virtual reality devices have (although many of them look terrible as well). One of the examples of this is that one key user interface is where the user can bite on a stick that is connected to the glasses. Probably because of the loud envornment that surronds most workers.
The graphical user interface have the look of a normal PC application which comes from the fact that Penny keeps connected to a computer. However this might not be the most optimal interface since nagivation comes from head movements.
\end{itemize}

\url{http://www.microsoft.com/microsoft-hololens/en-us}
\url{http://www.searchenginejournal.com/google-glass-alternatives/67018/}
\url{http://www.penny.se}