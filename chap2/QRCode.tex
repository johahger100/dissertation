As stated in section \ref{subsec:userinterface} Google Glass is equipped with a camera that can be used to take photos from the users perspective. One potential use of the camera... [TODO ADD TRANSITION] The Quick Responce (QR) Code was announced in 1994. Having been under development for several years at Denso Wave, barcode developer, the goal was to create a new form of barcode that could carry more information and be easily read.\cite{qrCodeHistory}
\\
	\begin{figure}[ht!]
		\centering
		\includegraphics[width=110mm]{images/qrcodestandard}
		\caption{The standardised fields in a QR Code.\cite{qrCodeWiki}}
		\label{qrcodestandard}
	\end{figure}
\\	
A conventional barcode is capable of storing approximately 20 digits while a QR code can store several thousand digits.\cite{qrCodeType} Information is encoded using standardised encoding modes and displayed as a 2D barcode. A QR code has several standardised fields, as seen in Figure \ref{qrcodestandard}. Using position fields a qr code can be read from any direction, compared to a conventional barcode which can only be read horizontally.\cite{qrCodeAbout}