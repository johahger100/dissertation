Having used Google Glass about every weekday for nearly four months there are a few comments that can ben made, and a few conclusions that can be drawn, simply from personal user experience. Please note that these comments are not based on any scientific studies, but are rather the opinions of the author of this dissertation.

First of all it should be said that Google Glass is very easy to use. The core features of the device is easy to grasp, such as how to navigate using voice commands, and how to use som of the built-in applications, such as taking a picture or finding the shortest route to a specific location.

However, getting used to wearing Google Glass might take some time, somewhere between a few days up to a week of constant usage. Having a display slightly above the line of sight could become irritating. However, the Google Glass display does not stay on at all times. Instead the display times out after a short period when Google Glass is not being used.

Since Google Glass projects an image on to glass, which is then perceived by the user as the display, Google Glass is see through when not active. As such Google Glass, although noticeable, can be ignored when not active after a few days of usage. As stated previously, simply wearing Google Glass will take a few days getting used to.

The fact that Google Glass uses a BCT instead of regular headphones is appreciated as Google Glass is meant to be worn at all times, and as such users might not want to plug their ears. Doing so would potentially mean users would not be able to hear surrounding sounds, and would have to take off Google Glass when for instance talking to another person.

However, Google Glass does have some serious issues. One major issue is the fact that Google Glass overheats very easily. Simply interacting with an application a little too fast will cause Google Glass to overheat. When overheated the Google Glass touchpad, behind which the CPU and such sits, gets really hot. The heat can be felt by the user as the backside of the Google Glass touchpad lies against the temple, and could potentially be very distracting or even uncomfortable depending on the user.

When overheating, Google Glass will also not run as smoothly and a message will even be displayed on the home screen, which states ``Google Glass must cool down to run smoothly''.

Another downside of Google Glass is the restrictions Google has put on the development side. The fact that voice commands must be approved does make sense in terms of the voice commands which starts an application. Users wants to be able to distinguish one application from another. However, when running an application developers should be able to use the voice commands best suited for the application.

The ``ok glass'' overlay is also not a very good solution as part of the screen becomes more or less useless when the dark overlay cover such a big part of the screen. Developers should be able to better customise the design of applications, and not be restricted to the rules and guidelines Google has decided. The fact that Google wants to make sure that Google Glass applications follows a certain standard might seem like a good idea. However, the restrictions limits developers creativity.

All in all Google Glass feels like a device meant for more casual use at this time. Google Glass is not stable enough to handle more industry focused work, where the workload on the device could be very high. Used as a device which could display simple information, Google Glass seems well suited, but put under pressure and used a lot for longer periods of time is when Google Glass falls short.

% easy to use

% getting used to wearing Glass, takes a few days up to a week of contant usage. display could be irritating, but the point is that it is only lit up while using and the timed out

% the heat

% restrictions

% casual use, perhaps not industry at this time