As the application launches, the first screen the user sees, in both versions, is the camera screen. The user must, in order to proceed further within the application, scan a QR code. Scanning a QR code is done by positioning the camera on the device (either Google Glass or smartphone) such that the QR code can be seen on screen. The user does not need to press any shutter button as the application automatically recognises the QR code pattern if seen on screen.%, as seen in Figure~\ref{}. The reasoning behind 

	\begin{figure}[ht!]
		\centering
		\includegraphics[width=110mm]{images/demo/qrCode}
		\caption{todo bild behöver uppdateras}
		\label{glassDemoQR}
	\end{figure}

The reason for not providing a menu on the start screen was because the application should be simple, easy to use and focus on what is important. Since the the focus of the application is to scan the QR code in order to receive the necessary instructions that is also the main focus of the first screen of the application.

When the QR code has been scanned the application decodes the QR code. The decoding process is done in the same way as described in Section~\ref{subsec:qrcode}. However, the decoding process is handled by the Zebra Crossing (ZXing) library~\cite{zxing}. ZXing is an open source barcode image processing library.

The smartphone application was based directly upon the ZXing library, where as the Google Glass application was based upon a port of the library to Google Glass, called ``BardcodeEye''~\cite{barcodeEye}. The main difference between ZXing and BarcodeEye is the fact that BarcodeEye is a full example application ready to be run, in contrast to the ZXing library which is only a library and as such needs to be attached to a runnable application.

The BarcodeEye application for Google Glass is however a bare bone application, used as an example and introduction as to how ZXing may be implemented in an application for Google Glass. BarcodeEye displayed the decoded information from the QR code and also gave the user the option to search the internet using the information previously decoded from the QR code.

As the QR code was intended to encode only a product ID, and the use the ID to download the instructions, rather than having all of the instructions encoded directly in the QR code the application had to be modified. However, prior to changing where the instructions were coming from the graphical layout of the application was changed. The change of layout was mostly done due to the fact that the application only displayed plain text, not taking in to account for instance a mix of image and text.

However, BarcodeEye also used the now deprecated class ``Card'', as seen in Listing~\ref{listingDeprecated}. Instead the application now uses the ``CardBuilder'' class, as seen in Listing~\ref{listingRecommended}, as recommended by Google~\cite{googleCard}. The CardBuilder class allows users to input a desired layout style as an argument to the constructor of the CardBuilder class.

\begin{lstlisting}[language=Java, caption={Instancing of the deprecated class Card}, label=listingDeprecated]
Card card = new Card(context);
\end{lstlisting}

\begin{lstlisting}[language=Java, caption={Instancing of the recommended class CardBuilder}, label=listingRecommended]
CardBuilder cardBuilder = new CardBuilder(context, CardBuilder.Layout.TITLE);
\end{lstlisting}

Since the smartphone application also used the ZXing library, but without any pre-existing application no changes similar to those done to the Google Glass application had to be done for the smartphone application.



%discuss differences (classes exclusive to the smartphone application and GG application respectivly)

%discuss downloading of product information


The download process also includes creating and initialise an instance of the Products class. The instance contains the name of the product, potentially an image of the product as the product will look when the user is done assembling all the components (the existence of an image is dependent of whether there was an image of the product stored in the database).

The Products class instance will also contain a list of components as well as a list of instructions. Both components and instructions are classes themselves. Similar to the Products class instances of both the Components class and the Instructions class will contain a string and potentially an image. In the case of components the string will contain the name of the component, in contrast to instances of the Instructions class where the string instead will contain the instruction itself.



%discuss sorting into classes

%discuss different layouts

	\begin{figure}[H]%ht!]
		\centering
		\includegraphics[width=110mm]{images/demo/titleCard}
		\caption{The title card of the demo application.}
		\label{glassDemotitleCard}
	\end{figure}
	
	\begin{figure}[H]%ht!]
		\centering
		\includegraphics[width=110mm]{images/demo/componentText}
		\caption{A component slide from the demo application.}
		\label{glassDemoQR}
	\end{figure}
	
	\begin{figure}[H]%ht!]
		\centering
		\includegraphics[width=110mm]{images/demo/instructionImage}
		\caption{An instruction slide from the demo application.}
		\label{glassDemoQR}
	\end{figure}
	
	\begin{figure}[H]%ht!]
		\centering
		\includegraphics[width=110mm]{images/demo/voiceCommand1}
		\caption{The voice command menu in the demo application.}
		\label{glassDemoQR}
	\end{figure}



\subsection{Android Studio}
Both the smartphone application as well as the Google Glass application were developed in Android Studio~\cite{androidStudio}. Android Studio is a development environment developed by Google. Both applications were initially being developed in Eclipse~\cite{eclipse}, however development soon shifted to Android Studio as Android Studio is now the official integrated development environment (IDE) for Android~\cite{androidIDE}. The shift was done without complications as Android Studio contains an import feature enabling developers to import projects previously not developed in Android Studio~\cite{androidIDE}.%[TODO why Android Studio]

%\subsection{ZXing}
%The application was built upon the open-source barcode image processing library, Zebra Crossing (ZXing). 
%[TODO Vilka förändringar har gjorts]
% https://github.com/zxing/zxing/

%\subsubsection{BarcodeEye}
%The Google Glass application was built upon the Google Glass port of the ZXing library, known as BarcodeEye~\cite{barcodeEye}.% [TODO vilka förändringar har gjorts]%While a slideview was implemented in BarcodeEye already, the information displayed was static [todo, var den statisk]. The slideview consited of only two slides [todo code example of how it was static]. Mixing images with text was not possible either. Information also had to be encoded directly into the QR code and could not be downloaded by an encoded link.
% https://github.com/BarcodeEye/BarcodeEye

%\subsection{View Slider}


%\subsection{AsyncTask}

%Used for image, as well as product

%\subsection{Text Split}


\subsection{Card Layout}
Google provides developers with a set of predefined layouts for different types of cards, which were used in the Google Glass application and used as basis for the design of the different layouts for the slides in the smartphone application. The following predefined layouts were used in the implementation: ``Title'', ``Columns'' and ``Text''. The Title layout was used for the first card of the slide view, which shows the product name as well as an image of the product as it is supposed to look when finished. TODO Image card? vilken layout? Använde den också title- layouten?

	\begin{figure}[ht!]
		\centering
    		\subfloat[The title card layout.]{{\includegraphics[width=70mm]{images/demo/titleCard}}}
   		 \qquad
		\subfloat[The column card layout.]{{\includegraphics[width=70mm]{images/demo/columnImage}}}
   		 \qquad
    		\subfloat[The text card layout.]{{\includegraphics[width=70mm]{images/demo/instructionText}}}
    		\qquad
        		\subfloat[The image card layout.]{{\includegraphics[width=70mm]{images/demo/instructionImage}}}
   		 \qquad
		\caption{The different layouts used within the Googla Glass application.}
		\label{fig:cardLayout}
	\end{figure}

The Columns card layout, seen in Figure~\ref{fig:cardLayout}~(b) was used for when an instruction or component was to be presented with both text and an image. Since the Columns layout split the card, with an image to the left and text to the right, the Columns layout was the most reasonable choice when presenting both text and an image. An alternative would have been to display the text on top of the image, the image could potentially have been hidden behind a larger amount of text. 

Such a layout design was instead used for the title card as the amount of text being displayed is only the name of the product, and the image is only to give an idea of what the finished product will look like. The layout design where the text overlapped the image was called Title and can be seen in Figure~\ref{fig:cardLayout}~(a).

If the information being presented, either a component or an instruction, instead were to be presented only as text the Text layout, seen in Figure~\ref{fig:cardLayout}~(c) was used. The Text layout displayed dynamically sized text. In other words, if there was a lot of text being displayed the text would be resized to fit the screen.

TODO IMAGE layout

Using the predefined layouts in the implementation was easily achieved as the process consisted mostly of plug-and-play. The \texttt{CardBuilder} class constructor took the layout as an argument, as seen in Listing~\ref{cardBuilderPlugPlay}. When and instance of the \texttt{CardBuilder} class was created what remained was to simply input the necessary information, such as the instruction text. Setting an image was done slightly differently than written information as images were loaded in using a separate thread. As soon as the \texttt{CardBuilder} method \texttt{getView} was called the card was built with the information that had been inputed.

\begin{lstlisting}[language=Java, caption={Initialisation of the CardBuilder class}, label=cardBuilderPlugPlay]
CardBuilder cardBuilder = new CardBuilder(context, CardBuilder.Layout.COLUMNS)
	.setText(getText())
	.setFootNote(mFootNote)
	.setTimestamp(mTimeStamp);

cardBuilder = (new LoadImage(isTitleCard(), getByteArray()).doInBackground(cardBuilder));

return cardBuilder.getView();
\end{lstlisting}

%which standard layout were used, one non-standard

\subsection{Voice Commands}
The Google Glass application gives users the option to use voice commands in order to navigate the slides. The user opens the voice command menu by saying ``ok glass'' at any point in the application when ``ok glass'' is written at the bottom of the screen. The voice command feature is available at all times except when the camera is active. In other words the voice commands are unavailable when the application is waiting to scan a QR code.

The voice command menu contains the following options.

\begin{itemize}
	\item \textbf{Show next slide}
	
	The application scrolls to the next slide. If the current slide is the last slide, and in other words no other slides are following, the application does nothing.
	\item \textbf{Show previous slide}
	
	The application scrolls to the previous slide. If the current slide is the first slide, and in other words no other slides are sits before it, the application does nothing.
	\item \textbf{Show components}
	
	The application scrolls to the first slide showing information on a component. If the user is currently on the first slide showing information on a component the application does nothing. 
	\item \textbf{Show instructions}
	
	The application scrolls to the first slide showing an instruction. If the user is currently on the first slide showing an instruction the application does nothing.
	\item \textbf{Scan again}
	
	The application launches the camera and expects the user to scan another QR code.
\end{itemize}

Implementing voice command in the Google Glass application is done by todo~\ref{voiceCommandXML}

\begin{lstlisting}[language=XML, caption={The voice command menu XML file}, label=voiceCommandXML]
<menu xmlns:android="http://schemas.android.com/apk/res/android">
	<item
		android:id="@+id/next_menu_item"
		android:title="Show next slide" >
	</item>
	<item
		android:id="@+id/previous_menu_item"
		android:title="Show previous slide" >
	</item>
	<item
		android:id="@+id/components_menu_item"
		android:title="Show components" >
	</item>
	<item
		android:id="@+id/instructions_menu_item"
		android:title="Show instructions" >
	</item>
	<item
		android:id="@+id/scan_menu_item"
		android:title="Scan again" >
	</item>
</menu>
\end{lstlisting}

Although none of the voice commands have been sent in for official approval by Google all of the voice commands follows the design guidelines provided by Google. 

	\begin{table}[ht!]
    		\caption{Voice Command Checklist~\cite{glassVoiceChecklist}.} \label{tab:voiceCommandCheckTableChecked}
		\centering \begin{tabularx}{\textwidth}{l|X|l} \hline
		 & \textbf{Guideline} & \textbf{Acheived} \\ \hline \hline
       
1	&	Is general enough to apply to multiple Glassware, but still has a clear purpose		&	Yes		\\ \hline
2	&	Is colloquial and can explain Glass features in a conversation					&	Yes		\\ \hline
3	&	Is comfortable to say in public											&	Yes		\\ \hline
4	&	Brings the user from intent to action as quickly as possible					&	Yes		\\ \hline
5	&	Avoids brand words													&	Yes		\\ \hline
6	&	Is long enough to ensure high recognition quality (at least three syllables)			&	Yes		\\ \hline
7	&	Fits on a single line													&	Yes		\\ \hline
8	&	Does not sound similar to existing commands								&	Yes		\\ \hline
9	&	Does not require immediate interactivity in Mirror API Glassware.				&	Yes		\\ \hline
10	&	Has an imperative verb with an object									&	Yes		\\ \hline
11	&	Uses articles when possible											&	No		\\ \hline
12	&	Uses definite articles only when the object is definite							&	No		\\ \hline
13	&	Uses ``this'' when there is only one relevant instance of the object				&	No		\\ \hline
14	&	Uses me and my when appropriate										&	No		\\ \hline
15	&	Refers to Glass as the subject carrying out the action						&	Yes		\\ \hline
		
		\end{tabularx}
	\end{table}

\subsection{Test Cases}
The following section describes how the tests were set up and carried out.

\subsubsection{Experimental Setup}
The tests were carried out using an optical bench to guarantee more scientific accuracy. The experimental setup contained an optical bench, with a screen holder at the zero point where the QR code was positioned. The device currently being tested, Google Glass or smartphone, was then positioned at the specified mark on the optical bench using a clamp and pointed towards the QR code. See Figure~\ref{experimentalSetup} for a better understanding of the experimental setup. 

	\begin{figure}[ht!]
		\centering
		\includegraphics[width=110mm]{images/demo/componentText}
		\caption{todo change image The experimental setup.}
		\label{experimentalSetup}
	\end{figure}

In order to measure the time needed for the results of each test a specific class was built, called \texttt{Timer} and seen in Listing~\ref{timerClass}. The \texttt{Timer} class was built using the singleton design pattern. A singleton class is a class that can only be instanced once during the entire execution of an application, however the instance lives throughout the entire execution and may be accessed from anywhere in the application.

Using this pattern meant that the timer could be started in one class, and stop in another without having to pass the instance around, which potentially could affect performance.

\begin{lstlisting}[language=Java, caption={The Timer class}, label=timerClass]

public class Timer {
	private static Timer ourInstance = new Timer();
	public static Timer getInstance() { return ourInstance; }
	private Timer() {  }
	
	private boolean timerRunning = false;
	private Long startTime;
	private Long stopTime;
	
	public void startTimer() { 
		if(timerRunning) { Log.d("TIMER", "Timer already running"); }
		else 	{ startIme = System.nanoTime(); }
	}
	
	public void stopTimer() {
		if(!timerRunning) { Log.d("TIMER", "No timer running"); }
		else { stopTime = System.nanoTime()); }
	}
	
	private long getElapsedTime(int timerID) { return stopTime - startTime; }
	
	public void logElapsedTime(String information) {
		Log.d("TIMER", information + ": " + String.valueOf(getElapsedTime() + " nano seconds");
	}
}

\end{lstlisting}

\subsubsection{Text Length}
\begin{lstlisting}[language=Java, caption={The randomizer class}, label=todo]
private double randfrom(double min, double max)
{
	Random rand = new Random();
	double range = (max - min);
	return min + range * rand.nextDouble();
}

private String getChar(int pos, double rand)
{
	if(rand <= doubleList.get(pos) || pos+1 <= alph.size())
		return alph.get(pos);
		
	return getChar(pos+1, rand);
}

public String randchar()
{
	double rand = randfrom(0, 1);
	return getChar(0, rand);
}
\end{lstlisting}

\subsubsection{Distance to the QR Code}

	\begin{table}[ht!]
    		\caption{Average time of registering a QR code with varying distance.} \label{tab:distanceAverage}
		\centering \begin{tabularx}{\textwidth}{l|X|X|X} \hline
		\textbf{Distance (dm)} & \textbf{Google Glass} & \textbf{Samsung Galaxy SII} & \textbf{Samsung Galaxy SIII} \\ \hline \hline
       
		1	&	&	&	\\ \hline
		2	&	&	&	\\ \hline
		3	&	&	&	\\ \hline
		
		\end{tabularx}
	\end{table}

\subsubsection{Complexity of the QR Code}

	\begin{table}[H]%ht!]
    		\caption{Average time of registering a QR code with varying density.} \label{tab:complexityAverage}
		\centering \begin{tabularx}{\textwidth}{l|X|X|X} \hline
		\textbf{Encoded Characters} & \textbf{Google Glass} & \textbf{Samsung Galaxy SII} & \textbf{Samsung Galaxy SIII} \\ \hline \hline
       
		1	&	&	&	\\ \hline
		50	&	&	&	\\ \hline
		100	&	&	&	\\ \hline
		
		\end{tabularx}
	\end{table}

\subsubsection{Display Time}

	\begin{table}[ht!]
    		\caption{Average display time for Google Glass with varying information size.} \label{tab:averageDisplaySpeedGoogleGlass}
		\centering \begin{tabularx}{\textwidth}{l|X|X|X} \hline
		\textbf{Information Size (Byte)} & \textbf{Google Glass (ms)}  & \textbf{Samsung Galaxy SII (ms)}  & \textbf{Samsung Galaxy SIII (ms)} \\ \hline \hline
       
		100 k	&	&	&	 \\ \hline
		1 M		&	&	&	 \\ \hline
		10 M		&	&	&	 \\ \hline

		\end{tabularx}
	\end{table}

\subsection{Summary}
\label{subsec:summary}
The application works as such that the first screen the user sees when launching the application, on both Google Glass and smartphones, is the camera screen. The user is then to hposition the device in such a was that the QR code may be scanned by the device's camera. The QR code contains a product ID for a specific product.

The user does not need to press any shutter button in order to scan the QR code. Instead the application will automatically recognise any QR code pattern that appears in camera view, as well as scan the QR code. The reason for not implementing a start menu or any similar start screen, to show the user before the camera screen, is to, according to Google's design guidelines, keep the focus on what the application is intended to do and to keep the application simple and easy to use.

Next the application will decode the QR code. The decoding process is done by the ZXing library, which is an open source  barcode image processing library. The QR code contains a product ID which is then used in the downloading process. The downloading process entails connecting to a database containing information on different products, and, by using the decoded product ID, retrieving the information on the specific product. 

The downloaded information contains the product name, as well as a list of components and the instructions necessary to construct the product. All the information is then sorted in to respective classes and the information may be displayed to the user. When the product information is being downloaded a loading animation is displayed on screen. On Google Glass the loading information is a loading bar at the bottom of the screen, and in the smartphone application the loading animation is a spinning wheel.

When the download process has finished the information is displayed to the user in the form of a slide show. The first slide that is displayed to the user is the title slide. The title slide contains the name of the product as well as an image  (if an image existed in the database). Following the title slide are the component slides. Each component has their own slide due to the fact that a component may be described in both text and an image. 

After all the component slides comes the instruction slides. Similar to the components an instruction my be presented by text only or by both text and an image. In contrast to the components, however, instructions may also be presented with an image and no text. 

As Google provides developers of Google Glass applications with predefined layouts, these layouts were also used for the Google Glass application. The layouts used were ``Title'', ``Columns'', ``Text'' and ``Caption''. The predefined layouts were also used as basis for the layouts used in the smartphone application.

The Title layout is used for the title card. The Columns layout is used for the slides with both text and an image. The Text layout is used for the slides with only text. The Caption layout is used for the slides with only an image. All layouts, except for the Title layout, also contains text at the bottom of the screen called ``footer'' and ``timestamp''. The footer contains information on wether the current slide is a component slide or an instruction slide. The timestamp contains information on which slide is currently being viewed.

While browsing through the slides in the Google Glass application, the user may also navigate using voice commands. The voice commands available in the Google Glass application are ``Show next slide'', ``Show previous slide'', ``Show components'', ``Show instructions'' and ``Scan again''.

Most of the voice commands follows 11 out of the 15 voice command guidelines provided by Google. For instance does ``Show components'' not follow the guidelines which states that ``Is general enough to apply to multiple Glassware, but still has a clear purpose''. ``Show components'' is a specific voice command and could potentially apply to multiple Google Glass applications, but not most.

While viewing the slides ``ok glass'' is shown at the bottom of the screen. ``ok glass'' indicates that voice commands are available and saying ``ok glass'' at that point brings up the voice command menu, showing all available voice commands. However, ``ok glass'' is also shown in combination with a dark, transparent overlay, which does ensure ``ok glass'' is always visible no matter what image is shown on screen, but the dark overlay does also mean that any image shown is darken by the overlay.

In terms of the testing done on the application the experimental setup consisted of an optical bench on which the QR code were positioned at the zero mark, and then each specific device were position as such that the camera of each device were positioned according to the specifications of each test. Each test result was then obtained through a laptop with which each device was connected via a USB cable. The test results was printed out from a timer class, called \texttt{Timer}, which was implemented using the singleton design pattern, meaning that the class only had one global instance which could be accessed from anywhere within the appllication.

The test which did not require the experimental setup was the text length test. Instead the text length test was performed using a class which randomised english characters, which were then added together to a large string.

Instead three other tests were done while using the experimental setup. In these tests the distance to the QR code, the complexity of the QR code and the size of the downloaded information respectively. Each of these tests were done 30 times, for each device and each different specification, to ensure statistical significance.

%o   Implementation - Present your project implemetion in general
%o   Information - Give details here (possibly several sub-sections)
%o   Summary - for this chapter

% Mobile phone application
% Uses Zxing - library for QR code scanning [Link to gihub repository missing!!!]
% can display both text and images 
% 
% Google Glass Application
% Uses BarcodeEye - library for QR Code Scanning (port of Zxing for Google Glass) [Link to github repository missing!!!]
% Can display text, images still todo
% Much easier to add slides on Google Glass compared to Mobile Phone Application. Probably because Cards is standard interface for Google Glass (might also be because simple API, check CardPresenter!!!)
% Very difficult to add images since cards can't be changed after .getView() has been called
% Need to call .notifyViewChanged() but does not work anyway (yet) seems to only start the activity over which calls the execute method again if no check has been put in place