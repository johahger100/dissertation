%% Sample dissertation using the hks_report LaTeX class
%%
%% 950203  Michael Kelsey -- LaTeX2e format (cit_thesis)
%% 990210  Anna Brunstrom -- Modified to KAU requirements

\documentclass[12pt,twoside]{kau_report}

%\usepackage{swedish}

% Define the parameters in the preamble

\title{Style Guidelines for Bachelor's Project Reports}

\pubnum{1999:01}

\author{Anna Brunstrom}

\date{Januari 13, 1999}

\advisor{NN}
\examiner{NN}

% The actual document starts here
\begin{document}

\makekautitle
%\makeswekautitle

\copyrightpage
%\swecopyrightpage

\begin{frontmatter}

\approved
%\approvedtwo{}{}
%\approvedthree{}{}{}
%\sweapproved
%\sweapprovedtwo{}{}
%\sweapprovedthree{}{}{}

\begin{abstract}
This document describes the formatting rules for Bachelor's project
reports as
specified by the Department of Computer Science at Karlstad University.
The required outline of the document as well as specific
formatting requirements such as page numbering and page layout are
described. A section with specific requirements for documents written
in Swedish is included. For documents produced in LaTeX a special
class file, called {\tt kau\_report.cls} is provided by the
department. Some information on how to use this class file is also
given in the document.
\end{abstract}

%\begin{acknowledgements}
%Lots of wonderful people helped me to produce this.  If I tried to thank
%them all, I'd be here forever, so I won't thank any of them.
%\end{acknowledgements}

  \tableofcontents

  \listoffigures

  \listoftables
\end{frontmatter}


\section{Introduction}
This document gives the style guidelines for Bachelor's project
reports written at the Department of Computer Science, Karlstad
University. It has been produced using the {\tt kau\_report} LaTeX
class file. Hence, this document, in addition to listing the style
guidelines, also work as a sample report with respect to the
formatting guidelines it describes\footnote{The content and 
length of the document are of course not representative of a report.}.
If you use formatting files for LaTeX or Microsoft Word provided by
the Department of Computer Science most of the formatting required
will be handled automatically. It is strongly recommended that you use
one of the provided formatting files.
If you choose to use a different word processor you are still required
to follow the described guidelines. Any deviations from the guidelines must
be approved by the course examiner. 

The reminder of the document is organized as follows. Section 2
describes the general layout required for the report. Some parts of
the report are described in more detail in section 3. Section 4
provides additional information required for
reports written in Swedish. Finally, section 5 provides some
information on how to use the {\tt kau\_report.cls} class file.

\section{General Layout}
This section provides information on the general layout of the report,
such as organization, page numbering etc. Again, this document serves
as an example of the required layout.

The outline of the report should follow the order specified in Figure \ref{outline}.
\begin{figure}[ht]
\begin{center}
\begin{minipage}{8cm}
\hrule
%\framebox[10cm][c]{
\begin{enumerate}
\item Title page
\item Copyright page
\item Approval page 
\item Abstract 
\item Acknowledgements (optional)
\item Contents 
\item List of Figures 
\item List of Tables 
\item Main Body 
\item References 
\item Appendices (optional)
\end{enumerate}
\hrule
\end{minipage}
\end{center}
\caption{Report outline}
\label{outline}
\end{figure}
As indicated in Figure \ref{outline}, some portions of the report are
optional. The ``List of Figures'' and/or the ``List of Tables'' may
also be omitted due to the lack of figures and/or tables in the main
body. All other parts are always required. The parts preceeding the
main body are called front matters. The table of contents
should list all parts following the front matters. Parts in the front
matters should not be included in the table of contents. If a list of
abbreviations is provided it should be placed as an appendix.

The report should be bound, with the cover in sulfur-yellow. The
color is available at the printing works at the University as
``svavelgul''. % (Colorit 72). 
We recommend that you bind your report
at the University. Two-sided printing
should be used. However, the approval page, the abstract and the main body
should start on the right hand side. All parts in the front matter
should start on a new page.
The first section of the main body should be Section 1. Any appendices
that are part of the report should be 
numbered alphabetically. 

The pages of the front matters should be
numbered using Roman numerals. (However, no page numbers are used on the title
page or the copyright page.)
The pages of the remaining parts, starting with the main body, should
be numbered 
using Arabic numerals. The page numbering should take
blank pages into account. For instance the page number for the approval
page is {\tt iii}. (The title page is not counted in the numbering.)
The page numbering for the main body should start
over with page one. Figures and tables should be numbered using the
section number followed by a running number. For instance, the first
figure in Section 2 is numbered 2.1.

The fonts size for regular text should be 12 point. The line-spacing
should be 1.5 with the exception of the Reference section which should
be single-spaced. The text should be in block format with even right and
left hand margins. This includes the abstract and the acknowledgements
(when present). The page style for the report should be plain with the page
number centered at the bottom of the page.

\section{Further Details}
The general layout of the report was described in the previous
section. This section provides some further details on some of the
required parts of the report.

\subsection{Title page}
The title page should conform to the format illustrated by the
cover page of this document. The logo of the university should be
centered at the top of the page and can be obtained from the
department in postscript format if needed.  The report number is to be
placed at the bottom of the page.
Ask your adviser for the report number to use. There
should be no page number on the title page.

\subsection{Approval page}
The approval page must be signed by the student, the
advisor, and the examiner. By signing the approval page the student
certifies that the work presented in the report has indeed been carried out
by the student. Material obtained from other sources must be properly
cited. The date on the approval page refers to the date when the
student successfully defended his or her work.

\subsection{References}
References to the literature should be done using a numbered
reference list, where the reference in the text appears as the number
of the reference in square brackets. For instance, Lamport's seminal
paper on causality is referenced as \cite{Lamport78}. References should
be ordered alphabetically in the reference lists. Only material
referenced in the text may appear in the reference list. The parts of
an individual reference in the reference list should be listed in
the following order: author, title, source, and date. The font and
punctuation used should be consistent for all entries in the reference
list. Examples on how to reference a book \cite{Silberschatz94} and
journal articles \cite{Lamport78,Dijkstra83} can be found in the
reference list of this document. 

When you want to mention the author(s) of a reference in running
text the convention is to list the names if there are one or two
author(s), but to use the first author followed by ``et. al.'' if there are
more than two authors. For instance, when we reference Silberschatz
and Galvins book on operating systems \cite{Silberschatz94} we mention
both authors. In contrast, we use the abbreviated form when we refer
to the paper on termination detection by Dijkstra
et. al. \cite{Dijkstra83}. 

\section{Reports Written in Swedish}
Documents written in Swedish should still use the guidelines described
in sections 2 and 3 of this document. The only exception is that two
abstracts are now required. The first one written in Swedish and the
second one written in English. The two abstracts should be placed on
separate pages, with the abstract written in Swedish positioned
first. Only the first abstract needs to be on a
right-hand page. 
The mapping from the English terms described in the outline of the
report to their Swedish counterparts is given in Table \ref{map}.
A document written in Swedish must use the Swedish names in the
document. 
 
\begin{table}[htb]
\vspace{5mm}
\begin{center}
\begin{tabular}{|clclc|}
\hline
& & & & \\
& Abstract & & Sammanfattning & \\
& Acknowledgements & & Tack & \\
& Contents & & Inneh{\aa}ll & \\
& List of Figures & & Figurer & \\
& List of Tables & & Tabeller & \\
& Appendix & & Bilaga & \\
& References & & Referenser & \\
& & & & \\
\hline
\end{tabular}
\end{center}
\caption{Language mapping}
\label{map}
\end{table}

\section{The Kau\_report Class}
{\tt Kau\_report.cls} is a LaTeX class file which formats a report
according to the rules specified by the Department of Computer Science
at Karlstad University. It extends the LaTeX article class file
and borrows heavily from the {\tt
cit\_thesis.cls} developed at Caltech. The class file sets up the
layout of the report and provides a number of helpful macro 
commands for defining structural elements of the 
document such as the title page and the approval page. 
To use the class file you should put
\begin{center}
\verb|\documentclass[12pt,twoside]{kau_report}|
\end{center}
at the beginning of your document. The options specify the fontsize and
the use of twosided printing.
The easiest way to create your report is to start with
the file {\tt skeleton.tex} which has a LaTeX skeleton containing all
the required commands. The {\tt skeleton.tex} file also contains
comments describing the changes needed for reports that are written in
Swedish. Be aware that if you switch between English and Swedish you
must remove all auxiliary files created by LaTeX before
recompiling. The {\tt kau\_report.cls} and {\tt 
skeleton.tex} files are contained in the compressed tar archive {\tt
kau\_report.tgz} along with additional useful files. A {\tt README} as
well as the LaTeX source for this document is also included in the
archive. The archive can be obtained from {\tt
<http://www.cs.kau.se/cs/docs/kau\_report.tgz>}. A complete listing of the files
contained in the archive can be found in appendix A.


\begin{singlespace}
\bibliography{rules}
\bibliographystyle{plain}
\end{singlespace}

\appendix
\section{LaTeX Files}
This appendix describes all the files contained in the compressed tar
archive {\tt kau\_report.tgz}. The archive is available at {\tt
<http://www.cs.kau.se/cs/report.tgz>}.
\begin{description}
\item{\bf README} README file for the kau\_report distribution
\item{\bf kau\_report.cls} The main class file to generate a report
according to the style guidelines described in this document.
\item {\bf skeleton.tex} A skeleton LaTeX file that contains all the
commands necessary to generate a correctly formatted report.
\item{\bf kau.eps} The University logo in postscript
format. Used on the Swedish title page.
\item{\bf kaueng.eps} The English version of the University logo in
postscript format. Used on the English title page.
\item{\bf setspace.sty} A style file to provide better line spacing.
Used in the main class file.
\item{\bf rules.tex} The LaTeX source for this document.
\item{\bf rules.bib} The bibtex file for this document.
\item{\bf rules.ps} This document.
\end{description}

\end{document}
