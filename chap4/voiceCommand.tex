The Google Glass application gives users the option to use voice commands in order to navigate the slides. The user opens the voice command menu by saying ``ok glass'' at any point in the application when ``ok glass'' is written at the bottom of the screen. The voice command feature is available at all times except when the camera is active. In other words the voice commands are unavailable when the application is waiting to scan a QR code.

The voice command menu contains the following options.

\begin{itemize}
	\item \textbf{Show next slide}
	
	The application scrolls to the next slide. If the current slide is the last slide, and in other words no other slides are following, the application does nothing.
	\item \textbf{Show previous slide}
	
	The application scrolls to the previous slide. If the current slide is the first slide, and in other words no other slides are sits before it, the application does nothing.
	\item \textbf{Show components}
	
	The application scrolls to the first slide showing information on a component. If the user is currently on the first slide showing information on a component the application does nothing. 
	\item \textbf{Show instructions}
	
	The application scrolls to the first slide showing an instruction. If the user is currently on the first slide showing an instruction the application does nothing.
	\item \textbf{Scan again}
	
	The application launches the camera and expects the user to scan another QR code.
\end{itemize}

Implementing voice command in the Google Glass application is done by following Google's step-by-step guide on how to implement voice commands in Goole Glass applications~\cite{howToVoiceInput}. Listing~\ref{voiceCommandXML} shows the resulting XML which gives the voice commands used in the application. Using contextual voice commands means that ``ok glass'' is displayed at the bottom of the screen at all times when voice commands are available. ``ok glass'' also comes with a black overlay, seen for instance in Figure~{fig:cardLayout}, which is transparent yet darkens the slides a bit, especially near the bottom of the slides where ``ok glass'' appears. Although the dark overlay could potentially distort the slides a bit, especially when the slide contains only an image, it was as of implementing the voice command feature not possible to alter the ``ok glass'' overlay in any way~\cite{voiceCommandCustom1, voiceCommandCustom2}. The dark overlay does however ensure that ``ok glass'' is always visible, no matter what the background image look like.

\begin{lstlisting}[language=XML, caption={The voice command menu XML file}, label=voiceCommandXML]
<menu xmlns:android="http://schemas.android.com/apk/res/android">
	<item
		android:id="@+id/next_menu_item"
		android:title="Show next slide" >
	</item>
	<item
		android:id="@+id/previous_menu_item"
		android:title="Show previous slide" >
	</item>
	<item
		android:id="@+id/components_menu_item"
		android:title="Show components" >
	</item>
	<item
		android:id="@+id/instructions_menu_item"
		android:title="Show instructions" >
	</item>
	<item
		android:id="@+id/scan_menu_item"
		android:title="Scan again" >
	</item>
</menu>
\end{lstlisting}

Although none of the voice commands have been sent in for official approval by Google most of the voice commands follows the design guidelines provided by Google. As seen in Table~\ref{tab:voiceCommandCheckTableChecked} 11 och the 15 voice command guidelines provided by Google has been followed. However, some of the guidelines has been applied to some or most of the voice command used within the application, and not all. For instance ``Show components'' does not follow the first guideline of the voice command checklist. ``Show components'' is however still a part of the application as ``Show components'' is a key feature of the application.

	\begin{table}[ht!]
    		\caption{Voice Command Checklist~\cite{glassVoiceChecklist}.} \label{tab:voiceCommandCheckTableChecked}
		\centering \begin{tabularx}{\textwidth}{l|X|l} \hline
		 & \textbf{Guideline} & \textbf{Acheived} \\ \hline \hline
       
1	&	Is general enough to apply to multiple Glassware, but still has a clear purpose		&	Yes		\\ \hline
2	&	Is colloquial and can explain Glass features in a conversation					&	Yes		\\ \hline
3	&	Is comfortable to say in public											&	Yes		\\ \hline
4	&	Brings the user from intent to action as quickly as possible					&	Yes		\\ \hline
5	&	Avoids brand words													&	Yes		\\ \hline
6	&	Is long enough to ensure high recognition quality (at least three syllables)			&	Yes		\\ \hline
7	&	Fits on a single line													&	Yes		\\ \hline
8	&	Does not sound similar to existing commands								&	Yes		\\ \hline
9	&	Does not require immediate interactivity in Mirror API Glassware.				&	Yes		\\ \hline
10	&	Has an imperative verb with an object									&	Yes		\\ \hline
11	&	Uses articles when possible											&	No		\\ \hline
12	&	Uses definite articles only when the object is definite							&	No		\\ \hline
13	&	Uses ``this'' when there is only one relevant instance of the object				&	No		\\ \hline
14	&	Uses me and my when appropriate										&	No		\\ \hline
15	&	Refers to Glass as the subject carrying out the action						&	Yes		\\ \hline
		
		\end{tabularx}
	\end{table}

The reason for not following guidelines 11--14 is because they would make the voice commands longer. As the voice commands may potentially be said often while using the application, as the user may proceed through the slides quite fast, shorter voice commands makes for more comfortable use. As the voice commands still follows guideline 6 the voice commands were deemed to be long enough to still ensure high recognition quality.