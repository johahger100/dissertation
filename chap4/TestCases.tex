\subsubsection{Experimental Setup}



\begin{lstlisting}[language=Java, caption={The Timer class}, label=todo]

public class Timer {
	private static Timer ourInstance = new Timer();

	public static Timer getInstance() {
		return ourInstance;
	}
	
	private Timer() {  }
	
	private ArrayList<Long> startTime = new ArrayList<Long>();
	private ArrayList<Long> stopTime = new ArrayList<Long>();
	
	public void startTimer(int TimerID) {
		startTime.add(timerID, System.nanoTime());
	}
	
	public void stopTimer(int TimerID) {
		stopTime.add(add, System.nanoTime());
	}
	
	private long getElapsedTime(int timerID) {
		return stopTime.get(timerID) - startTime.get(timerID);
	}
	
	public void logElapsedTime(int timerID) {
		Log.d("TIMER", String.valueOf(timerID) + ": " + String.valueOf(getElapsedTime(timerID)) + " nano seconds");
	}
}

\end{lstlisting}

\subsubsection{Text Length}
\begin{lstlisting}[language=Java, caption={The randomizer class}, label=todo]
private double randfrom(double min, double max)
{
	Random rand = new Random();
	double range = (max - min);
	return min + range * rand.nextDouble();
}

private String getChar(int pos, double rand)
{
	if(rand <= doubleList.get(pos) || pos+1 <= alph.size())
		return alph.get(pos);
		
	return getChar(pos+1, rand);
}

public String randchar()
{
	double rand = randfrom(0, 1);
	return getChar(0, rand);
}
\end{lstlisting}

\subsubsection{Distance to the QR Code}

	\begin{table}[ht!]
    		\caption{Average time of registering a QR code with varying distance.} \label{tab:distanceAverage}
		\centering \begin{tabularx}{\textwidth}{l|X|X|X} \hline
		\textbf{Distance (dm)} & \textbf{Google Glass} & \textbf{Samsung Galaxy SII} & \textbf{Samsung Galaxy SIII} \\ \hline \hline
       
		1	&	&	&	\\ \hline
		2	&	&	&	\\ \hline
		3	&	&	&	\\ \hline
		
		\end{tabularx}
	\end{table}

\subsubsection{Complexity of the QR Code}

	\begin{table}[H]%ht!]
    		\caption{Average time of registering a QR code with varying density.} \label{tab:complexityAverage}
		\centering \begin{tabularx}{\textwidth}{l|X|X|X} \hline
		\textbf{Encoded Characters} & \textbf{Google Glass} & \textbf{Samsung Galaxy SII} & \textbf{Samsung Galaxy SIII} \\ \hline \hline
       
		1	&	&	&	\\ \hline
		50	&	&	&	\\ \hline
		100	&	&	&	\\ \hline
		
		\end{tabularx}
	\end{table}

\subsubsection{Display Time}

	\begin{table}[ht!]
    		\caption{Average display time for Google Glass with varying information size.} \label{tab:averageDisplaySpeedGoogleGlass}
		\centering \begin{tabularx}{\textwidth}{l|X|X|X} \hline
		\textbf{Information Size (Byte)} & \textbf{Google Glass (ms)}  & \textbf{Samsung Galaxy SII (ms)}  & \textbf{Samsung Galaxy SIII (ms)} \\ \hline \hline
       
		100 k	&	&	&	 \\ \hline
		1 M		&	&	&	 \\ \hline
		10 M		&	&	&	 \\ \hline

		\end{tabularx}
	\end{table}