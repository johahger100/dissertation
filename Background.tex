%How did Google Glass come about? History!
Mobiles today are used as a nervous tick. It is a distraction and something that pulls your attention away from the real world. At least that is what Sergey Brin, one of the founders of Google, claimed during a Ted Talk presentation in February 2013.\cite{tedtalkWhyGlass} Brin stated that if he was a smoker he would probably light a cigarette at those times when he now uses his phone.
%\url{http://www.ted.com/talks/sergey_brin_why_google_glass}

Brin and his team wanted to create something that would make interaction with technology easy and fast and not distract from reality. They wanted to keep the information more handy and close by than a phone stuck in the users pocket. But they also wanted to keep the line of sight free. Thad Starner, technical lead/manager on Google Glass, wrote in an article in 2013,\cite{6504855} that he sought out to build something as intuitive as a watch. An extension of the self, as he stated. And so Google started working on Project Glass. 




% The idea behind Glass was to minimise the time between intention and action
% users should not have to bring up something from their pockets each time they want to interact with technology
% they should be able to just simply interact
% They wanted to create something as intuitive as a watch. 
% checking the watch is something a user might do without actually thinking about what the time is.
% they might have to check again if someone were to ask them what the time is.
% Glass should be an extension of the self rather than another device.
% // Thad Starner - technical lead/manager on Google Glass

% Sergey Brin, one of the founders at Google, has similar ideas
% ted talk he spoke about how checking the phone was something he did without reason
% putting notifications more easily accessible would minimise interaction with technology because the user
% would not have to check if any updates have come in, they would know right away



April 4th, 2012, Google announced ``Project Glass''. Glass was under development for several years at Google's research and development department, Google X. The idea behind the device was to make technology easier to access but also to only be available when the users wants to. Serge Brin, on of the founders of Google, did a Ted Talk in February 2013 where he talked about why they decided to produce the device. His argument was that smartphones was something users kept looking at even when they did not want to. They might have missed something. Instead Google wanted to create a device that would notify the user [TODO WTF, doesn't a phone do this!?!?!]

Thad Starner, technical lead/manager on Google Glass, claimed that Glass is supposed to be an extension of the self. He compared the device to a watch. A watch is an easy to access device that seems natural to its wearers. Starner claimed that when an individual looks at his or her's watch 




\subsection{How does Google Glass work?}
\label{subsec:googleglass}
Google Glass, or simply ``Glass'' as the device is known within Google, is a head-mounted display (HMD) that can be seen as an augmented reality device (see Section~\ref{subsubsec:hmd} and Section~\ref{subsubsec:ar} respectively) designed to bring notifications to the user more easily than a smartphone does. Google Glass is shown in Figure~\ref{GoogleGlassHardware}. According to Google ``Glass is designed to be there when the user needs it and to stay out of the way when the user does not''~\cite{glassDesignPrinciples}. Google Glass is meant to give the user relevant information at relevant times.
%\url{https://developers.google.com/glass/design/principles}

	\begin{figure}[ht!]
		\centering
    \subfloat[The user can control Google Glass with the touchpad.]{{\includegraphics[width=70mm]{images/GoogleGlassHardwareTouchpad} }}
    \qquad
    \subfloat[The display sits slightly above the user's line of sight, on the right hand side.]{{\includegraphics[width=70mm]{images/GoogleGlassHardware} }}
    \qquad
		%\includegraphics[width=110mm]{images/GoogleGlassHardware}
		\caption{Google Glass is equipped with a touchpad and a camera~\cite{ImagesGoogleGlassUI}.}
		\label{GoogleGlassHardware}
	\end{figure}

Google Glass is partially controlled with a touchpad, but can also be controlled through voice commands. The touchpad sits on the right hand side of the user's glass frame and runs from the temple to the ear (see in Figure~\ref{GoogleGlassHardware}~(a)). When the user touches anywhere on the touchpad Google Glass ``wakes up'' from stand by and displays the start screen (which consists of a clock). The display is mounted above the user's line of sight, on the right hand side (see Figure~\ref{GoogleGlassHardware}~(b)) and can be slightly adjusted so that the user can see all that is currently being displayed.

The display is a projection that goes through an optic lense in the glass piece, seen in Figure~\label{GoogleGlassHardware}~(b), which creates a virtual image. A virtual image is an image that, projected through optic lenses, appears to be located at a point where the actual projection is not~\cite{virtualImageWiki}. In the case of Google Glass the display appears to be located further away from the user than the display actually is. The display is said to be equivalent of a 25 inch high definition screen seen from a distance of approximately 2.5 meters~\cite{GlassSpecs}.

%
%
%
%
%
%
%
%	\begin{figure}[ht!]
%		\centering
%		\includegraphics[width=110mm]{images/GoogleGlassUI}
%		\caption{A virtual representation of the Google Glass user interface as the graphical user interface is perceived by the user.\cite{ImagesGoogleGlassUI}}
%		\label{GoogleGlassUI}
%	\end{figure}
%	
%	
%	
%
%
%
%
%The graphical user interface (GUI) is called a timeline (see Figure \ref{GoogleGlassUI}). The timeline consists of a row of cards. Cards are basic applications such as a clock or information about the weather. Cards can also represent more in-depth applications, on Google Glass called ``Immersions''. An immersion handles activities such as browsing an image gallery or playing a game.\\
%
%On the timeline cards to the left of the home screen are upcoming activities such as an event in the user's calendar or an upcoming flight. Cards to the right of the home screen are from the past. Cards from the past will for instance show text messages or photos. When the user wants to turn of Google Glass the user swipes down on the touchpad. Swiping down on the touchpad will put Google Glass in stand by. If the user wants to turn of Google Glass entirely there is a power button on the opposite side of the touchpad. Holding down the power button for a few seconds will turn of Google Glass. For a better visual understanding of how Google Glass works see Figure \ref{GoogleGlassUI} as well as the video referenced in the caption.\\
%
%Glass uses a small display placed to the upper right of the user's line of sight and is mounted on the user as a regular pair of glasses. Equipped with a camera, microphone and speakers it is capable of performing a lot of the tasks users normally would do on a smartphone such as taking photographs, video chatting, writing text messages.
%
%
\subsubsection{Head-Mounted Display (HMD)}
\label{subsubsec:hmd}
A head-mounted display (HMD)~\cite{hmdWiki} is a device that is worn on the head and that places a small display in front of one or both of the user's eyes. The device can either be a stand alone device or a part of a helmet. A branch of HMDs are optical head-mounted displays (OHMDs)~\cite{ohmdWiki}. A OHMD is a HMD with a see-through display, for instance Google Glass.
%A head mounted display is just as it sounds a display that goes on ones head and that displays stuff

\subsubsection{Heads-Up Display (HUD)}
\label{subsubsec:hud}
A heads-up display (HUD)~\cite{hudWiki} is defined as any transparent display that, when presenting information, does not require userss to look away from their usual viewpoints. In other words, a HUD may be a HMD and a HMD may be a HUD. While a HMD is always worn on the head a HUD can be a stand-alone display. In contrast a HUD must be a transparent display. A requirement a HMD does not have. A OHMD, however, is always a HUD since a OHMD has a transparent display.

\subsubsection{Virtual Reality}
\label{subsubsec:vr}
Virtual reality~\cite{virtualRealityDef} is defined as a computer generated simulation that enables users to interact with a three-dimensional environment. Virtual realities are common in interactive mediums such as video games. Virtual realities can also be combined with a HMD in order to completely engulf the user in the virtual reality. One such example is the Oculus Rift, seen in Figure~\ref{OculusRift}, that completely covers the user's eyes, allowing the user to experience the virtual reality.

	\begin{figure}[ht!]
		\centering
		\includegraphics[width=110mm]{images/OculusRift}
		\caption{The HMD ``Oculus Rift'' is a virtual reality device~\cite{ImagesOculusRift}.}
		\label{OculusRift}
	\end{figure}

Google Glass is able to display a virtual reality but does not work as a virtual reality device. Google Glass only covers a small part of the user's field of vision and as such does not have the capability of simulating a three-dimensional, interactive, environment in contrast to the Oculus Rift. Oculus Rift, unlike Google Glass, is able to replace the user's reality with a completely virtual reality since Oculus Rift completely covers the user's eyes.

\subsubsection{Augmented Reality}
\label{subsubsec:ar}
Augmented reality~\cite{augmentedRealityDef} is defined as the combination of reality (or what is within current context being perceived as reality~\footnote{Augmented reality is for instance common in video games to give the player environmental and health information.}) with useful, computer generated, data. Augmented reality, unlike virtual reality, is not meant to replace reality, but rather to enhance interaction with the current reality.

A HUD may create an augmented reality. The reason a HUD does not always create an augmented reality is due to the fact that the information being presented might not be useful within the current context. An augmented reality is, as stated above, meant to enhance reality, while a HUD does not have that requirement.

Google Glass is a HUD that has the potential (and intent) to create an augmented reality. Google Glass is intended to present useful information to users while not distracting them from reality. One example of useful information that could enhance users interaction with reality would be a shopping list while users are out shopping, as seen in Figure~\ref{GlassShopping}.

	\begin{figure}[ht!]
		\centering
		\includegraphics[width=110mm]{images/GoogleGlassKeepRevelant}
		\caption{A shopping list while the user is out shopping is useful information~\cite{glassDesignPrinciples}.}
		\label{GlassShopping}
	\end{figure}
	
	
%\subsubsection{Augmented Reality vs Virtual Reality}
%\label{subsubsec:augmentedrealityvsvirtualreality}

%[TODO WRITE ABOUT HUD AND HMD]

%When discussing head mounted displays it is possible that the first image that pops into ones head is similar to Figure \ref{OculusRift}. What is important to note about Oculus Rift (Figure \ref{OculusRift}) and other similar product that completely covers the user's eyes is that these are all virtual reality devices. Virtual reality is not the same as augmented reality, which is what Google Glass gives the user.

%The difference lies in how much of what the user can see is computer generated. In a virtual reality the entire environment is computer generated. Augmented reality on the other hand is based in reality where computer generated elements of the environment help enhance reality. In other words: virtual reality replaces reality and augmented reality enhance reality. Since Google Glass does not remove the user from reality but rather display information that can be consumed at the same time as the user experience the real world Google Glass is an augmented reality device compared to Oculus Rift which is a virtual reality device.

%TODO --- ADD HUD vs AUGMENTED REALITY!!!!!!


%What is it?
%Augmented Reality vs Virtual Reality
%Define Augmented Reality
%Define Virtual Reality


\subsection{User Interface}
\label{subsec:userinterface}
Google Glass' graphical user interface (GUI) is called a timeline (see Figure \ref{GoogleGlassUI}).\cite{ImagesGoogleGlassUI} The timeline consists of a row of cards. Cards are basic applications such as a clock (see Figure \ref{GoogleGlassCards} (a)) or information about the weather. Cards can also represent more in-depth applications, on Google Glass called ``Immersions'' (see Figure \ref{GoogleGlassCards} (b) and (c)). Immersions handles activities such as browsing an image gallery or playing a game.

	\begin{figure}[ht!]
		\centering
		\includegraphics[width=110mm]{images/GoogleGlassUI}
		\caption{A visual representation of the Google Glass GUI as the GUI is perceived by the user. In reality only one card can be displayed at a time.\cite{ImagesGoogleGlassUI}}
		\label{GoogleGlassUI}
	\end{figure}

The first screen the user sees when starting up Google Glass is the home screen. The home screen displays a clock and also shows the text "ok glass", as seen in Figure \ref{GoogleGlassCards} (a). The home screen is a part of the timeline and acts as the center point. Cards to the left of the home screen are upcoming activities such as an event in the user's calendar or an upcoming flight. Cards to the right of the home screen are from the past. Cards from the past will for instance show text messages or photos.

	\begin{figure}[ht!]
		\centering
   	\subfloat[The Google Glass home screen is a card that displays a clock.]{{\includegraphics[width=70mm]{images/GoogleGlassHomescreen} }}
  	 \qquad
   	\subfloat[The card ``Explore stars'' represents an immersion.]{{\includegraphics[width=70mm]{images/GoogleGlassStarImmersion} }}
   	\qquad
	\subfloat[The immersion ``Explore stars'' allows the user to look around at stars using the built-in head motion tracker.]{{\includegraphics[width=70mm]{images/GoogleGlassStarImmersion2} }}
   	\qquad
		\caption{Cards can either display basic applications, such as a clock, or represent more in-depth applications (immersions), such as an application that lets the user look at a map of stars.}
		\label{GoogleGlassCards}
	\end{figure}

In order to move left on the timeline (forward in time) the user must swipe a finger backwards on the touchpad. In order to move right on the timeline (backward in time) the user must swipe a finger forward on the touchpad. The fact that the user must swipe backwards when stepping forward in time might not seem especially intuitive. In western culture a timeline is normally represented as going from left to right. One example of that are books. However, one might think of this action as swiping cards behind the back. Swiping forward when stepping backwards in time would then in turn mean bringing cards placed behind the back into focus. Cards in the past are behind the user while cards in the future are in front of the user.

When the user wants to turn off Google Glass the user swipes down on the touchpad. Swiping down on the touchpad will put Google Glass in stand by. If the user wants to turn off Google Glass entirely, in other words power down the device, there is a power button on the opposite side of the touchpad. Holding down the power button for a few seconds will turn off Google Glass. For a better visual understanding of how Google Glass works see Figure \ref{GoogleGlassUI} as well as the video referenced in the caption.

Google Glass uses a Bone Conduction Transducer (BCT) to transfer sound to the user.\cite{GlassSpecs} The BCT transfers sound to the inner ear by conducting sound through the bones of the skull.\cite{boneConductionWiki} The advantage of this technique is that the sound maintains clarity, even in noisy environments. Also, since the user does not plug any earphone into the ears, outside sound is not blocked out.

Google Glass also features a 5 megapixels camera. The camera is placed between the touchpad and the display, as seen in Figure \ref{GoogleGlassHardware} (b), and is capable of capturing video at a 720p resolution. The camera can be used for video conferencing, as Google showed in 2012\cite{glassLiveDemo}, but the camera can for instance also be used when the user wants to scan a QR Code\footnote{See section \ref{subsec:qrcode}}.

The user may also interact with Google Glass using voice commands. As seen in Figure \ref{GoogleGlassCards} the home screen consists not only of a clock but also of the words ``ok glass'', in quotes. ``ok glass'' tells the user that voice commands are available. The voice command menu is accessed as soon as the user says the words ``ok glass''. Doing so brings up a list of voice command available, as seen in Figure \ref{voiceCommandMenu}.

	\begin{figure}[ht!]
		\centering
		\includegraphics[width=110mm]{images/GoogleGlassUI}
		\caption{TODO INSERT PICTURE OF VOICE COMMAND MENU}
		\label{voiceCommandMenu}
	\end{figure}

In order to progress further the user must say one of the options displayed out loud. Doing so will make Google Glass perform the task spoken. For instance, if the user where to say ``ok glass, take a picture'', Google Glass would take a picture.
%The main way for a user to give input to Google Glass is via the touchpanel that is mounted on the right hand side of Glass, along the frame. Users are able to swipe as well as tap, which gives them control similar to that of a Smart TV's user interface. Where with a TV controller the user would maneuver with a simple cross layout (up, down, right and left) the buttons have on Glass been replaced by a touchpanel.\\

% Insert image of Google Glass graphical user interface here!!!

%The graphical interface is displayed at the top right through a projection coming from the right on a thick piece of glass. This technique lines up the image with the users sight but does not give any projection outwards.\\

%The interface is built with cards. Each card represents an activity.\\

%What’s unique?
%Standards?
%\url{https://developers.google.com/glass/design/}

\subsection{Compared to Smartphones}
\label{comparedtophones}
%Compare to phones!
Despite being two very different devices, the mobile phone and Glass, Google's design recommendations are not vastly different for the two. For mobile phones the ask developers to think of simplicity and clarity. They put much emphasis on making things easy to use.\\

There are some differences however. For mobile phones Google also recommend that developers keep track of what the users have done in the past. They ask developers to remember the user's input history and customisation, all to make it easier for the user when they (hopefully) come back to the application.
\url{https://developer.android.com/design/index.html}\\

\url{https://developer.android.com/design/get-started/principles.html}







Google differ in how they want developers to design applications for mobile phone and Glass respectively. On mobile phones they are much more open to developers using their own ides. They encourage freedom and give more subtle hints of how to design. For instance they want developers to make applications fun and easy to use. They recommend consistency and a rewarding application.\\

Designing for Google Glass comes with a bit more restrictions. \\

% Color schemes
% pre defined layouts
% pre defined typography (fonts)









\subsection{Limitations with Google Glass}
\label{subsec:limitations}
%Screen size
%Resolution
%Information on screen (possible result/discussion)
%How about people wearing glasses normally? Will the boss require employees to get lenses?
An early concern with Google Glass came from people who wore regular glasses every day as Google Glass seemed to require their own separate frames. Isabelle Olsson at Google responded on the issue on April 12th 2012 with the following: ``We ideally want Project Glass to work for everyone, and we're experimenting with designs that are meant to be extendable to different types of frames.''.\cite{GoogleGlassFrameResponse}\\

Today many eyecare providers have been trained for Google Glass and Glass frames. These trained eyecare providers are however mostly located in the United States,\cite{frameProviders} but Google points out that many eyecare providers should be able to help replace the lenses on Google Glass' frame\cite{framesGlass}.\\ 

% not relevant reference
%\url{http://www.google.com/glass/help/frames/} 

\cite{ackerman13}\\
A more alarming concern has been that of the health of the eyes. Concerns regarded

%Eye pain? 
%\url{http://www.forbes.com/sites/eliseackerman/2013/03/04/could-google-glass-hurt-your-eyes-a-harvard-vision-scientist-and-project-glass-advisor-responds/}

%documented eye pain from looking at a screen for too long. Also conserns regarding looking at something that not both eyes can see. Can give headache and slighted eye aligntment.\\

A study performed in 2002\cite{laramee02}, regarding the effects of HMDs, showed that HMDs may only be of help to users under controlled forms. Whenever the surrounding gets too distracting, for instance within a moving crowd, performance goes down. The study however noted that pilots had been able to successfully turn HMD into something they could use to their advantage. Since the study was not done over a long period of time the participants was potentially not given enough time to get used to wearing and using their HMDs.\\

%HMD:s could potential be of great service to users as long as users take the time to get use to the HMD device.\cite{laramee02}\\

\subsection{Presenting Information Within a Limited Space}
\label{subsec:informationlimitedspace}
The principles behind designing for glass is to keep the information relevant. Google ranks different computational devices and services in terms of time periods. Google talks about how the cloud stores information ``forever'', a computer keeps about a years worth of information, a mobile phone is keeping track of the last month and Glass are for the present.

Therefore Google asks developers to keep the information relevant and simple. Glass is designed not to get in the way of the user and, as stated previously, be usable when the users wants to.\cite{glassDesignPrinciples}
%\url{https://developers.google.com/glass/design/principles}

Despite Glass being places close to the user's eye the amount of information that can be displayed is still very limited. Google have therefore provided developers with a few guidelines when writing text which will be presented on Glass. These guidelines, five in total, 



%*** REWRITE !!!!

\begin{itemize}
	\item \textbf{Keep it brief.} Be concise, simple and precise. Look for alternatives to long text such as reading the content aloud, showing images or video, or removing features.
	\item \textbf{Keep it simple.} Pretend you're speaking to someone who's smart and competent, but doesn't know technical jargon and may not speak English very well. Use short words, active verbs, and common nouns.
	\item \textbf{Be friendly.} Use contractions. Talk directly to the reader using second person ("you"). If your text doesn't read the way you'd say it in casual conversation, it's probably not the way you should write it.
	\item \textbf{Put the most important thing first.} The first two words (around 11 characters, including spaces) should include at least a taste of the most important information in the string. If they don't, start over. Describe only what's necessary, and no more. Don't try to explain subtle differences. They will be lost on most users.
	\item \textbf{Avoid repetition.} If a significant term gets repeated within a screen or block of text, find a way to use it just once.
\end{itemize}

%*** REWRITE !!!


\subsection{Similar Products}
\label{subsec:similarproducts}
Today there are several products either already on the market or under development that are more or less similar to Google Glass. Following is a short list describing some of the competition Google Glass faces.

\begin{itemize}
\item \textbf{Microsoft Hololens}

Microsoft's offer in the augmented reality device space is a HUD that displays information over both of the user's eyes.\cite{hololens} The intention, according to Microsoft, is not to be an immediate competitor to Google Glass. Microsoft's aim is not to make the same device as Google Glass. Google Glass is meant to be worn all the time, at all times. Microsoft Hololens is rather a device users only puts on when they intend to use it.\\

But the mot striking difference between Microsoft Hololens and Google Glass lies in the interaction with the real world. Google Glass is a two dimensional (2D) display that sits slightly above the users line of sight (see \ref{subsec:googleglass}). Microsoft Hololens, on the other hand, is meant to interact with the world even further. \\

Microsoft intends to give the user tools to work in a three dimensional (3D) space. Microsoft's concept video\cite{hololensConceptVideo} of Microsoft Hololens shows examples of 3D modelling with the use of kinetic hand-movement detection, meaning that users will be able to see what they are working on from different angles simply by walking around it, just as if the object in question was real and had a physical mass.\\

\item \textbf{Recon Jet}\cite{reconJet} (HUD for sports)
[TODO FORTSÄTT GENOMLÄSNING HÄRIFRÅN EFTER RENSKRIVNING]
Recon Jet is a HMD developed by Recon Instruments. Recon Jet is suited for athletes. Because of the suitiness Recon Jet has been fitted with a display that has high contrast in order to give good readability in high ambient lighting. The display's virtual image appears as  a 30 inch wide screen at approximately 2 meters distance, to be compared with Google Glass [TODO SCREEN SIZE APPEARS TO BE WHAT?].\cite{reconJetSpecs}
\\
\\
todo more on recon jet\\

\item \textbf{GlassUp}\cite{glassUp} (Sued by Google)

GlassUp is an Italian company that received most of its founding through the crowdfunding site Indiegogo.\cite{glassUpIndiegogo} GlassUp have been sued by Google for being to similar to Google Glass [TODO REFERENCE]. GlassUp does however make distinctions between the two products. On GlassUp's Indiegogo page the company made the comparison that looking at Google Glass was like looking in the back view mirror in while GlassUp was like looking out the windscreen.\\

GlassUp displays information close to the center of the user's vision where as Google Glass keeps the information on the user's upper right. GlassUp claim that this decision was made so that there would be less strain on the user's eye.\\

\item \textbf{C Wear Interactive Glasses}\cite{penny}

C Wear Interactive Glasses is an industry focues device developed by Penny in V{\"a}ster{\aa}s, Sweden. It does not feature the same slick design many of the other virtual reality devices have (although many of them look terrible as well). One of the examples of this is that one key user interface is where the user can bite on a stick that is connected to the glasses. Probably because of the loud envornment that surronds most workers.\\

The GUI of Penny have the look of a normal PC application which comes from the fact that Penny keeps connected to a computer. However this might not be the most optimal interface since nagivation comes from head movements.
\end{itemize}
gfdsgdfs
\\
	\begin{figure}[ht!]
		\centering
    	\subfloat[Microsoft Hololens\cite{hololens}]{{\includegraphics[width=70mm]{images/similarProducts/hololens} }}
    \qquad
    	\subfloat[Recon Jet\cite{reconJet}]{{\includegraphics[width=70mm]{images/similarProducts/reconJet} }}
    \qquad
        \subfloat[GlassUp\cite{glassUpFeatures}]{{\includegraphics[width=70mm]{images/similarProducts/glassUp} }}
    \qquad
  	\subfloat[C Wear Interactive Glasses\cite{pennyProducts}]{{\includegraphics[width=70mm]{images/similarProducts/penny} }}
    \qquad
		\caption{Today there are many OHMD devices, either already on the market or under development. A more extensive list of devices can be found on wikipedia.\cite{ohmdWiki}}
		\label{imagesSimilarProducts}
	\end{figure}
\\
%\url{http://www.microsoft.com/microsoft-hololens/en-us}
%\url{http://www.searchenginejournal.com/google-glass-alternatives/67018/}
%\url{http://www.penny.se}

\subsection{Summary}
\label{subsec:summary}
o   Introduce problem area / give relevant background info
\\o   Introduction - Explain WHY you are doing this study
\\o   Information - Background / your study in the wider context
\\o   Similar work (projects, systems etc.)
\\o   Summary - for this chapter


%\subsection{Working handsfree}
%\label{subsec:workinghandsfree}
%Why?
%Multitasking (is this background or discussion?)
%Studies?
%\url{http://www.theguardian.com/science/2015/jan/18/modern-world-bad-for-brain-daniel-j-levitin-organized-mind-information-overload}
%\subsection{Multitasking on Google Glass}
%\label{subsec:multitaskingongoogleglass}







