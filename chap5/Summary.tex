While the Google Glass application and the smartphone application essentially have the same functionality, there are some differences between the two. The Google Glass application may be controlled using voice commands, which is a feature the smartphone application does not have. Otherwise both of the applications follow Google's design guidelines in terms of being easy to use and both the Google Glass application and the smartphone application focus on what is important in the application, which is to scan a QR code and view product information.

Examining the amount of text which may fit the screen on the different devices, it was shown that a full screen of text on the Samsung Galaxy SII covered about two and a half screens on Google Glass. When filling up an entire screen on the Samsung Galaxy SIII, the same amount of text filled up about three and a half screens on Google Glass. In other words, the amount of information that may be displayed on Google Glass is relatively limited compared to smartphone devices.

In terms of the other test results it can be seen that the Google Glass application is almost always slower that the smartphone application, on both the Samsung Galaxy SII and the Samsung Galaxy SIII. Google Glass was only about 0.4 seconds behind the Samsung Galaxy SII when scanning the QR code from different distances.

However, when the complexity of the QR code was varied, Google Glass proved to be less effective. While the Samsung Galaxy SII was not able to scan and decode the QR code which encoded 100 characters, Google Glass was not even able to scan and decode the QR code which encoded 50 characters. The Samsung Galaxy SIII managed to scan and decode all the different QR codes, although the more complex QR codes took about half a second longer to scan and decode.

The display time test was the test where Google Glass proved to be least effective. Both the Samsung Galaxy SII and the Samsung Galaxy SIII manage to send information to the display in under 0.1 seconds, even though the information size was 1 megabyte. Google Glass, however, always took longer than 0.1 seconds, even when the information size was only 1 kilobyte