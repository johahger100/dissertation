%o   Conclusion
%o   Project Evaluation
%o   Problems - How would you do this the next time?
%o   Future work
As seen in the test results the Google Glass application was almost always the slowest in all of the tests. However, as the test results only shows the times from when the QR code has been scanned until when the information is being displayed on screen the difference in time might not be of significance. 

After the user has scanned a QR code and the information is displayed, that is when the actually usage of the application starts. A user might be able to look past the fact that the Google Glass application is slower at scanning and presenting information as the advantages of Google Glass lies not in the speed of the device but rather in the user interface.

The fact that users do not have be use their hands is the major advantage of using Google Glass over a smartphone. Although voice commands are possible to implement in the smartphone application as well, the user must still hold the smartphone or place the smartphone on a surface in order to see the information being displayed. Google Glass puts the display in front of the user and does not require the user to hold the display in any way.

However, the fact that Google Glass was not able to scan more complex QR codes is definitely a negative aspect. Although the product ID:s used in this test were not long enough as to where complexity would become an issue, the complexity of the QR code will be an issue when the database contains many, many more products and the product ID:s are such that the complexity of the QR code is too high for Google Glass to handle.

%Hopefully Google will replace the camera on Google Glass with a better one. 

\subsection{Personal User Experience}
\label{subsec:personalexperience}
Having used about every weekday Google Glass for nearly four months there are a few comments that can ben made, and a few conclusions that can be drawn, simply from personal user experience. Please note that these comments are not based on any scientific studies, but are rather the opinions of the author of this dissertation.

% easy to use

% getting used to wearing Glass, takes a few days up to a week of contant usage. display could be irritating, but the point is that it is only lit up while using and the timed out

% the heat

% restrictions

% casual use, perhaps not industry at this time

\subsection{Back End Conclusions}

\subsection{Future Work}
\label{subsec:futurework}
\subsubsection{Official approval of Voice Commands}
The voice commands should be officially approved by Google.

\subsubsection{Customised Voice Command}
Construct own or use \url{https://github.com/RIVeR-Lab/google_glass_driver/blob/master/android/RobotManager/src/com/riverlab/robotmanager/voice_recognition/VoiceRecognitionThread.java}

\subsubsection{TextResultProcessor}
In the Google Glass application the class TextResultProcessor is not needed any more and should as such be removed. At this point the class in only used as middleware between an instance of the Products class and a list of CardPresenter. Instead the CardPresenter class should only be instanced once for each product, and keep the instance of the Products class.

The reason the TextResultProcessor class exists in the first place is due to how the Google Glass application was built originally, were all information presented was encoded directly in the QR code. At that point the TextResultProcessor was used when the encoded information was a text string. At this point the only information encoded in the QR codes are product ID:s.

The smartphone application already functions in this way, where the information stored in the instance of the Products class is used directly when a slide is created, instead of first being sorted through a middleware class.

[TODO possibly uml diagram of how the application works now and how it should work]

\subsubsection{A General Fragment}
The smartphone application should only have one general fragment instead of a bunch of different ones for different purposes. This should be done in order to be even more similar to the Google Glass application which uses the CardBuilder class, that is a general case that takes the layout as input.

\subsection{Concluding Remarks}