%o   Conclusion
%o   Project Evaluation
%o   Problems - How would you do this the next time?
%o   Future work
As seen in the test results the Google Glass application was almost always the slowest in all of the tests. However, as the test results only show the time from when the QR code has been scanned until when the information is being displayed on screen does not differ by more then about half a second between Google Glass and the Samsung Galaxy SII, and the difference in time might not be significant. 

After the user has scanned a QR code and the information is displayed, that is when the actual usage of the application starts. A user might be able to look past the fact that the Google Glass application is slower at scanning and presenting information as the advantages of Google Glass lies not in the speed of the device but rather in the user interface.

The fact that users do not have be use their hands is the major advantage of using Google Glass over a smartphone. Although voice commands are possible to implement in the smartphone application as well, the user must still hold the smartphone or place the smartphone on a surface in order to see the information being displayed. Google Glass puts the display in front of the user and does not require the user to hold the display in any way.

However, the fact that Google Glass was not able to scan more complex QR codes is definitely a negative aspect. Although the product ID:s used in this test were not long enough as to where complexity would become an issue, the complexity of the QR code will be an issue when the database contains many, many more products and the product ID:s are such that the complexity of the QR code is too high for Google Glass to handle.

As such, one clear improvement Google must make to Google Glass is to upgrade the camera. Not only to be able to compete with smartphone cameras, which, as seen, were much better already at the time of Google Glass' release, but simply to be able to identify QR codes with higher complexity and at longer distances.

Another important aspect of Google Glass is the amount of information the device can display on one screen. As shown in the text length test, a card on Google Glass which only displays text can hold somewhere between 200 and 220 characters, in comparison to smartphones with similar layout design, which can hold somewhere between 550 and 750 characters.

According to Google's guidelines for writing text for Google Glass, the text should be brief and simple, with the most important part of the text first and without repetition. Clearly, with the limitation on text length, writing short and precise text is important on Google Glass. Displaying information to the user without having to divide text up on several cards is important, for instance when writing instructions. Requiring the user to scroll back and forth between cards in order to read one instruction would not be good design, even though scrolling back and forth can be performed via voice commands, and as such instructions should be simple and brief, similar to the way they are in the application described in this dissertation.

\subsection{Personal User Experience}
\label{subsec:personalexperience}
Having used Google Glass about every weekday for nearly four months there are a few comments that can ben made, and a few conclusions that can be drawn, simply from personal user experience. Please note that these comments are not based on any scientific studies, but are rather the opinions of the author of this dissertation.

First of all it should be said that Google Glass is very easy to use. The core features of the device is easy to grasp, such as how to navigate using voice commands, and how to use som of the built-in applications, such as taking a picture or finding the shortest route to a specific location.

However, getting used to wearing Google Glass might take some time, somewhere between a few days up to a week of constant usage. Having a display slightly above the line of sight could become irritating. However, the Google Glass display does not stay on at all times. Instead the display times out after a short period when Google Glass is not being used.

Since Google Glass projects an image on to glass, which is then perceived by the user as the display, Google Glass is see through when not active. As such Google Glass, although noticeable, can be ignored when not active after a few days of usage. As stated previously, simply wearing Google Glass will take a few days getting used to.

The fact that Google Glass uses a BCT instead of regular headphones is appreciated as Google Glass is meant to be worn at all times, and as such users might not want to plug their ears. Doing so would potentially mean users would not be able to hear surrounding sounds, and would have to take off Google Glass when for instance talking to another person.

However, Google Glass does have some serious issues. One major issue is the fact that Google Glass overheats very easily. Simply interacting with an application a little too fast will cause Google Glass to overheat. When overheated the Google Glass touchpad, behind which the CPU and such sits, gets really hot. The heat can be felt by the user as the backside of the Google Glass touchpad lies against the temple, and could potentially be very distracting or even uncomfortable depending on the user.

When overheating, Google Glass will also not run as smoothly and a message will even be displayed on the home screen, which states ``Google Glass must cool down to run smoothly''.

Another downside of Google Glass is the restrictions Google has put on the development side. The fact that voice commands must be approved does make sense in terms of the voice commands which starts an application. Users wants to be able to distinguish one application from another. However, when running an application developers should be able to use the voice commands best suited for the application.

The ``ok glass'' overlay is also not a very good solution as part of the screen becomes more or less useless when the dark overlay cover such a big part of the screen. Developers should be able to better customise the design of applications, and not be restricted to the rules and guidelines Google has decided. The fact that Google wants to make sure that Google Glass applications follows a certain standard might seem like a good idea. However, the restrictions limits developers creativity.

All in all Google Glass feels like a device meant for more casual use at this time. Google Glass is not stable enough to handle more industry focused work, where the workload on the device could be very high. Used as a device which could display simple information, Google Glass seems well suited, but put under pressure and used a lot for longer periods of time is when Google Glass falls short.

% easy to use

% getting used to wearing Glass, takes a few days up to a week of contant usage. display could be irritating, but the point is that it is only lit up while using and the timed out

% the heat

% restrictions

% casual use, perhaps not industry at this time

\subsection{Project Evaluation}
Overall, the project went well. The project goal was to create a proof-of-concept of an instruction application using Google Glass, as well as to evaluate Google Glass in terms of the user experience. These goals have been reached  without any major setbacks. 

More difficult aspect of the project included understanding the Google Glass example application, BarcodeEye, as the application in some ways covered more than what the proof-of-concept application was meant to cover. For instance, BarcodeEye handled ISBN, which was not needed in this project. If redone, the BarcodeEye code could have been better refactored along the way, where unnecessary code would have been removed as soon as such code was deemed unnecessary. Doing so might have compressed the code somewhat and also made changes easier to implement. 

Being able to update an image on a slide also proved difficult, as the slides were to be loaded in faster than the images could be loaded in on the slides, and as such the user either had to wait for the application or only have a default image instead of the intended one. This was first solved by passing the \texttt{ImageView}, which contained the image, around, which is normally not good praxis as you rather want to pass the object going inside the view around as argument instead.

Through code review and evaluation the issue was solved by implementing a listener on the specific slide containing an image, such that when the image was loaded the method implemented through the listener was called, and the slide updated with the loaded image.

The project was developed through sprints, lasting two weeks each, with a sprint demonstration at the end of each sprint. The sprint demonstration consisted of a demonstration the work done so far to the supervisor, making sure the project was still on the right track and that the development was not falling behind schedule.

\subsection{Back-End Conclusions}
The dissertation involving the back-end part of the project, written by Richard Hoorn, concludes that while Google Glass is an impressive piece of technology, Google Glass was still outperformed by smartphones since Google Glass is not able to handle larger data sizes. Data sizes above one megabyte were very difficult or even impossible for Google Glass to handle.

A possible solution would be to not let Google Glass do all the work, but rather connect Google Glass to a smartphone, from which information is then streamed. As such, Google Glass would simply act as a display, displaying the information sent from the smartphone, and would not have to handle all of the downloaded information at the same time.

Hoorn also discussed the possibilities of headaches which Google Glass could cause and the fact that looking up at the screen could cause eye strain. These concerns have been adressed by Google, who states that Google Glass is not meant for usage of longer periods of time. Using Google Glass longer periods of time will also cause them to overheat, causing both performance issue as well as a very hot exterior. As such, Google Glass is perhaps not suited for industry use, where Google Glass would potentially be used over longer periods of time, but is rather suited and designed for the more casual user.

\subsection{Future Work}
\label{subsec:futurework}
Since Google has discontinued the Google Glass Explorer programme and as such Google Glass is not available for purchase at the time of writing this dissertation, it is somewhat hard to argue for any future work to be done on the application. However, that aside, some further testing should be done.

For instance the voice commands should be tested, both in a silent environment, but also in a noisy environment. The test should be done in order to better determine if Google Glass is a viable device to be used in an industry environment, where there is potentially a lot of noise. 

The application should also be tried out by a general public for an extensive period of time, and the results of such a test period evaluated in order to determine whether some parts of the application should be redesigned or not. Following Google's design guidelines is only one aspect of the development process. In the end it is the users who should be satisfied with the result.

\subsubsection{Official approval of Voice Commands}
The voice commands used in the Google Glass application have not yet been officially approved by Google, nor have they been submitted to Google. Since the discontinuation of the Google Glass Explorer programme Google has removed the voice command approval form, which developers were meant to fill in and submit to Google.

Having custom-made, contextual voice commands officially approved by Google was a requirement for Google Glass applications with custom-made, contextual voice commands to be released on MyGlass. If this requirement is still present when the new version of Google Glass is eventually is released remains to be seen.

Potentially customised voice commands could be designed using the voice recognition feature built in to the Android operating system. However, doing so would require a design similar to that of Google's contextual voice command system, which uses ``ok glass'' as the phrase that loads up the voice command menu and listens for other voice commands. An application which always listens for all voice commands could become difficult to use in a noisy environment since Google Glass would listen for several phrases, instead of one specific phrase.

%\subsubsection{Customised Voice Command}
%A potential option to getting the voice commands approved by Google would be to use customised voice commands, as suggested by a few number of people online .

%Construct own or use \url{https://github.com/RIVeR-Lab/google_glass_driver/blob/master/android/RobotManager/src/com/riverlab/robotmanager/voice_recognition/VoiceRecognitionThread.java}

\subsubsection{TextResultProcessor}
In the Google Glass application the class \texttt{TextResultProcessor} is not required any more and should as such be removed. At this point the \texttt{TextResultProcessor} class is only used as middleware between an instance of the \texttt{Products} class, sent as an argument from the \texttt{DownloadProductTask} class, and a list of \texttt{CardPresenter}. Instead the \texttt{CardPresenter} class should only be instanced once for each product, and keep the instance of the Products class itself.

The reason the \texttt{TextResultProcessor} class exists in the first place is due to how the Google Glass application was originally built, were all information presented was encoded directly in the QR code. At that point the \texttt{TextResultProcessor} was used when the encoded information was a text string. At this point the only information encoded in the QR codes are product ID:s.

The smartphone application already functions in this way, where the information stored in the instance of the \texttt{Products} class is used directly when a slide is created, instead of first being sorted through a middleware class.

%[TODO possibly uml diagram of how the application works now and how it should work]

\subsubsection{A General Fragment}
The smartphone application should only have one general fragment instead of different ones for different purposes. This should be done in order to be even more similar to the Google Glass application which uses the \texttt{CardBuilder} class, which is a general case that takes the layout as input. The smartphone application could be designed in a similar way, where a general fragment takes the layout as an argument. At this point there is one fragment for each individual layout.

\subsection{Concluding Remarks}
Google Glass is an interesting attempt at developing technology further. The potential of Google Glass is exciting. However, the fact that Google Glass is intended for use in short bursts, simply to take a picture or to send a message, makes it hard to argue in favour of using Google Glass in any situation apart from more casual ones. As long as any work performed on the device is not too complex or demanding in terms of hardware, Google Glass can be very helpful.

However, Google Glass should not be seen as a competitor to smartphones, but rather a complementary device which may be used together with a smartphone. In terms of displaying instructions to a user, who is assembling components, Google Glass could be useful as the user does not have to shift focus and look away. As long as the instructions are short, simple and do not take up too much data space, Google Glass would be the preferred device over smartphones.

However, the smartphone's bigger screen and stronger hardware make smartphones the preferred choice when the instructions are complex and contain, for instance, a video, as the video's data size potentially would be too large for Google Glass to handle without performance issues.

One solution would be to compress the video, both in length and data size, or to have the video stream from a smartphone on to Google Glass, using Google Glass and a smartphone as complements to each other. As Google continues to develop Google Glass behind closed doors, it will be interesting to see how Google Glass has evolved once it is revealed again. Until such time, one can only hope Google keeps improving on Google Glass, and relaxes of some of their restrictions, as Google Glass, despite its flaws, shows great potential.