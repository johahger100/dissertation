%o   Project goal and motivation
%o   Project summary and overview - the "red thread"
%o   Project results (brief summary)
%o   Dissertation Layout

%Nytta med projektet, bakomliggande motivering, hypotes kring resultat (Google Glass kommer vara bättre än smartphone eftersom handsfree and stuff), layout av rapporten.

%Prata allmänt om vad det finns för problem idag, mer specifikt vad kommer vår applikation att lösa, mixa med frågor som kan besvaras bland slutsatserna

Assembling components in to a fully functional product can be a tedious task. Although,aAn employee constructing many different products may however not be able to learn the process by heart, and will need to rely on instruction manuals.

However, carrying instruction manuals around can be tedious work in it self plus while assembling components the instructions will have to be put down. As such, the constructor's focus will have to shift and potentially the constructor may even need to use their hands in order to switch between instructions. 

Using a smartphone at least gets rid of any potential binders, but a smartphone still requires touch and as such will occupy at least on of the constructor's hands. New technology may fix that issue. Google Glass is a glass frame with a computational device, and a display positioned slightly above the user's line of sight. Google Glass may also be entirely controlled via voice commands, meaning that the user may operate Google Glass completely hands free.

This dissertation describeds the design, implementation and evaluation of an application for Google Glass. Using the application user's will be able to scan a QR code related to a specific product and get information on which components is needed to construct the specific product, as well as instructions on how to assemble the components.

The application will also be built for smartphones, giving a reference point for the evaluation as well as a mean of comparison for the Google Glass application. Using the Google Glass application users will be able to construct products by following instructions, but without having to deviate their focus from what they are doing, and neither will the users have to use any of their hands in orderto control the application.

\subsection{Hypothesis}
The hypothesis is that the Google Glass application will prove useful and valuable, more so than the smartphone application. Although Google Glass might not be the most powerful device, the features Google Glass has are the selling points. Google Glass should at least be close enough to the smartphone in terms of performance, giving weight to the argument that Google Glass should be used in order to make the assembling of components an easier task.

\subsection{Project Results}


\subsection{Dissertation Layout}
Chapter~\ref{sec:background} discuss relevant background information regarding Google Glass. The chapter will include an introduction to what Google Glass is, how they came about and what features they have. The background chapter will also discuss similar products to Google Glass, as well as compare Google Glass to smartphones. Finally chapter two will include some discussion on topics relevant to the project, including QR code and ways of presenting information.

The~\ref{sec:design}rd chapter is about the design of the project. The discussion revolves around how the application is intended to work, and what limitations may apply to the implementation of the application, both on Google Glass and smartphone. The third chapter also discusses the design of the tests done on the application.

Chapter~\ref{sec:implementation} describes the implementation part of the project. The flow of the application is described in detail. Specific aspect of the application is also described in more detail. The layout of the slides as well as the voice commands. The experimental setup and how the tests were performed is also described here.

In chapter~\ref{sec:resultevaluation} the results of the tests are presented. The results are presented along with comments regarding how the results are to be interpreted as well as comments on any potential error factors during testing.

The~\ref{sec:conclusion}th and final chapter contains conclusions on the project. The conclusions regards the test results, as well as conclusions on the project as a whole. Chapter six also includes comments based on personal user experience from using Google Glass for about three months. Finally future work is discussed and the report is concluded with some concluding remarks.

Attached to the dissertation, after the reference list, are appendixes. In appendix~\ref{sec:abbreviations} abbreviations used throughout the dissertation are listed. Appendix~\ref{app:results} shows all the individual test results. Project code can be found in appendix~\ref{app:code}. Appendix~\ref{app:projectspec} is the last appendix which contains the original project specification, written in Swedish.