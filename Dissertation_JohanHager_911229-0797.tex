%% Sample dissertation using the kau_report LaTeX class
%%
%% 950203  Michael Kelsey -- LaTeX2e format (cit_thesis)
%% 990210  Anna Brunstrom -- Modified to KAU requirements
%% 990420  Anna Brunstrom -- Copyright page added + some minor updates

%\documentclass{kaumasters}
\documentclass[12pt,twoside]{kau_report}

\usepackage{float}
\PassOptionsToPackage{hyphens}{url}\usepackage{hyperref}
\usepackage{url}
\usepackage{subfig}
\usepackage{tabularx}
\usepackage{listings}
\usepackage{color}
\usepackage{pdfpages}
\usepackage{mathtools}
\usepackage{chngcntr}
%\usepackage{refcheck}

\AtBeginDocument{\counterwithin{lstlisting}{section}}

\setlength{\fboxsep}{0pt}

\definecolor{dkgreen}{rgb}{0, 0.6, 0}
\definecolor{gray}{rgb}{0.5, 0.5, 0.5}
\definecolor{mauve}{rgb}{0.58, 0, 0.82}
\lstset
{
	frame=tb,
	aboveskip=3mm,
	belowskip=3mm,
	showstringspaces=false,
	columns=flexible,
	basicstyle={\small\ttfamily},
	numbers=left,
	numberstyle=\color{gray},
	keywordstyle=\color{blue},
	commentstyle=\color{dkgreen},
	stringstyle=\color{mauve},
	breaklines=true,
	breakatwhitespace=true,
	tabsize=3
}

% If you write in Swedish, uncomment line below
%\usepackage{swedish}

% Define the parameters in the preamble

\title{An Evaluation of Google Glass\\ 
\large Design, Implementation and Evaluation of an Instruction Application for Google Glass and Smartphones}

\pubnum{June, 2015}

\author{Johan H\"{a}ger}

\date{2015-06-05}

\advisor{Donald F. Ross}
\examiner{Thijs. J. Holleboom}

% The actual document starts here
\begin{document}


% Create the title page. If you write in swedish use the second
% line below instead.
\makekautitle
%\makeswekautitle

%% Create the copyright page. If you write in swedish use the second
% line below instead.
\copyrightpage
%\swecopyrightpage

 %Start roman numbering
\begin{frontmatter}

% Create approval page. select the appropriate command depending
% on how many authors you have. If you write in swedish use 
% one of the sweapproved lines below instead.
%\approved
\approved
%\approvedthree{Author One}{Author Two}{Author Three}
%\sweapproved
%\sweapprovedtwo{Author One}{Author Two}
%\sweapprovedthree{Author One}{Author Two}{Author Three}

% Create abstract 
\begin{abstract}
Assembling components in a production line could potentially be a tedious task, and done by the books, step-by-step. However, an employee who is constructing many different product may not know all the steps by heart. As such they will be reliant on an instruction manual. But an instruction manual must be carried around and lied down. Instead new technology might make the process more efficient. Google Glass puts a display slightly above the user's line of sight and can be controlled via voice commands, and as such solves many of the problems with carrying around instruction manuals. This dissertation is an evaluation of Google Glass and describes the design, implementation and evaluation of an instruction application for both Google Glass and smartphones. The smartphone version was done in order to provide a reference point as well as a method of comparison with the Google Glass application. Testing was done on the different steps of the application, from the QR code has been scanned until the information is being displayed to the user. The results show that Google Glass is almost always about half a second slower, in all steps, compared to smartphone equivalents. The conclusion is that Google must upgrade and improve on, especially, the hardware of Google Glass. Google Glas overheats a lot and the camera is bad. Google's restrictions also limits what developers might be able to do with the device. However, Google Glass is easy to use and has potential to become a more useful device in the future.
\\
% KEYWORDS
\\
\textbf{Keywords:} Google Glass, Android
%Motivation:
%Why do we care about the problem and the results? If the problem isn't obviously "interesting" it might be better to put motivation first; but if your work is incremental progress on a problem that is widely recognized as important, then it is probably better to put the problem statement first to indicate which piece of the larger problem you are breaking off to work on. This section should include the importance of your work, the difficulty of the area, and the impact it might have if successful.
%Problem statement:
%What problem are you trying to solve? What is the scope of your work (a generalized approach, or for a specific situation)? Be careful not to use too much jargon. In some cases it is appropriate to put the problem statement before the motivation, but usually this only works if most readers already understand why the problem is important.
%Approach:
%How did you go about solving or making progress on the problem? Did you use simulation, analytic models, prototype construction, or analysis of field data for an actual product? What was the extent of your work (did you look at one application program or a hundred programs in twenty different programming languages?) What important variables did you control, ignore, or measure?
%Results:
%What's the answer? Specifically, most good computer architecture papers conclude that something is so many percent faster, cheaper, smaller, or otherwise better than something else. Put the result there, in numbers. Avoid vague, hand-waving results such as "very", "small", or "significant." If you must be vague, you are only given license to do so when you can talk about orders-of-magnitude improvement. There is a tension here in that you should not provide numbers that can be easily misinterpreted, but on the other hand you don't have room for all the caveats.
%Conclusions:
%What are the implications of your answer? Is it going to change the world (unlikely), be a significant "win", be a nice hack, or simply serve as a road sign indicating that this path is a waste of time (all of the previous results are useful). Are your results general, potentially generalizable, or specific to a particular case?

\end{abstract}
\cleardoublepage


% If you write in swedish you need a second abstract in 
% English. Uncomment lines below for the second absract.
%\renewcommand{\abstractname}{Your English Title}
%\begin{abstract}
%Put the text of your second abstract here
%\end{abstract}


% Create acknowledgements (optional part)
\begin{acknowledgements}
Mats Persson - handledare Sogeti sprintdemo

Donald F. Ross - handledare kau

Kalle Henriksson - kodgranskning

Daniel Johansson - utlåning av testequipment kau

Sogeti i allmänhet

A final thank you to everyone who have been supportive and wished me good luck throughout the process of working on this project as well as writing this dissertation.

todo - förklara vad glassware är och använd det istället för application igenom rapporten?

% If you want acknowledgements they go here 
\end{acknowledgements}
\cleardoublepage

% Set up contents, list of figures and list of tables
  \tableofcontents
  \cleardoublepage

  \listoffigures
  \cleardoublepage

  \listoftables
  \cleardoublepage
  
  \lstlistoflistings
  \cleardoublepage

\end{frontmatter}
% End of roman numbering

% Here comes the main body, it's your job to fill this in...
\section{Introduction}
\label{sec:introduction}
o   Project goal and motivation
o   Project summary and overview - the "red thread"
o   Project results (brief summary)
o   Dissertation Layout

Nytta med projektet, bakomliggande motivering, hypotes kring resultat (Google Glass kommer vara bättre än smartphone eftersom handsfree and stuff), layout av rapporten.

Prata allmänt om vad det finns för problem idag, mer specifikt vad kommer vår applikation att lösa, mixa med frågor som kan besvaras bland slutsatserna

\subsection{Hypothesis}


\subsection{Project Results}


\subsection{Dissertation Layout}
Chapter~\ref{sec:background} discuss relevant background information regarding Google Glass. The chapter will include an introduction to what Google Glass is, how they came about and what features they have. The background chapter will also discuss similar products to Google Glass, as well as compare Google Glass to smartphones. Finally chapter two will include some discussion on topics relevant to the project, including QR code and ways of presenting information.

The~\ref{sec:design}rd chapter is about the design of the project. The discussion revolves around how the application is intended to work, and what limitations may apply to the implementation of the application, both on Google Glass and smartphone. The third chapter also discusses the design of the tests done on the application.

Chapter~\ref{sec:implementation} describes the implementation part of the project. The flow of the application is described in detail. Specific aspect of the application is also described in more detail. The layout of the slides as well as the voice commands. The experimental setup and how the tests were performed is also described here.

In chapter~\ref{sec:resultevaluation} the results of the tests are presented. The results are presented along with comments regarding how the results are to be interpreted as well as comments on any potential error factors during testing.

The~\ref{sec:conclusion}th and final chapter contains conclusions on the project. The conclusions regards the test results, as well as conclusions on the project as a whole. Chapter six also includes comments based on personal user experience from using Google Glass for about three months. Finally future work is discussed and the report is concluded with some concluding remarks.

Attached to the dissertation, after the reference list, are appendixes. In appendix~\ref{sec:abbreviations} abbreviations used throughout the dissertation are listed. Appendix~\ref{app:results} shows all the individual test results. Project code can be found in appendix~\ref{app:code}. Appendix~\ref{app:projectspec} is the last appendix which contains the original project specification, written in Swedish.
\cleardoublepage

\section{Background}
\label{sec:background}
%How did Google Glass come about? History!
Mobiles today are used as a nervous tick. It is a distraction and something that pulls your attention away from the real world. At least that is what Sergey Brin, one of the founders of Google, claimed during a Ted Talk presentation in February 2013.\cite{tedtalkWhyGlass} Brin stated that if he was a smoker he would probably light a cigarette at those times when he now uses his phone.
%\url{http://www.ted.com/talks/sergey_brin_why_google_glass}

Brin and his team wanted to create something that would make interaction with technology easy and fast and not distract from reality. They wanted to keep the information more handy and close by than a phone stuck in the users pocket. But they also wanted to keep the line of sight free. Thad Starner, technical lead/manager on Google Glass, wrote in an article in 2013,\cite{6504855} that he sought out to build something as intuitive as a watch. An extension of the self, as he stated. And so Google started working on Project Glass. 




% The idea behind Glass was to minimise the time between intention and action
% users should not have to bring up something from their pockets each time they want to interact with technology
% they should be able to just simply interact
% They wanted to create something as intuitive as a watch. 
% checking the watch is something a user might do without actually thinking about what the time is.
% they might have to check again if someone were to ask them what the time is.
% Glass should be an extension of the self rather than another device.
% // Thad Starner - technical lead/manager on Google Glass

% Sergey Brin, one of the founders at Google, has similar ideas
% ted talk he spoke about how checking the phone was something he did without reason
% putting notifications more easily accessible would minimise interaction with technology because the user
% would not have to check if any updates have come in, they would know right away



April 4th, 2012, Google announced ``Project Glass''. Glass was under development for several years at Google's research and development department, Google X. The idea behind the device was to make technology easier to access but also to only be available when the users wants to. Serge Brin, on of the founders of Google, did a Ted Talk in February 2013 where he talked about why they decided to produce the device. His argument was that smartphones was something users kept looking at even when they did not want to. They might have missed something. Instead Google wanted to create a device that would notify the user [TODO WTF, doesn't a phone do this!?!?!]

Thad Starner, technical lead/manager on Google Glass, claimed that Glass is supposed to be an extension of the self. He compared the device to a watch. A watch is an easy to access device that seems natural to its wearers. Starner claimed that when an individual looks at his or her's watch 




\subsection{How does Google Glass work?}
\label{subsec:googleglass}
Google Glass, or simply ``Glass'' as the device is known within Google, is a head-mounted display (HMD) that can be seen as an augmented reality device (see Section~\ref{subsubsec:hmd} and Section~\ref{subsubsec:ar} respectively) designed to bring notifications to the user more easily than a smartphone does. Google Glass is shown in Figure~\ref{GoogleGlassHardware}. According to Google ``Glass is designed to be there when the user needs it and to stay out of the way when the user does not''~\cite{glassDesignPrinciples}. Google Glass is meant to give the user relevant information at relevant times.
%\url{https://developers.google.com/glass/design/principles}

	\begin{figure}[ht!]
		\centering
    \subfloat[The user can control Google Glass with the touchpad.]{{\includegraphics[width=70mm]{images/GoogleGlassHardwareTouchpad} }}
    \qquad
    \subfloat[The display sits slightly above the user's line of sight, on the right hand side.]{{\includegraphics[width=70mm]{images/GoogleGlassHardware} }}
    \qquad
		%\includegraphics[width=110mm]{images/GoogleGlassHardware}
		\caption{Google Glass is equipped with a touchpad and a camera~\cite{ImagesGoogleGlassUI}.}
		\label{GoogleGlassHardware}
	\end{figure}

Google Glass is partially controlled with a touchpad, but can also be controlled through voice commands. The touchpad sits on the right hand side of the user's glass frame and runs from the temple to the ear (see in Figure~\ref{GoogleGlassHardware}~(a)). When the user touches anywhere on the touchpad Google Glass ``wakes up'' from stand by and displays the start screen (which consists of a clock). The display is mounted above the user's line of sight, on the right hand side (see Figure~\ref{GoogleGlassHardware}~(b)) and can be slightly adjusted so that the user can see all that is currently being displayed.

The display is a projection that goes through an optic lense in the glass piece, seen in Figure~\label{GoogleGlassHardware}~(b), which creates a virtual image. A virtual image is an image that, projected through optic lenses, appears to be located at a point where the actual projection is not~\cite{virtualImageWiki}. In the case of Google Glass the display appears to be located further away from the user than the display actually is. The display is said to be equivalent of a 25 inch high definition screen seen from a distance of approximately 2.5 meters~\cite{GlassSpecs}.

%
%
%
%
%
%
%
%	\begin{figure}[ht!]
%		\centering
%		\includegraphics[width=110mm]{images/GoogleGlassUI}
%		\caption{A virtual representation of the Google Glass user interface as the graphical user interface is perceived by the user.\cite{ImagesGoogleGlassUI}}
%		\label{GoogleGlassUI}
%	\end{figure}
%	
%	
%	
%
%
%
%
%The graphical user interface (GUI) is called a timeline (see Figure \ref{GoogleGlassUI}). The timeline consists of a row of cards. Cards are basic applications such as a clock or information about the weather. Cards can also represent more in-depth applications, on Google Glass called ``Immersions''. An immersion handles activities such as browsing an image gallery or playing a game.\\
%
%On the timeline cards to the left of the home screen are upcoming activities such as an event in the user's calendar or an upcoming flight. Cards to the right of the home screen are from the past. Cards from the past will for instance show text messages or photos. When the user wants to turn of Google Glass the user swipes down on the touchpad. Swiping down on the touchpad will put Google Glass in stand by. If the user wants to turn of Google Glass entirely there is a power button on the opposite side of the touchpad. Holding down the power button for a few seconds will turn of Google Glass. For a better visual understanding of how Google Glass works see Figure \ref{GoogleGlassUI} as well as the video referenced in the caption.\\
%
%Glass uses a small display placed to the upper right of the user's line of sight and is mounted on the user as a regular pair of glasses. Equipped with a camera, microphone and speakers it is capable of performing a lot of the tasks users normally would do on a smartphone such as taking photographs, video chatting, writing text messages.
%
%
\subsubsection{Head-Mounted Display (HMD)}
\label{subsubsec:hmd}
A head-mounted display (HMD)~\cite{hmdWiki} is a device that is worn on the head and that places a small display in front of one or both of the user's eyes. The device can either be a stand alone device or a part of a helmet. A branch of HMDs are optical head-mounted displays (OHMDs)~\cite{ohmdWiki}. A OHMD is a HMD with a see-through display, for instance Google Glass.
%A head mounted display is just as it sounds a display that goes on ones head and that displays stuff

\subsubsection{Heads-Up Display (HUD)}
\label{subsubsec:hud}
A heads-up display (HUD)~\cite{hudWiki} is defined as any transparent display that, when presenting information, does not require userss to look away from their usual viewpoints. In other words, a HUD may be a HMD and a HMD may be a HUD. While a HMD is always worn on the head a HUD can be a stand-alone display. In contrast a HUD must be a transparent display. A requirement a HMD does not have. A OHMD, however, is always a HUD since a OHMD has a transparent display.

\subsubsection{Virtual Reality}
\label{subsubsec:vr}
Virtual reality~\cite{virtualRealityDef} is defined as a computer generated simulation that enables users to interact with a three-dimensional environment. Virtual realities are common in interactive mediums such as video games. Virtual realities can also be combined with a HMD in order to completely engulf the user in the virtual reality. One such example is the Oculus Rift, seen in Figure~\ref{OculusRift}, that completely covers the user's eyes, allowing the user to experience the virtual reality.

	\begin{figure}[ht!]
		\centering
		\includegraphics[width=110mm]{images/OculusRift}
		\caption{The HMD ``Oculus Rift'' is a virtual reality device~\cite{ImagesOculusRift}.}
		\label{OculusRift}
	\end{figure}

Google Glass is able to display a virtual reality but does not work as a virtual reality device. Google Glass only covers a small part of the user's field of vision and as such does not have the capability of simulating a three-dimensional, interactive, environment in contrast to the Oculus Rift. Oculus Rift, unlike Google Glass, is able to replace the user's reality with a completely virtual reality since Oculus Rift completely covers the user's eyes.

\subsubsection{Augmented Reality}
\label{subsubsec:ar}
Augmented reality~\cite{augmentedRealityDef} is defined as the combination of reality (or what is within current context being perceived as reality~\footnote{Augmented reality is for instance common in video games to give the player environmental and health information.}) with useful, computer generated, data. Augmented reality, unlike virtual reality, is not meant to replace reality, but rather to enhance interaction with the current reality.

A HUD may create an augmented reality. The reason a HUD does not always create an augmented reality is due to the fact that the information being presented might not be useful within the current context. An augmented reality is, as stated above, meant to enhance reality, while a HUD does not have that requirement.

Google Glass is a HUD that has the potential (and intent) to create an augmented reality. Google Glass is intended to present useful information to users while not distracting them from reality. One example of useful information that could enhance users interaction with reality would be a shopping list while users are out shopping, as seen in Figure~\ref{GlassShopping}.

	\begin{figure}[ht!]
		\centering
		\includegraphics[width=110mm]{images/GoogleGlassKeepRevelant}
		\caption{A shopping list while the user is out shopping is useful information~\cite{glassDesignPrinciples}.}
		\label{GlassShopping}
	\end{figure}
	
	
%\subsubsection{Augmented Reality vs Virtual Reality}
%\label{subsubsec:augmentedrealityvsvirtualreality}

%[TODO WRITE ABOUT HUD AND HMD]

%When discussing head mounted displays it is possible that the first image that pops into ones head is similar to Figure \ref{OculusRift}. What is important to note about Oculus Rift (Figure \ref{OculusRift}) and other similar product that completely covers the user's eyes is that these are all virtual reality devices. Virtual reality is not the same as augmented reality, which is what Google Glass gives the user.

%The difference lies in how much of what the user can see is computer generated. In a virtual reality the entire environment is computer generated. Augmented reality on the other hand is based in reality where computer generated elements of the environment help enhance reality. In other words: virtual reality replaces reality and augmented reality enhance reality. Since Google Glass does not remove the user from reality but rather display information that can be consumed at the same time as the user experience the real world Google Glass is an augmented reality device compared to Oculus Rift which is a virtual reality device.

%TODO --- ADD HUD vs AUGMENTED REALITY!!!!!!


%What is it?
%Augmented Reality vs Virtual Reality
%Define Augmented Reality
%Define Virtual Reality


\subsection{User Interface}
\label{subsec:userinterface}
Google Glass' graphical user interface (GUI) is called a timeline (see Figure \ref{GoogleGlassUI}).\cite{ImagesGoogleGlassUI} The timeline consists of a row of cards. Cards are basic applications such as a clock (see Figure \ref{GoogleGlassCards} (a)) or information about the weather. Cards can also represent more in-depth applications, on Google Glass called ``Immersions'' (see Figure \ref{GoogleGlassCards} (b) and (c)). Immersions handles activities such as browsing an image gallery or playing a game.

	\begin{figure}[ht!]
		\centering
		\includegraphics[width=110mm]{images/GoogleGlassUI}
		\caption{A visual representation of the Google Glass GUI as the GUI is perceived by the user. In reality only one card can be displayed at a time.\cite{ImagesGoogleGlassUI}}
		\label{GoogleGlassUI}
	\end{figure}

The first screen the user sees when starting up Google Glass is the home screen. The home screen displays a clock and also shows the text "ok glass", as seen in Figure \ref{GoogleGlassCards} (a). The home screen is a part of the timeline and acts as the center point. Cards to the left of the home screen are upcoming activities such as an event in the user's calendar or an upcoming flight. Cards to the right of the home screen are from the past. Cards from the past will for instance show text messages or photos.

	\begin{figure}[ht!]
		\centering
   	\subfloat[The Google Glass home screen is a card that displays a clock.]{{\includegraphics[width=70mm]{images/GoogleGlassHomescreen} }}
  	 \qquad
   	\subfloat[The card ``Explore stars'' represents an immersion.]{{\includegraphics[width=70mm]{images/GoogleGlassStarImmersion} }}
   	\qquad
	\subfloat[The immersion ``Explore stars'' allows the user to look around at stars using the built-in head motion tracker.]{{\includegraphics[width=70mm]{images/GoogleGlassStarImmersion2} }}
   	\qquad
		\caption{Cards can either display basic applications, such as a clock, or represent more in-depth applications (immersions), such as an application that lets the user look at a map of stars.}
		\label{GoogleGlassCards}
	\end{figure}

In order to move left on the timeline (forward in time) the user must swipe a finger backwards on the touchpad. In order to move right on the timeline (backward in time) the user must swipe a finger forward on the touchpad. The fact that the user must swipe backwards when stepping forward in time might not seem especially intuitive. In western culture a timeline is normally represented as going from left to right. One example of that are books. However, one might think of this action as swiping cards behind the back. Swiping forward when stepping backwards in time would then in turn mean bringing cards placed behind the back into focus. Cards in the past are behind the user while cards in the future are in front of the user.

When the user wants to turn off Google Glass the user swipes down on the touchpad. Swiping down on the touchpad will put Google Glass in stand by. If the user wants to turn off Google Glass entirely, in other words power down the device, there is a power button on the opposite side of the touchpad. Holding down the power button for a few seconds will turn off Google Glass. For a better visual understanding of how Google Glass works see Figure \ref{GoogleGlassUI} as well as the video referenced in the caption.

Google Glass uses a Bone Conduction Transducer (BCT) to transfer sound to the user.\cite{GlassSpecs} The BCT transfers sound to the inner ear by conducting sound through the bones of the skull.\cite{boneConductionWiki} The advantage of this technique is that the sound maintains clarity, even in noisy environments. Also, since the user does not plug any earphone into the ears, outside sound is not blocked out.

Google Glass also features a 5 megapixels camera. The camera is placed between the touchpad and the display, as seen in Figure \ref{GoogleGlassHardware} (b), and is capable of capturing video at a 720p resolution. The camera can be used for video conferencing, as Google showed in 2012\cite{glassLiveDemo}, but the camera can for instance also be used when the user wants to scan a QR Code\footnote{See section \ref{subsec:qrcode}}.

The user may also interact with Google Glass using voice commands. As seen in Figure \ref{GoogleGlassCards} the home screen consists not only of a clock but also of the words ``ok glass'', in quotes. ``ok glass'' tells the user that voice commands are available. The voice command menu is accessed as soon as the user says the words ``ok glass''. Doing so brings up a list of voice command available, as seen in Figure \ref{voiceCommandMenu}.

	\begin{figure}[ht!]
		\centering
		\includegraphics[width=110mm]{images/GoogleGlassUI}
		\caption{TODO INSERT PICTURE OF VOICE COMMAND MENU}
		\label{voiceCommandMenu}
	\end{figure}

In order to progress further the user must say one of the options displayed out loud. Doing so will make Google Glass perform the task spoken. For instance, if the user where to say ``ok glass, take a picture'', Google Glass would take a picture.
%The main way for a user to give input to Google Glass is via the touchpanel that is mounted on the right hand side of Glass, along the frame. Users are able to swipe as well as tap, which gives them control similar to that of a Smart TV's user interface. Where with a TV controller the user would maneuver with a simple cross layout (up, down, right and left) the buttons have on Glass been replaced by a touchpanel.\\

% Insert image of Google Glass graphical user interface here!!!

%The graphical interface is displayed at the top right through a projection coming from the right on a thick piece of glass. This technique lines up the image with the users sight but does not give any projection outwards.\\

%The interface is built with cards. Each card represents an activity.\\

%What’s unique?
%Standards?
%\url{https://developers.google.com/glass/design/}

\subsection{Compared to Smartphones}
\label{comparedtophones}
%Compare to phones!
Despite being two very different devices, the mobile phone and Glass, Google's design recommendations are not vastly different for the two. For mobile phones the ask developers to think of simplicity and clarity. They put much emphasis on making things easy to use.\\

There are some differences however. For mobile phones Google also recommend that developers keep track of what the users have done in the past. They ask developers to remember the user's input history and customisation, all to make it easier for the user when they (hopefully) come back to the application.
\url{https://developer.android.com/design/index.html}\\

\url{https://developer.android.com/design/get-started/principles.html}







Google differ in how they want developers to design applications for mobile phone and Glass respectively. On mobile phones they are much more open to developers using their own ides. They encourage freedom and give more subtle hints of how to design. For instance they want developers to make applications fun and easy to use. They recommend consistency and a rewarding application.\\

Designing for Google Glass comes with a bit more restrictions. \\

% Color schemes
% pre defined layouts
% pre defined typography (fonts)









\subsection{Limitations with Google Glass}
\label{subsec:limitations}
%Screen size
%Resolution
%Information on screen (possible result/discussion)
%How about people wearing glasses normally? Will the boss require employees to get lenses?
An early concern with Google Glass came from people who wore regular glasses every day as Google Glass seemed to require their own separate frames. Isabelle Olsson at Google responded on the issue on April 12th 2012 with the following: ``We ideally want Project Glass to work for everyone, and we're experimenting with designs that are meant to be extendable to different types of frames.''.\cite{GoogleGlassFrameResponse}\\

Today many eyecare providers have been trained for Google Glass and Glass frames. These trained eyecare providers are however mostly located in the United States,\cite{frameProviders} but Google points out that many eyecare providers should be able to help replace the lenses on Google Glass' frame\cite{framesGlass}.\\ 

% not relevant reference
%\url{http://www.google.com/glass/help/frames/} 

\cite{ackerman13}\\
A more alarming concern has been that of the health of the eyes. Concerns regarded

%Eye pain? 
%\url{http://www.forbes.com/sites/eliseackerman/2013/03/04/could-google-glass-hurt-your-eyes-a-harvard-vision-scientist-and-project-glass-advisor-responds/}

%documented eye pain from looking at a screen for too long. Also conserns regarding looking at something that not both eyes can see. Can give headache and slighted eye aligntment.\\

A study performed in 2002\cite{laramee02}, regarding the effects of HMDs, showed that HMDs may only be of help to users under controlled forms. Whenever the surrounding gets too distracting, for instance within a moving crowd, performance goes down. The study however noted that pilots had been able to successfully turn HMD into something they could use to their advantage. Since the study was not done over a long period of time the participants was potentially not given enough time to get used to wearing and using their HMDs.\\

%HMD:s could potential be of great service to users as long as users take the time to get use to the HMD device.\cite{laramee02}\\

\subsection{Presenting Information Within a Limited Space}
\label{subsec:informationlimitedspace}
The principles behind designing for glass is to keep the information relevant. Google ranks different computational devices and services in terms of time periods. Google talks about how the cloud stores information ``forever'', a computer keeps about a years worth of information, a mobile phone is keeping track of the last month and Glass are for the present.

Therefore Google asks developers to keep the information relevant and simple. Glass is designed not to get in the way of the user and, as stated previously, be usable when the users wants to.\cite{glassDesignPrinciples}
%\url{https://developers.google.com/glass/design/principles}

Despite Glass being places close to the user's eye the amount of information that can be displayed is still very limited. Google have therefore provided developers with a few guidelines when writing text which will be presented on Glass. These guidelines, five in total, 



%*** REWRITE !!!!

\begin{itemize}
	\item \textbf{Keep it brief.} Be concise, simple and precise. Look for alternatives to long text such as reading the content aloud, showing images or video, or removing features.
	\item \textbf{Keep it simple.} Pretend you're speaking to someone who's smart and competent, but doesn't know technical jargon and may not speak English very well. Use short words, active verbs, and common nouns.
	\item \textbf{Be friendly.} Use contractions. Talk directly to the reader using second person ("you"). If your text doesn't read the way you'd say it in casual conversation, it's probably not the way you should write it.
	\item \textbf{Put the most important thing first.} The first two words (around 11 characters, including spaces) should include at least a taste of the most important information in the string. If they don't, start over. Describe only what's necessary, and no more. Don't try to explain subtle differences. They will be lost on most users.
	\item \textbf{Avoid repetition.} If a significant term gets repeated within a screen or block of text, find a way to use it just once.
\end{itemize}

%*** REWRITE !!!


\subsection{Similar Products}
\label{subsec:similarproducts}
Today there are several products either already on the market or under development that are more or less similar to Google Glass. Following is a short list describing some of the competition Google Glass faces.

\begin{itemize}
\item \textbf{Microsoft Hololens}

Microsoft's offer in the augmented reality device space is a HUD that displays information over both of the user's eyes.\cite{hololens} The intention, according to Microsoft, is not to be an immediate competitor to Google Glass. Microsoft's aim is not to make the same device as Google Glass. Google Glass is meant to be worn all the time, at all times. Microsoft Hololens is rather a device users only puts on when they intend to use it.\\

But the mot striking difference between Microsoft Hololens and Google Glass lies in the interaction with the real world. Google Glass is a two dimensional (2D) display that sits slightly above the users line of sight (see \ref{subsec:googleglass}). Microsoft Hololens, on the other hand, is meant to interact with the world even further. \\

Microsoft intends to give the user tools to work in a three dimensional (3D) space. Microsoft's concept video\cite{hololensConceptVideo} of Microsoft Hololens shows examples of 3D modelling with the use of kinetic hand-movement detection, meaning that users will be able to see what they are working on from different angles simply by walking around it, just as if the object in question was real and had a physical mass.\\

\item \textbf{Recon Jet}\cite{reconJet} (HUD for sports)
[TODO FORTSÄTT GENOMLÄSNING HÄRIFRÅN EFTER RENSKRIVNING]
Recon Jet is a HMD developed by Recon Instruments. Recon Jet is suited for athletes. Because of the suitiness Recon Jet has been fitted with a display that has high contrast in order to give good readability in high ambient lighting. The display's virtual image appears as  a 30 inch wide screen at approximately 2 meters distance, to be compared with Google Glass [TODO SCREEN SIZE APPEARS TO BE WHAT?].\cite{reconJetSpecs}
\\
\\
todo more on recon jet\\

\item \textbf{GlassUp}\cite{glassUp} (Sued by Google)

GlassUp is an Italian company that received most of its founding through the crowdfunding site Indiegogo.\cite{glassUpIndiegogo} GlassUp have been sued by Google for being to similar to Google Glass [TODO REFERENCE]. GlassUp does however make distinctions between the two products. On GlassUp's Indiegogo page the company made the comparison that looking at Google Glass was like looking in the back view mirror in while GlassUp was like looking out the windscreen.\\

GlassUp displays information close to the center of the user's vision where as Google Glass keeps the information on the user's upper right. GlassUp claim that this decision was made so that there would be less strain on the user's eye.\\

\item \textbf{C Wear Interactive Glasses}\cite{penny}

C Wear Interactive Glasses is an industry focues device developed by Penny in V{\"a}ster{\aa}s, Sweden. It does not feature the same slick design many of the other virtual reality devices have (although many of them look terrible as well). One of the examples of this is that one key user interface is where the user can bite on a stick that is connected to the glasses. Probably because of the loud envornment that surronds most workers.\\

The GUI of Penny have the look of a normal PC application which comes from the fact that Penny keeps connected to a computer. However this might not be the most optimal interface since nagivation comes from head movements.
\end{itemize}
gfdsgdfs
\\
	\begin{figure}[ht!]
		\centering
    	\subfloat[Microsoft Hololens\cite{hololens}]{{\includegraphics[width=70mm]{images/similarProducts/hololens} }}
    \qquad
    	\subfloat[Recon Jet\cite{reconJet}]{{\includegraphics[width=70mm]{images/similarProducts/reconJet} }}
    \qquad
        \subfloat[GlassUp\cite{glassUpFeatures}]{{\includegraphics[width=70mm]{images/similarProducts/glassUp} }}
    \qquad
  	\subfloat[C Wear Interactive Glasses\cite{pennyProducts}]{{\includegraphics[width=70mm]{images/similarProducts/penny} }}
    \qquad
		\caption{Today there are many OHMD devices, either already on the market or under development. A more extensive list of devices can be found on wikipedia.\cite{ohmdWiki}}
		\label{imagesSimilarProducts}
	\end{figure}
\\
%\url{http://www.microsoft.com/microsoft-hololens/en-us}
%\url{http://www.searchenginejournal.com/google-glass-alternatives/67018/}
%\url{http://www.penny.se}

\subsection{Summary}
\label{subsec:summary}
o   Introduce problem area / give relevant background info
\\o   Introduction - Explain WHY you are doing this study
\\o   Information - Background / your study in the wider context
\\o   Similar work (projects, systems etc.)
\\o   Summary - for this chapter


%\subsection{Working handsfree}
%\label{subsec:workinghandsfree}
%Why?
%Multitasking (is this background or discussion?)
%Studies?
%\url{http://www.theguardian.com/science/2015/jan/18/modern-world-bad-for-brain-daniel-j-levitin-organized-mind-information-overload}
%\subsection{Multitasking on Google Glass}
%\label{subsec:multitaskingongoogleglass}








\cleardoublepage

\section{Design}
\label{sec:design}
%android studio vs eclipse with android sdk
%\url{http://wahidgazzah.olympe.in/integrating-zxing-in-your-android-app-as-standalone-scanner/}

%API level 12 because getIntent.getExtras.getString requires it. want default value there incase of error

\subsection{Presenting Information on Google Glass}
\label{subsec:informationlimitedspace}
[TODO Design guidelines Google Glass]

Margins layout templates

Text, Font

Google Glass Design tool

Keep it relevant  - shopping list

The principles behind designing for glass is to keep the information relevant. Google ranks different computational devices and services in terms of time periods. Google talks about how the cloud stores information ``forever'', a computer keeps about a years worth of information, a mobile phone is keeping track of the last month and Glass are for the present.

Therefore Google asks developers to keep the information relevant and simple. Glass is designed not to get in the way of the user and, as stated previously, be usable when the users wants to~\cite{glassDesignPrinciples}.

As part of the design guidelines for Google Glass, Google provides developers with a card layout template, seen in Figure~\ref{CardDesignStyle}.TODO The thick blue stripe almost at the bottom marks the footer. In the footer supplementary information, such as a user name or a timestamp.

	\begin{figure}[ht!]
		\centering
		\includegraphics[width=110mm]{images/standard-template}
		\caption{Google's design guidelines include a card layout template~\cite{glassDesignStyle}.}
		\label{GlassDesignStyle}
	\end{figure}

Google Glass is placed very close to the user's eye which makes the small projection seem Despite the fact that the display for Google Glass is placed very close to the user's eye the amount of information that can be displayed is still very limited. Google have therefore provided developers with a few guidelines when writing text which will be presented on Glass~\cite{glassDesignStyle}. These guidelines, five in total, reads...

\begin{itemize}
	\item \textbf{Keep it brief.} Be concise, simple and precise. Look for alternatives to long text such as reading the content aloud, showing images or video, or removing features.
	\item \textbf{Keep it simple.} Pretend you're speaking to someone who's smart and competent, but doesn't know technical jargon and may not speak English very well. Use short words, active verbs, and common nouns.
	\item \textbf{Be friendly.} Use contractions. Talk directly to the reader using second person ("you"). If your text doesn't read the way you'd say it in casual conversation, it's probably not the way you should write it.
	\item \textbf{Put the most important thing first.} The first two words (around 11 characters, including spaces) should include at least a taste of the most important information in the string. If they don't, start over. Describe only what's necessary, and no more. Don't try to explain subtle differences. They will be lost on most users.
	\item \textbf{Avoid repetition.} If a significant term gets repeated within a screen or block of text, find a way to use it just once.
\end{itemize}

\subsubsection{Glassware Flow Designer}
Google also provides developers with a design tool to help them visualise applications prior to implementation. The design tool, called ``Glassware Flow Designer''~\cite{glasswareFlowDesigner}, allows developers to discover recommended design patterns and to see the flow of the application prior to implementation.

	\begin{figure}[ht!]
		\centering
		\includegraphics[width=110mm]{images/glaswareFlowDesignerScreenshot}
		\caption{The Google Glass Flow Designer.}
		\label{GlassDesignStyle}
	\end{figure}

\subsection{Summary}
\label{subsec:summary}
\input{chap3/summary.tex}

o   Design - Present your project design in general
o   Information - Give details here (possibly several sub-sections)


\cleardoublepage

\section{Implementation}
\label{sec:implementation}
As the application launches, the first screen the user sees, in both versions, is the camera screen. The user must, in order to proceed further within the application, scan a QR code. Scanning a QR code is done by positioning the camera on the device (either Google Glass or smartphone) such that the QR code can be seen on screen. The user does not need to press any shutter button as the application automatically recognises the QR code pattern if seen on screen.%, as seen in Figure~\ref{}. The reasoning behind 

	\begin{figure}[ht!]
		\centering
		\includegraphics[width=110mm]{images/demo/qrCode}
		\caption{todo bild behöver uppdateras}
		\label{glassDemoQR}
	\end{figure}

The reason for not providing a menu on the start screen was because the application should be simple, easy to use and focus on what is important. Since the the focus of the application is to scan the QR code in order to receive the necessary instructions that is also the main focus of the first screen of the application.

When the QR code has been scanned the application decodes the QR code. The decoding process is done in the same way as described in Section~\ref{subsec:qrcode}. However, the decoding process is handled by the Zebra Crossing (ZXing) library~\cite{zxing}. ZXing is an open source barcode image processing library.

The smartphone application was based directly upon the ZXing library, where as the Google Glass application was based upon a port of the library to Google Glass, called ``BardcodeEye''~\cite{barcodeEye}. The main difference between ZXing and BarcodeEye is the fact that BarcodeEye is a full example application ready to be run, in contrast to the ZXing library which is only a library and as such needs to be attached to a runnable application.

The BarcodeEye application for Google Glass is however a bare bone application, used as an example and introduction as to how ZXing may be implemented in an application for Google Glass. BarcodeEye displayed the decoded information from the QR code and also gave the user the option to search the internet using the information previously decoded from the QR code.

As the QR code was intended to encode only a product ID, and the use the ID to download the instructions, rather than having all of the instructions encoded directly in the QR code the application had to be modified. However, prior to changing where the instructions were coming from the graphical layout of the application was changed. The change of layout was mostly done due to the fact that the application only displayed plain text, not taking in to account for instance a mix of image and text.

However, BarcodeEye also used the now deprecated class ``Card'', as seen in Listing~\ref{listingDeprecated}. Instead the application now uses the ``CardBuilder'' class, as seen in Listing~\ref{listingRecommended}, as recommended by Google~\cite{googleCard}. The CardBuilder class allows users to input a desired layout style as an argument to the constructor of the CardBuilder class.

\begin{lstlisting}[language=Java, caption={Instancing of the deprecated class Card}, label=listingDeprecated]
Card card = new Card(context);
\end{lstlisting}

\begin{lstlisting}[language=Java, caption={Instancing of the recommended class CardBuilder}, label=listingRecommended]
CardBuilder cardBuilder = new CardBuilder(context, CardBuilder.Layout.TITLE);
\end{lstlisting}

Since the smartphone application also used the ZXing library, but without any pre-existing application no changes similar to those done to the Google Glass application had to be done for the smartphone application.



%discuss differences (classes exclusive to the smartphone application and GG application respectivly)

%discuss downloading of product information


The download process also includes creating and initialise an instance of the Products class. The instance contains the name of the product, potentially an image of the product as the product will look when the user is done assembling all the components (the existence of an image is dependent of whether there was an image of the product stored in the database).

The Products class instance will also contain a list of components as well as a list of instructions. Both components and instructions are classes themselves. Similar to the Products class instances of both the Components class and the Instructions class will contain a string and potentially an image. In the case of components the string will contain the name of the component, in contrast to instances of the Instructions class where the string instead will contain the instruction itself.



%discuss sorting into classes

%discuss different layouts

	\begin{figure}[H]%ht!]
		\centering
		\includegraphics[width=110mm]{images/demo/titleCard}
		\caption{The title card of the demo application.}
		\label{glassDemotitleCard}
	\end{figure}
	
	\begin{figure}[H]%ht!]
		\centering
		\includegraphics[width=110mm]{images/demo/componentText}
		\caption{A component slide from the demo application.}
		\label{glassDemoQR}
	\end{figure}
	
	\begin{figure}[H]%ht!]
		\centering
		\includegraphics[width=110mm]{images/demo/instructionImage}
		\caption{An instruction slide from the demo application.}
		\label{glassDemoQR}
	\end{figure}
	
	\begin{figure}[H]%ht!]
		\centering
		\includegraphics[width=110mm]{images/demo/voiceCommand1}
		\caption{The voice command menu in the demo application.}
		\label{glassDemoQR}
	\end{figure}



\subsection{Android Studio}
Both the smartphone application as well as the Google Glass application were developed in Android Studio~\cite{androidStudio}. Android Studio is a development environment developed by Google. Both applications were initially being developed in Eclipse~\cite{eclipse}, however development soon shifted to Android Studio as Android Studio is now the official integrated development environment (IDE) for Android~\cite{androidIDE}. The shift was done without complications as Android Studio contains an import feature enabling developers to import projects previously not developed in Android Studio~\cite{androidIDE}.%[TODO why Android Studio]

%\subsection{ZXing}
%The application was built upon the open-source barcode image processing library, Zebra Crossing (ZXing). 
%[TODO Vilka förändringar har gjorts]
% https://github.com/zxing/zxing/

%\subsubsection{BarcodeEye}
%The Google Glass application was built upon the Google Glass port of the ZXing library, known as BarcodeEye~\cite{barcodeEye}.% [TODO vilka förändringar har gjorts]%While a slideview was implemented in BarcodeEye already, the information displayed was static [todo, var den statisk]. The slideview consited of only two slides [todo code example of how it was static]. Mixing images with text was not possible either. Information also had to be encoded directly into the QR code and could not be downloaded by an encoded link.
% https://github.com/BarcodeEye/BarcodeEye

%\subsection{View Slider}


%\subsection{AsyncTask}

%Used for image, as well as product

%\subsection{Text Split}


\subsection{Card Layout}
Google provides developers with a set of predefined layouts for different types of cards, which were used in the Google Glass application and used as basis for the design of the different layouts for the slides in the smartphone application. The following predefined layouts were used in the implementation: ``Title'', ``Columns'' and ``Text''. The Title layout was used for the first card of the slide view, which shows the product name as well as an image of the product as it is supposed to look when finished. TODO Image card? vilken layout? Använde den också title- layouten?

	\begin{figure}[ht!]
		\centering
    		\subfloat[The title card layout.]{{\includegraphics[width=70mm]{images/demo/titleCard}}}
   		 \qquad
		\subfloat[The column card layout.]{{\includegraphics[width=70mm]{images/demo/columnImage}}}
   		 \qquad
    		\subfloat[The text card layout.]{{\includegraphics[width=70mm]{images/demo/instructionText}}}
    		\qquad
        		\subfloat[The image card layout.]{{\includegraphics[width=70mm]{images/demo/instructionImage}}}
   		 \qquad
		\caption{The different layouts used within the Googla Glass application.}
		\label{fig:cardLayout}
	\end{figure}

The Columns card layout, seen in Figure~\ref{fig:cardLayout}~(b) was used for when an instruction or component was to be presented with both text and an image. Since the Columns layout split the card, with an image to the left and text to the right, the Columns layout was the most reasonable choice when presenting both text and an image. An alternative would have been to display the text on top of the image, the image could potentially have been hidden behind a larger amount of text. 

Such a layout design was instead used for the title card as the amount of text being displayed is only the name of the product, and the image is only to give an idea of what the finished product will look like. The layout design where the text overlapped the image was called Title and can be seen in Figure~\ref{fig:cardLayout}~(a).

If the information being presented, either a component or an instruction, instead were to be presented only as text the Text layout, seen in Figure~\ref{fig:cardLayout}~(c) was used. The Text layout displayed dynamically sized text. In other words, if there was a lot of text being displayed the text would be resized to fit the screen.

TODO IMAGE layout

Using the predefined layouts in the implementation was easily achieved as the process consisted mostly of plug-and-play. The \texttt{CardBuilder} class constructor took the layout as an argument, as seen in Listing~\ref{cardBuilderPlugPlay}. When and instance of the \texttt{CardBuilder} class was created what remained was to simply input the necessary information, such as the instruction text. Setting an image was done slightly differently than written information as images were loaded in using a separate thread. As soon as the \texttt{CardBuilder} method \texttt{getView} was called the card was built with the information that had been inputed.

\begin{lstlisting}[language=Java, caption={Initialisation of the CardBuilder class}, label=cardBuilderPlugPlay]
CardBuilder cardBuilder = new CardBuilder(context, CardBuilder.Layout.COLUMNS)
	.setText(getText())
	.setFootNote(mFootNote)
	.setTimestamp(mTimeStamp);

cardBuilder = (new LoadImage(isTitleCard(), getByteArray()).doInBackground(cardBuilder));

return cardBuilder.getView();
\end{lstlisting}

%which standard layout were used, one non-standard

\subsection{Voice Commands}
The Google Glass application gives users the option to use voice commands in order to navigate the slides. The user opens the voice command menu by saying ``ok glass'' at any point in the application when ``ok glass'' is written at the bottom of the screen. The voice command feature is available at all times except when the camera is active. In other words the voice commands are unavailable when the application is waiting to scan a QR code.

The voice command menu contains the following options.

\begin{itemize}
	\item \textbf{Show next slide}
	
	The application scrolls to the next slide. If the current slide is the last slide, and in other words no other slides are following, the application does nothing.
	\item \textbf{Show previous slide}
	
	The application scrolls to the previous slide. If the current slide is the first slide, and in other words no other slides are sits before it, the application does nothing.
	\item \textbf{Show components}
	
	The application scrolls to the first slide showing information on a component. If the user is currently on the first slide showing information on a component the application does nothing. 
	\item \textbf{Show instructions}
	
	The application scrolls to the first slide showing an instruction. If the user is currently on the first slide showing an instruction the application does nothing.
	\item \textbf{Scan again}
	
	The application launches the camera and expects the user to scan another QR code.
\end{itemize}

Implementing voice command in the Google Glass application is done by todo~\ref{voiceCommandXML}

\begin{lstlisting}[language=XML, caption={The voice command menu XML file}, label=voiceCommandXML]
<menu xmlns:android="http://schemas.android.com/apk/res/android">
	<item
		android:id="@+id/next_menu_item"
		android:title="Show next slide" >
	</item>
	<item
		android:id="@+id/previous_menu_item"
		android:title="Show previous slide" >
	</item>
	<item
		android:id="@+id/components_menu_item"
		android:title="Show components" >
	</item>
	<item
		android:id="@+id/instructions_menu_item"
		android:title="Show instructions" >
	</item>
	<item
		android:id="@+id/scan_menu_item"
		android:title="Scan again" >
	</item>
</menu>
\end{lstlisting}

Although none of the voice commands have been sent in for official approval by Google all of the voice commands follows the design guidelines provided by Google. 

	\begin{table}[ht!]
    		\caption{Voice Command Checklist~\cite{glassVoiceChecklist}.} \label{tab:voiceCommandCheckTableChecked}
		\centering \begin{tabularx}{\textwidth}{l|X|l} \hline
		 & \textbf{Guideline} & \textbf{Acheived} \\ \hline \hline
       
1	&	Is general enough to apply to multiple Glassware, but still has a clear purpose		&	Yes		\\ \hline
2	&	Is colloquial and can explain Glass features in a conversation					&	Yes		\\ \hline
3	&	Is comfortable to say in public											&	Yes		\\ \hline
4	&	Brings the user from intent to action as quickly as possible					&	Yes		\\ \hline
5	&	Avoids brand words													&	Yes		\\ \hline
6	&	Is long enough to ensure high recognition quality (at least three syllables)			&	Yes		\\ \hline
7	&	Fits on a single line													&	Yes		\\ \hline
8	&	Does not sound similar to existing commands								&	Yes		\\ \hline
9	&	Does not require immediate interactivity in Mirror API Glassware.				&	Yes		\\ \hline
10	&	Has an imperative verb with an object									&	Yes		\\ \hline
11	&	Uses articles when possible											&	No		\\ \hline
12	&	Uses definite articles only when the object is definite							&	No		\\ \hline
13	&	Uses ``this'' when there is only one relevant instance of the object				&	No		\\ \hline
14	&	Uses me and my when appropriate										&	No		\\ \hline
15	&	Refers to Glass as the subject carrying out the action						&	Yes		\\ \hline
		
		\end{tabularx}
	\end{table}

\subsection{Test Cases}
The following section describes how the tests were set up and carried out.

\subsubsection{Experimental Setup}
The tests were carried out using an optical bench to guarantee more scientific accuracy. The experimental setup contained an optical bench, with a screen holder at the zero point where the QR code was positioned. The device currently being tested, Google Glass or smartphone, was then positioned at the specified mark on the optical bench using a clamp and pointed towards the QR code. See Figure~\ref{experimentalSetup} for a better understanding of the experimental setup. 

	\begin{figure}[ht!]
		\centering
		\includegraphics[width=110mm]{images/demo/componentText}
		\caption{todo change image The experimental setup.}
		\label{experimentalSetup}
	\end{figure}

In order to measure the time needed for the results of each test a specific class was built, called \texttt{Timer} and seen in Listing~\ref{timerClass}. The \texttt{Timer} class was built using the singleton design pattern. A singleton class is a class that can only be instanced once during the entire execution of an application, however the instance lives throughout the entire execution and may be accessed from anywhere in the application.

Using this pattern meant that the timer could be started in one class, and stop in another without having to pass the instance around, which potentially could affect performance.

\begin{lstlisting}[language=Java, caption={The Timer class}, label=timerClass]

public class Timer {
	private static Timer ourInstance = new Timer();
	public static Timer getInstance() { return ourInstance; }
	private Timer() {  }
	
	private boolean timerRunning = false;
	private Long startTime;
	private Long stopTime;
	
	public void startTimer() { 
		if(timerRunning) { Log.d("TIMER", "Timer already running"); }
		else 	{ startIme = System.nanoTime(); }
	}
	
	public void stopTimer() {
		if(!timerRunning) { Log.d("TIMER", "No timer running"); }
		else { stopTime = System.nanoTime()); }
	}
	
	private long getElapsedTime(int timerID) { return stopTime - startTime; }
	
	public void logElapsedTime(String information) {
		Log.d("TIMER", information + ": " + String.valueOf(getElapsedTime() + " nano seconds");
	}
}

\end{lstlisting}

\subsubsection{Text Length}
\begin{lstlisting}[language=Java, caption={The randomizer class}, label=todo]
private double randfrom(double min, double max)
{
	Random rand = new Random();
	double range = (max - min);
	return min + range * rand.nextDouble();
}

private String getChar(int pos, double rand)
{
	if(rand <= doubleList.get(pos) || pos+1 <= alph.size())
		return alph.get(pos);
		
	return getChar(pos+1, rand);
}

public String randchar()
{
	double rand = randfrom(0, 1);
	return getChar(0, rand);
}
\end{lstlisting}

\subsubsection{Distance to the QR Code}

	\begin{table}[ht!]
    		\caption{Average time of registering a QR code with varying distance.} \label{tab:distanceAverage}
		\centering \begin{tabularx}{\textwidth}{l|X|X|X} \hline
		\textbf{Distance (dm)} & \textbf{Google Glass} & \textbf{Samsung Galaxy SII} & \textbf{Samsung Galaxy SIII} \\ \hline \hline
       
		1	&	&	&	\\ \hline
		2	&	&	&	\\ \hline
		3	&	&	&	\\ \hline
		
		\end{tabularx}
	\end{table}

\subsubsection{Complexity of the QR Code}

	\begin{table}[H]%ht!]
    		\caption{Average time of registering a QR code with varying density.} \label{tab:complexityAverage}
		\centering \begin{tabularx}{\textwidth}{l|X|X|X} \hline
		\textbf{Encoded Characters} & \textbf{Google Glass} & \textbf{Samsung Galaxy SII} & \textbf{Samsung Galaxy SIII} \\ \hline \hline
       
		1	&	&	&	\\ \hline
		50	&	&	&	\\ \hline
		100	&	&	&	\\ \hline
		
		\end{tabularx}
	\end{table}

\subsubsection{Display Time}

	\begin{table}[ht!]
    		\caption{Average display time for Google Glass with varying information size.} \label{tab:averageDisplaySpeedGoogleGlass}
		\centering \begin{tabularx}{\textwidth}{l|X|X|X} \hline
		\textbf{Information Size (Byte)} & \textbf{Google Glass (ms)}  & \textbf{Samsung Galaxy SII (ms)}  & \textbf{Samsung Galaxy SIII (ms)} \\ \hline \hline
       
		100 k	&	&	&	 \\ \hline
		1 M		&	&	&	 \\ \hline
		10 M		&	&	&	 \\ \hline

		\end{tabularx}
	\end{table}

\subsection{Summary}
\label{subsec:summary}
The application works as such that the first screen the user sees when launching the application, on both Google Glass and smartphones, is the camera screen. The user is then to hposition the device in such a was that the QR code may be scanned by the device's camera. The QR code contains a product ID for a specific product.

The user does not need to press any shutter button in order to scan the QR code. Instead the application will automatically recognise any QR code pattern that appears in camera view, as well as scan the QR code. The reason for not implementing a start menu or any similar start screen, to show the user before the camera screen, is to, according to Google's design guidelines, keep the focus on what the application is intended to do and to keep the application simple and easy to use.

Next the application will decode the QR code. The decoding process is done by the ZXing library, which is an open source  barcode image processing library. The QR code contains a product ID which is then used in the downloading process. The downloading process entails connecting to a database containing information on different products, and, by using the decoded product ID, retrieving the information on the specific product. 

The downloaded information contains the product name, as well as a list of components and the instructions necessary to construct the product. All the information is then sorted in to respective classes and the information may be displayed to the user. When the product information is being downloaded a loading animation is displayed on screen. On Google Glass the loading information is a loading bar at the bottom of the screen, and in the smartphone application the loading animation is a spinning wheel.

When the download process has finished the information is displayed to the user in the form of a slide show. The first slide that is displayed to the user is the title slide. The title slide contains the name of the product as well as an image  (if an image existed in the database). Following the title slide are the component slides. Each component has their own slide due to the fact that a component may be described in both text and an image. 

After all the component slides comes the instruction slides. Similar to the components an instruction my be presented by text only or by both text and an image. In contrast to the components, however, instructions may also be presented with an image and no text. 

As Google provides developers of Google Glass applications with predefined layouts, these layouts were also used for the Google Glass application. The layouts used were ``Title'', ``Columns'', ``Text'' and ``Caption''. The predefined layouts were also used as basis for the layouts used in the smartphone application.

The Title layout is used for the title card. The Columns layout is used for the slides with both text and an image. The Text layout is used for the slides with only text. The Caption layout is used for the slides with only an image. All layouts, except for the Title layout, also contains text at the bottom of the screen called ``footer'' and ``timestamp''. The footer contains information on wether the current slide is a component slide or an instruction slide. The timestamp contains information on which slide is currently being viewed.

While browsing through the slides in the Google Glass application, the user may also navigate using voice commands. The voice commands available in the Google Glass application are ``Show next slide'', ``Show previous slide'', ``Show components'', ``Show instructions'' and ``Scan again''.

Most of the voice commands follows 11 out of the 15 voice command guidelines provided by Google. For instance does ``Show components'' not follow the guidelines which states that ``Is general enough to apply to multiple Glassware, but still has a clear purpose''. ``Show components'' is a specific voice command and could potentially apply to multiple Google Glass applications, but not most.

While viewing the slides ``ok glass'' is shown at the bottom of the screen. ``ok glass'' indicates that voice commands are available and saying ``ok glass'' at that point brings up the voice command menu, showing all available voice commands. However, ``ok glass'' is also shown in combination with a dark, transparent overlay, which does ensure ``ok glass'' is always visible no matter what image is shown on screen, but the dark overlay does also mean that any image shown is darken by the overlay.

In terms of the testing done on the application the experimental setup consisted of an optical bench on which the QR code were positioned at the zero mark, and then each specific device were position as such that the camera of each device were positioned according to the specifications of each test. Each test result was then obtained through a laptop with which each device was connected via a USB cable. The test results was printed out from a timer class, called \texttt{Timer}, which was implemented using the singleton design pattern, meaning that the class only had one global instance which could be accessed from anywhere within the appllication.

The test which did not require the experimental setup was the text length test. Instead the text length test was performed using a class which randomised english characters, which were then added together to a large string.

Instead three other tests were done while using the experimental setup. In these tests the distance to the QR code, the complexity of the QR code and the size of the downloaded information respectively. Each of these tests were done 30 times, for each device and each different specification, to ensure statistical significance.

%o   Implementation - Present your project implemetion in general
%o   Information - Give details here (possibly several sub-sections)
%o   Summary - for this chapter

% Mobile phone application
% Uses Zxing - library for QR code scanning [Link to gihub repository missing!!!]
% can display both text and images 
% 
% Google Glass Application
% Uses BarcodeEye - library for QR Code Scanning (port of Zxing for Google Glass) [Link to github repository missing!!!]
% Can display text, images still todo
% Much easier to add slides on Google Glass compared to Mobile Phone Application. Probably because Cards is standard interface for Google Glass (might also be because simple API, check CardPresenter!!!)
% Very difficult to add images since cards can't be changed after .getView() has been called
% Need to call .notifyViewChanged() but does not work anyway (yet) seems to only start the activity over which calls the execute method again if no check has been put in place
\cleardoublepage

\section{Results} % Result / Evaluation
\label{sec:resultevaluation}
\subsection{Distance to the QR Code}

	\begin{table}[ht!]
    		\caption{Average time of registering a QR code with varying distance.} \label{tab:distanceAverage}
		\centering \begin{tabularx}{\textwidth}{l|X|X|X} \hline
		\textbf{Distance (dm)} & \textbf{Google Glass} & \textbf{Samsung Galaxy SII} & \textbf{Samsung Galaxy SIII} \\ \hline \hline
       
		1	&	&	&	\\ \hline
		2	&	&	&	\\ \hline
		3	&	&	&	\\ \hline
		
		\end{tabularx}
	\end{table}

\subsection{Complexity of the QR Code}

	\begin{table}[H]%ht!]
    		\caption{Average time of registering a QR code with varying density.} \label{tab:complexityAverage}
		\centering \begin{tabularx}{\textwidth}{l|X|X|X} \hline
		\textbf{Encoded Characters} & \textbf{Google Glass} & \textbf{Samsung Galaxy SII} & \textbf{Samsung Galaxy SIII} \\ \hline \hline
       
		1	&	&	&	\\ \hline
		50	&	&	&	\\ \hline
		100	&	&	&	\\ \hline
		
		\end{tabularx}
	\end{table}

\subsection{Display Time}

	\begin{table}[ht!]
    		\caption{Average display time for Google Glass with varying information size.} \label{tab:averageDisplaySpeedGoogleGlass}
		\centering \begin{tabularx}{\textwidth}{l|X|X|X} \hline
		\textbf{Information Size (Byte)} & \textbf{Google Glass (ms)}  & \textbf{Samsung Galaxy SII (ms)}  & \textbf{Samsung Galaxy SIII (ms)} \\ \hline \hline
       
		100 k	&	&	&	 \\ \hline
		1 M		&	&	&	 \\ \hline
		10 M		&	&	&	 \\ \hline

		\end{tabularx}
	\end{table}


%%o   Introduction - Summarise your main results
%%o   Give details of the results
%%o   Best presentation? (text, tables, diagrams?)
%o   Implementation Evaluation - your results against your expectations%
%%o   Summary - for this chapter

%\subsection{The Application}

%\subsubsection{Google Glass}

%\subsubsection{Smartphone}

%\subsection{Test Cases}

%\subsubsection{Text Length}

%\subsubsection{Image Size}

%\subsubsection{Comparing Text and Images}

%\subsubsection{Download Speed}

%\subsubsection{Interaction Delay}

%\subsubsection{Background Noise}

%\subsubsection{Size of QR Code}

%\subsubsection{Complexity of QR Code}

%\subsubsection{``Tap Counter''}

%\subsubsection{User Experience}

%\subsubsection{Multitasking}

%\subsubsection{Battery}

%\subsubsection{Connected to Mobile Device}

%\subsubsection{Overall Conclusions}
\cleardoublepage

\section{Conclusions}
\label{sec:conclusion}
%o   Conclusion
%o   Project Evaluation
%o   Problems - How would you do this the next time?
%o   Future work
As seen in the test results the Google Glass application was almost always the slowest in all of the tests. However, as the test results only shows the time from when the QR code has been scanned until when the information is being displayed on screen does not differ by more then about half a second between Google Glass and Samsung Galaxy SII, and the difference in time might not be of significance. 

After the user has scanned a QR code and the information is displayed, that is when the actual usage of the application starts. A user might be able to look past the fact that the Google Glass application is slower at scanning and presenting information as the advantages of Google Glass lies not in the speed of the device but rather in the user interface.

The fact that users do not have be use their hands is the major advantage of using Google Glass over a smartphone. Although voice commands are possible to implement in the smartphone application as well, the user must still hold the smartphone or place the smartphone on a surface in order to see the information being displayed. Google Glass puts the display in front of the user and does not require the user to hold the display in any way.

However, the fact that Google Glass was not able to scan more complex QR codes is definitely a negative aspect. Although the product ID:s used in this test were not long enough as to where complexity would become an issue, the complexity of the QR code will be an issue when the database contains many, many more products and the product ID:s are such that the complexity of the QR code is too high for Google Glass to handle.

As such, one clear improvement Google must do on Google Glass is upgrade the camera. Not only to be able to compete with smartphone cameras, which, as seen, were much better already at the time of Google Glass's release, but simply to be able to identify QR codes with high complexity and at longer distances.

Another important aspect of Google Glass is the amount of information the device can display on one screen. As shown in the text length test, a card on Google Glass which only displays text can hold somewhere between 200 and 220 characters, in comparison to smartphones with similar layout design, which can hold somewhere between 550 and 750 characters.

According to Google's guidelines for writing text for Google Glass, the text should be brief and simple, with the most important part of the text first and without repetition. Clearly, with the limit in text length, writing short and precise text is important on Google Glass. Getting information across to the user without having to divide text up on several cards is important, for instance when writing instructions. Depending on the user to scroll back and forth between cards in order to read one instruction would not be good design, and as such instructions should be simple and brief, similar to the way the are in the application described in this dissertation.

\subsection{Personal User Experience}
\label{subsec:personalexperience}
Having used about every weekday Google Glass for nearly four months there are a few comments that can ben made, and a few conclusions that can be drawn, simply from personal user experience. Please note that these comments are not based on any scientific studies, but are rather the opinions of the author of this dissertation.

% easy to use

% getting used to wearing Glass, takes a few days up to a week of contant usage. display could be irritating, but the point is that it is only lit up while using and the timed out

% the heat

% restrictions

% casual use, perhaps not industry at this time

\subsection{Project Evaluation}
Overall the project has gone well. 

getting in to the barcodeeye code took time shoudl have done more refactoring on barcodeeye along the way, might have made the code smaller than it is at this point



todo Update image on slide - got help through code evaluation - listener - 

\subsection{Back End Conclusions}
The dissertation revolving the back end part of the project, written by Richard Hoorn, concludes that while Google Glass is an impressive peice of technology, Google Glass was still outperformed by smartphones since Google Glass is not able to handle larger data sizes. Data sizes above one mega byte was very difficult or even impossible for Google Glass to handle.

A possible solution would be to not let Google Glass do all the work, but rather connect Google Glass to a smartphone, from which information is then streamed. As such, Google Glass would simply act as a display, displaying the information sent from the smartphone, and would not have to handle all of the downloaded information at the same time.

Hoorn also discuss the possibilities of headache which Google Glass could cause and the fact that looking up at the screen could cause eye strain. These concerns have been adressed by Google, who states that Google Glass is not meant for usage of longer periods of time. Using Google Glass longer periods of time will also cause them to overheat, causing both performance issue as well as a very hot exterior. As such, Google Glass is perhaps not suited for industry use, where Google Glass would potentially be used over longer periods of time, but is rather suited and designed for the more casual user.

\subsection{Future Work}
\label{subsec:futurework}
\subsubsection{Official approval of Voice Commands}
The voice commands should be officially approved by Google.

\subsubsection{Customised Voice Command}
Construct own or use \url{https://github.com/RIVeR-Lab/google_glass_driver/blob/master/android/RobotManager/src/com/riverlab/robotmanager/voice_recognition/VoiceRecognitionThread.java}

\subsubsection{TextResultProcessor}
In the Google Glass application the class TextResultProcessor is not needed any more and should as such be removed. At this point the class in only used as middleware between an instance of the Products class and a list of CardPresenter. Instead the CardPresenter class should only be instanced once for each product, and keep the instance of the Products class.

The reason the TextResultProcessor class exists in the first place is due to how the Google Glass application was built originally, were all information presented was encoded directly in the QR code. At that point the TextResultProcessor was used when the encoded information was a text string. At this point the only information encoded in the QR codes are product ID:s.

The smartphone application already functions in this way, where the information stored in the instance of the Products class is used directly when a slide is created, instead of first being sorted through a middleware class.

[TODO possibly uml diagram of how the application works now and how it should work]

\subsubsection{A General Fragment}
The smartphone application should only have one general fragment instead of a bunch of different ones for different purposes. This should be done in order to be even more similar to the Google Glass application which uses the CardBuilder class, that is a general case that takes the layout as input.

\subsection{Concluding Remarks}
Google Glass is an interesting attempt at pushing technology further. The potential of Google Glass is exciting. However, the fact that Google Glass intended for use in short burts, simple to take a picture or to send a message, makes it hard to argue in favour of using Google Glass in any situation besides more casual ones. As long as any work done on the device is not too complex or demanding in terms of hardware, Google Glass can be very helpful.

However, Google Glass should not be seen as a competitor to smartphones, but rather a complementary device which may be used together with a smartphone. In terms of displaying instructions to a user, who is assembling components, Google Glass could be useful as the user does not have to shift focus and look away. As long as the instructions are short, simple and does not make up too much data size, Google Glass would be the preferred device over smartphones.

However, smartphones bigger screens and stronger hardware make smartphones the preferred choice when the instructions are complex and contains, for instance, a video, as the video's data size potentially would be too large for Google Glass to handle without performance issues.

A solution would be to compress the video, both in length and data size, or to have the video stream from a smartphone on to Google Glass, using Google Glass and a smartphone as complements to each other. As Google continues to develop Google Glass behind closed doors, it will be interesting to see how Google Glass has evolved once it is shown again. Until such time, one can only hope Google keep improving one Google Glass, and let go of some of their restrictions, as the potential for an interesting future has been exposed. 
\cleardoublepage

%\nocite{*}

% Start of reference section. 
\begin{singlespace}
\bibliography{myBibliography}
\bibliographystyle{plain}
\end{singlespace}
\cleardoublepage

% Appendix if present goes here (optional part)
\appendix
\section{Abbreviations}
\label{sec:abbreviations}
\begin{description}
	\item [GUI] Graphical User Interface
	\item [HMD] Head-Mounted Display
	\item [HUD] Heads-Up Display
	\item [OHMD] Optical Head-Mounted Display
\end{description}
	
\cleardoublepage

\section{Results}
\label{app:results}
	\begin{table}[ht!]
    		\caption{Results of scanning QR code at different distances with Google Glass} \label{tab:distamceSmartphoneFull}
		\centering \begin{tabularx}{\textwidth}{l|X|X|X} \hline
		\textbf{Test Number} & \textbf{1 decimeter} & \textbf{2 decimeters} & \textbf{3 decimeters} \\ \hline \hline
       
		1&	1884216308	&	1798065186	&	2296325683	\\ \hline
		2&	1682800293	&	1705902100	&	2415893555	\\ \hline
		3&	2043151856	&	1937561035	&	2408782959	\\ \hline
		4&	2487091065	&	1779327392	&	2346679688	\\ \hline
		5&	1316070557	&	1948822022	&	2336975098	\\ \hline
		6&	1631652832	&	1777801514	&	2341278076	\\ \hline
		7&	1576843262	&	1941070556	&	2184875488	\\ \hline
		8&	1772125244	&	2041961670	&	2347503662	\\ \hline
		9&	2140167236	&	2060516358	&	2296203613	\\ \hline
		10&	1911987304	&	1757171630	&	2170745849	\\ \hline
		11&	1884277345	&	1767211914	&	2357238769	\\ \hline
		12&	1929656983	&	2313415528	&	2337341308	\\ \hline
		13&	1709838868	&	1731719971	&	2460479736	\\ \hline
		14&	1819946288	&	1789154053	&	2208862304	\\ \hline
		15&	1881225586	&	1959869384	&	2198150634	\\ \hline
		16&	1790405272	&	1852996826	&	2317962646	\\ \hline
		17&	1574829101	&	1649383545	&	1774505615	\\ \hline
		18&	1825531006	&	1702728271	&	2231719971	\\ \hline
		19&	1731201172	&	1674377441	&	2281311035	\\ \hline
		20&	2475097657	&	1735839844	&	1698455811	\\ \hline
		21&	2370330810	&	1776123047	&	2375762940	\\ \hline
		22&	1872375489	&	1714141846	&	2186004639	\\ \hline
		23&	2327819824	&	2099090575	&	2258178712	\\ \hline
		24&	1753479003	&	1826843263	&	2193664551	\\ \hline
		25&	1721496582	&	1849182128	&	2280761718	\\ \hline
		26&	2481842039	&	1802490234	&	1814788818	\\ \hline
		27&	1602935791	&	1912353515	&	1650085449	\\ \hline
		28&	2181121826	&	1636383056	&	2349945068	\\ \hline
		29&	1797210694	&	1716491699	&	2250000000	\\ \hline
		30&	2422576904	&	1698699951	&	2444274902	\\ \hline

		\end{tabularx}
	\end{table}

	\begin{table}[ht!]
    		\caption{Results of scanning QR code at different distances with Samsung Galaxy SII} \label{tab:distamceGoogleGlassFull}
		\centering \begin{tabularx}{\textwidth}{l|X|X|X} \hline
		Test Number & \textbf{1 decimeter} & \textbf{2 decimeters} & \textbf{3 decimeters} \\ \hline \hline
		
		1&	2139868085	&	1486211750	&	1643240375	\\ \hline
		2&	1467511210	&	1433458333	&	1645772458	\\ \hline
		3&	1531975085	&	1544185626	&	2269734332	\\ \hline
		4&	1999989252	&	1474127458	&	1652179501	\\ \hline
		5&	2117759919	&	1437559418	&	1784042751	\\ \hline
		6&	1590932000	&	1346899542	&	1959105167	\\ \hline
		7&	2309159125	&	1542022917	&	2183700002	\\ \hline
		8&	1722160542	&	1421151250	&	1430372917	\\ \hline
		9&	1442118042	&	1537669377	&	2517021085	\\ \hline
		10&	2225043835	&	1922686667	&	1584863210	\\ \hline
		11&	2559335792	&	1634340084	&	1544045376	\\ \hline
		12&	2087568125	&	1493966751	&	2455554542	\\ \hline
		13&	2399355752	&	1654488126	&	2096796210	\\ \hline
		14&	1835165960	&	1681548458	&	1444782750	\\ \hline
		15&	2013470417	&	1577887875	&	2794100627	\\ \hline
		16&	1947420919	&	1971720543	&	2164583833	\\ \hline
		17&	1914936335	&	1497237917	&	1976264377	\\ \hline
		18&	1964812292	&	1473154916	&	2001674627	\\ \hline
		19&	1836791835	&	1332783794	&	2090852043	\\ \hline
		20&	1291630877	&	1760558543	&	1331759376	\\ \hline
		21&	1715746333	&	1808396667	&	1840536334	\\ \hline
		22&	2281021168	&	1940200418	&	2039864458	\\ \hline
		23&	2106378627	&	1551906084	&	2214860250	\\ \hline
		24&	2048848293	&	2127196127	&	1541294919	\\ \hline
		25&	1953401960	&	1389226667	&	1873071125	\\ \hline
		26&	1901472292	&	1534745500	&	2040191418	\\ \hline
		27&	2009275501	&	1439262916	&	1916855625	\\ \hline
		28&	1843962793	&	2145346709	&	1627921959	\\ \hline
		29&	2639261585	&	2315900375	&	1939550709	\\ \hline
		30&	2082421543	&	2017870959	&	2048423918	\\ \hline
		
		\end{tabularx}
	\end{table}
	
	\begin{table}[ht!]
    		\caption{Results of scanning QR code at different distances with Samsung Galaxy SIII} \label{tab:distamceGoogleGlassFull}
		\centering \begin{tabularx}{\textwidth}{l|X|X|X} \hline
		Test Number & \textbf{1 decimeter} & \textbf{2 decimeters} & \textbf{3 decimeters} \\ \hline \hline
		
		1&	1749676834	&	1779497917	&	1513588627	\\ \hline
		2&	1980956916	&	1389859375	&	1552651834	\\ \hline
		3&	1784749169	&	1264649875	&	1513143459	\\ \hline
		4&	1842951960	&	1513206585	&	1352977542	\\ \hline
		5&	1754621419	&	1473651001	&	1307146084	\\ \hline
		6&	1644033335	&	1392401627	&	1453716167	\\ \hline
		7&	1730130585	&	1282980458	&	1582548250	\\ \hline
		8&	1672027044	&	1688112459	&	1311798583	\\ \hline
		9&	1737328751	&	1432261543	&	1462818377	\\ \hline
		10&	2373175125	&	1317017667	&	1316425376	\\ \hline
		11&	1848027542	&	1463129292	&	2047718001	\\ \hline
		12&	1576518375	&	1676956292	&	1675308210	\\ \hline
		13&	1642278708	&	1709122334	&	1916630376	\\ \hline
		14&	1750943335	&	1423191250	&	2168636127	\\ \hline
		15&	1792001168	&	1402313542	&	1774322918	\\ \hline
		16&	1733112083	&	1544061083	&	1798746335	\\ \hline
		17&	2075140127	&	1554050960	&	2053470334	\\ \hline
		18&	1940637085	&	1406057500	&	1583177875	\\ \hline
		19&	1879888500	&	1666031252	&	1327241125	\\ \hline
		20&	1690253250	&	1350420626	&	1350043708	\\ \hline
		21&	1694755291	&	1506406918	&	1429665709	\\ \hline
		22&	1966512835	&	1445166958	&	1401702668	\\ \hline
		23&	1741643376	&	1664807709	&	1242497667	\\ \hline
		24&	2036178543	&	1692449501	&	1511735834	\\ \hline
		25&	1694977000	&	1474073667	&	1383570126	\\ \hline
		26&	1851916252	&	1655224458	&	1549562168	\\ \hline
		27&	1653187293	&	1203773709	&	1560395085	\\ \hline
		28&	1966861500	&	1297300542	&	1317753292	\\ \hline
		29&	1885895208	&	1287375710	&	1785062417	\\ \hline
		30&	2499305419	&	1667945793	&	1821073459	\\ \hline

		\end{tabularx}
	\end{table}
\input{tables/ComplexityTable.tex}
	\begin{table}[ht!]
    		\caption{Google Glass} \label{tab:distamceSmartphoneFull}
		\centering \begin{tabularx}{\textwidth}{l|X|X|X} \hline
		Test Number & \textbf{1 kB} & \textbf{100 kB} & \textbf{1 MB} \\ \hline \hline

		1&	22128417	&	23806542	&	21965000	\\ \hline
		2&	15872960	&	25955458	&	23531209	\\ \hline
		3&	22888792	&	35965208	&	19634833	\\ \hline
		4&	28067583	&	38730916	&	24453667	\\ \hline
		5&	19896250	&	38804333	&	32512083	\\ \hline
		6&	23419042	&	29999625	&	26414750	\\ \hline
		7&	21411542	&	38175167	&	24274417	\\ \hline
		8&	26785041	&	31706084	&	27390000	\\ \hline
		9&	22456042	&	36254834	&	24890666	\\ \hline
		10&	21889541	&	35148375	&	28355083	\\ \hline
		11&	27834042	&	42737374	&	23042333	\\ \hline
		12&	22855541	&	42614042	&	25686667	\\ \hline
		13&	26527083	&	28933625	&	27155416	\\ \hline
		14&	23235959	&	22850875	&	26465708	\\ \hline
		15&	21809916	&	38291375	&	25557583	\\ \hline
		16&	20713958	&	24469625	&	25121875	\\ \hline
		17&	21938833	&	32561208	&	22315125	\\ \hline
		18&	22112709	&	48090375	&	20184625	\\ \hline
		19&	26409875	&	30071209	&	20032209	\\ \hline
		20&	27796334	&	27028876	&	19081542	\\ \hline
		21&	22990750	&	31401792	&	27814292	\\ \hline
		22&	22468667	&	31754459	&	17277167	\\ \hline
		23&	15608750	&	31208458	&	24796125	\\ \hline
		24&	22743958	&	30416209	&	25758250	\\ \hline
		25&	25656166	&	22642625	&	18390541	\\ \hline
		26&	21030250	&	22831791	&	18389583	\\ \hline
		27&	25625666	&	25375541	&	20224416	\\ \hline
		28&	21925167	&	48635958	&	18421000	\\ \hline
		29&	22280792	&	23115125	&	20458334	\\ \hline
		30&	22374167	&	42448750	&	18195875	\\ \hline
		
		\end{tabularx}
	\end{table}

	\begin{table}[ht!]
    		\caption{Samsung Galaxy SII} \label{tab:distamceGoogleGlassFull}
		\centering \begin{tabularx}{\textwidth}{l|X|X|X} \hline
		Test Number & \textbf{1 kB} & \textbf{100 kB} & \textbf{1 MB} \\ \hline \hline
       
		1&	6797250	&	14496709	&	64364083		\\ \hline
		2&	6326709	&	43183958	&	100282458	\\ \hline
		3&	6746292	&	24071374	&	18386459		\\ \hline
		4&	7058875	&	31881374	&	80973374		\\ \hline
		5&	11829542	&	28462958	&	84470625		\\ \hline
		6&	10667375	&	13489125	&	109992750	\\ \hline
		7&	8890167	&	12720376	&	18701084		\\ \hline
		8&	21200959	&	28104084	&	11930708		\\ \hline
		9&	5742833	&	14041624	&	37545749		\\ \hline
		10&	7117500	&	34152708	&	37839583		\\ \hline
		11&	6092042	&	13555542	&	32261416		\\ \hline
		12&	7110333	&	14802667	&	26862500		\\ \hline
		13&	11012625	&	14330666	&	32655250		\\ \hline
		14&	9552500	&	29462791	&	38692001		\\ \hline
		15&	8780000	&	40578625	&	73218541		\\ \hline
		16&	10533500	&	11147917	&	11013791		\\ \hline
		17&	5902250	&	13767874	&	10906625		\\ \hline
		18&	10135791	&	12711084	&	137211250	\\ \hline
		19&	8074709	&	13150458	&	29192460		\\ \hline
		20&	8935250	&	13934084	&	102380958	\\ \hline
		21&	6071583	&	14262083	&	29384416		\\ \hline
		22&	6577125	&	15152625	&	31167750		\\ \hline
		23&	6078875	&	23965041	&	32326793		\\ \hline
		24&	5932333	&	29859749	&	38545333		\\ \hline
		25&	6505000	&	28264250	&	35596375		\\ \hline
		26&	8270792	&	19016750	&	79581583		\\ \hline
		27&	7437583	&	13678375	&	88178667		\\ \hline
		28&	10711625	&	13444250	&	69627500		\\ \hline
		29&	5428708	&	19024708	&	26571250		\\ \hline
		30&	9704542	&	14632958	&	38301208		\\ \hline

		\end{tabularx}
	\end{table}
	
	\begin{table}[ht!]
    		\caption{Samsung Galaxy SIII} \label{tab:distamceGoogleGlassFull}
		\centering \begin{tabularx}{\textwidth}{l|X|X|X} \hline
		Test Number & \textbf{1 kB} & \textbf{100 kB} & \textbf{1 MB} \\ \hline \hline
		
		1&	253570557	&	442565918	&	605865479	\\ \hline
		2&	302947998	&	368041992	&	542236329	\\ \hline
		3&	309509278	&	388549804	&	591369630	\\ \hline
		4&	330413818	&	382720947	&	637756348	\\ \hline
		5&	325225830	&	408355713	&	436096192	\\ \hline
		6&	308532715	&	345550537	&	547424317	\\ \hline
		7&	339141846	&	468597412	&	575408936	\\ \hline
		8&	163146973	&	405181885	&	383300781	\\ \hline
		9&	350677490	&	363525391	&	563415528	\\ \hline
		10&	317047119	&	258819580	&	511657714	\\ \hline
		11&	289825439	&	387298584	&	581054688	\\ \hline
		12&	334625243	&	368591309	&	683319091	\\ \hline
		13&	357788086	&	358184814	&	531799316	\\ \hline
		14&	286682129	&	256286621	&	553283693	\\ \hline
		15&	342498779	&	376190186	&	551635742	\\ \hline
		16&	309417724	&	403717041	&	580993652	\\ \hline
		17&	298553467	&	462371827	&	549407959	\\ \hline
		18&	283508300	&	394836426	&	582153320	\\ \hline
		19&	348236084	&	396270752	&	706787110	\\ \hline
		20&	319580079	&	516387940	&	476898193	\\ \hline
		21&	313873291	&	540405274	&	576416016	\\ \hline
		22&	320098878	&	521606446	&	563262940	\\ \hline
		23&	288299561	&	438934327	&	603668214	\\ \hline
		24&	308593750	&	727264404	&	553710937	\\ \hline
		25&	338897705	&	550872803	&	564880371	\\ \hline
		26&	321868897	&	534332277	&	582336426	\\ \hline
		27&	332733154	&	549041748	&	693267823	\\ \hline
		28&	297607421	&	557159424	&	748901368	\\ \hline
		29&	317657470	&	543640137	&	506378174	\\ \hline
		30&	313507080	&	557922364	&	880432129	\\ \hline
		
		\end{tabularx}
	\end{table}
\cleardoublepage

%\section{Voice Command Checklist}
%\label{app:voiceCommandChecklist}
%	\begin{table}[ht!]
    		\caption{The Google Glass voice command checklist~\cite{glassVoiceChecklist}.} \label{tab:voiceCommandCheckTableUnchecked}
		\centering \begin{tabularx}{\textwidth}{l|X|X|X} \hline
		\textbf{todo} & \textbf{Guideline} & \textbf{Good Example} & \textbf{Bad Example} \\ \hline \hline
       
1	&	Is general enough to apply to multiple Glassware, but still has a clear purpose	&	``ok glass, learn a song''	&	``ok glass, learn something'', ``ok glass, learn a song on guitar''	\\ \hline
2	&	Is colloquial and can explain Glass features in a conversation	&	``ok glass, take a picture'' (``You can use Glass to take a picture'')	&	``ok glass, take picture'' (``You can use Glass to take picture'')	\\ \hline
3	&	Is comfortable to say in public	&	``ok glass, find a doctor''	&	``ok glass, find a gynecologist''	\\ \hline
4	&	Brings the user from intent to action as quickly as possible	&	``ok glass, find a recipe for'' (this allows users to speak ``chicken kiev'' and immediately see the recipe)	&	``ok glass, show me a cookbook'' (this forces users to look through a list for what they want)	\\ \hline
5	&	Avoids brand words	&	``ok glass, make a video call''	&	``ok glass, start a hangout''	\\ \hline
6	&	Is long enough to ensure high recognition quality (at least three syllables)	&	``ok glass, make a video call''	&	``ok glass, hangout''	\\ \hline
7	&	Fits on a single line (less than 600px wide at 40px Roboto Thin)	&	``ok glass, add a calendar event''	&	``ok glass, create a new calendar event''	\\ \hline \pagebreak
8	&	Does not sound similar to existing commands	&		&	``ok glass, find a race'' (too similar to ``ok glass, find a place'')	\\ \hline
9	&	Does not require immediate interactivity in Mirror API Glassware. Immediate interactivity is only supported with GDK Glassware.	&	``ok glass, take a note'' (This allows users to speak a note and move on to their next task without worrying about a response from the Glassware.)	&	``ok glass, find a recipe'' (This requires a response from the Glassware so users can view the results. This is an acceptable GDK voice command but not acceptable for the Mirror API.)	\\ \hline
10	&	Has an imperative verb with an object	&	``ok glass, make a video call''	&``ok glass, video call''	\\ \hline
11	&	Uses articles when possible	&	``ok glass, record a video''	&	``ok glass, record video''	\\ \hline
12	&	Uses definite articles only when the object is definite	&	``ok glass, show me the weather''	& ``ok glass, take the picture''	\\ \hline
13	&	Uses ``this'' when there is only one relevant instance of the object	&	``ok glass, recognize this song''	&	``ok glass, recognize songs''	\\ \hline
14	&	Uses me and my when appropriate	&	``ok glass, show me the news''	&	``ok glass, show the news''	\\ \hline
15	&	Refers to Glass as the subject carrying out the action	&	``ok glass, start a run'' (Glass starts Glassware that tracks a run)	&	``ok glass, go running'' (The user is the one that actually goes running)	\\ \hline
		
		\end{tabularx}
	\end{table}
%\cleardoublepage


\section{Code}
\label{app:code}
\subsection{Google Glass application}
\label{app:code:ggApp}


\subsection{Smartphone application}
\label{app:code:spApp}

\cleardoublepage

\section{Project Specification (In Swedish)}
\label{app:projectspec}
\includepdfmerge{Google_Glass.pdf,-}

\end{document}
