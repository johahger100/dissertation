[TODO Inledande text]

\subsection{Text Length}
Since the Google Glass display is small and limited in space the amount of text that may by displayed on screen is as a result also limited. As such one interesting test case is to see where the limit in text lies. The test consists of trying different text lengths and reaching a conclusion on how much text may be displayed. The test also includes using different characters as different characters allocates different amounts of space.

\subsection{Image Size}
Similar to text images are also limited to the screen size. However, in terms of images there is a slightly different issue compared to text. Images may be resized to fit the screen. Is there a point where and image is no longer usable as details in the original resolution can no longer be spotted in the resized version? The test consits of using images with different original resolutions and comparing how well details are shown.

\subsection{Comparing Text and Images}
todo

\subsection{Download Speed}
The download speed is important as users might not want to wait too long for the application to load in the instructions after having scanned the QR code. As such the download speed will be measured for different amounts of data sized, on both the smartphone application as well as the Google Glass application. 

\subsection{Interaction Delay}
todo

\subsection{Background Noise}
todo

\subsection{Size of QR Code}
todo

\subsection{Complexity of QR Code}
todo

\subsection{``Tap Counter''}
The ``tap counting test'' simply consists of counting the amount of taps a user must perform in order to reach specific destinations. For instance, how many taps must the user perform in order to start the application?

\subsection{User Experience}
todo

\subsection{Multitasking}
todo

\subsection{Battery}
todo

\subsection{Connected to Mobile Device}
todo

\subsection{Overall Personal Opinions}
todo